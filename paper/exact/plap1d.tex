\documentclass[11pt]{amsart}
%prepared in AMSLaTeX, under LaTeX2e
\addtolength{\oddsidemargin}{-.9in} 
\addtolength{\evensidemargin}{-.9in}
\addtolength{\topmargin}{-.9in}
\addtolength{\textwidth}{1.5in}
\addtolength{\textheight}{1.5in}

\renewcommand{\baselinestretch}{1.05}

\usepackage{verbatim} % for "comment" environment

\usepackage{palatino}

\usepackage[final]{graphicx}

\usepackage{tikz}
\usetikzlibrary{positioning}

\usepackage{enumitem,xspace,fancyvrb}

\newtheorem*{thm}{Theorem}
\newtheorem*{defn}{Definition}
\newtheorem*{example}{Example}
\newtheorem*{problem}{Problem}
\newtheorem*{remark}{Remark}

\DefineVerbatimEnvironment{mVerb}{Verbatim}{numbersep=2mm,frame=lines,framerule=0.1mm,framesep=2mm,xleftmargin=4mm,fontsize=\footnotesize}

% macros
\usepackage{amssymb}
\newcommand{\bA}{\mathbf{A}}
\newcommand{\bB}{\mathbf{B}}
\newcommand{\bE}{\mathbf{E}}
\newcommand{\bF}{\mathbf{F}}
\newcommand{\bJ}{\mathbf{J}}

\newcommand{\bb}{\mathbf{b}}
\newcommand{\bi}{\mathbf{i}}
\newcommand{\bj}{\mathbf{j}}
\newcommand{\bk}{\mathbf{k}}
\newcommand{\br}{\mathbf{r}}
\newcommand{\bu}{\mathbf{u}}
\newcommand{\bv}{\mathbf{v}}
\newcommand{\bw}{\mathbf{w}}
\newcommand{\bx}{\mathbf{x}}

\newcommand{\ppr}[1]{\frac{\partial #1}{\partial r}}
\newcommand{\ppt}[1]{\frac{\partial #1}{\partial t}}
\newcommand{\ppx}[1]{\frac{\partial #1}{\partial x}}
\newcommand{\ppy}[1]{\frac{\partial #1}{\partial y}}
\newcommand{\ppz}[1]{\frac{\partial #1}{\partial z}}

\newcommand{\Div}{\ensuremath{\nabla\cdot}}
\newcommand{\Curl}{\ensuremath{\nabla\times}}

\newcommand{\eps}{\epsilon}
\newcommand{\grad}{\nabla}
\newcommand{\ip}[2]{\ensuremath{\left<#1,#2\right>}}
\newcommand{\lam}{\lambda}
\newcommand{\lap}{\triangle}

\newcommand{\RR}{\mathbb{R}}
\newcommand{\ZZ}{\mathbb{Z}}
\newcommand{\prob}[1]{\bigskip\noindent\textbf{#1.}\quad }

\newcommand{\Matlab}{\textsc{Matlab}\xspace}

\newcommand{\ds}{\displaystyle}

\begin{document}
\scriptsize \hfill \today

\Large
\bigskip
\centerline{\textbf{Exact solution to a 1D $p$-Laplacian obstacle problem}}
\bigskip

\normalsize

\thispagestyle{empty}

For the purposes of building a nonlinear variational inequality verification case, consider the $p$-Laplacian obstacle problem
\begin{align*}
-\Div(|\grad u|^{p-2} \grad u) &= f(x) \\
u(x) &\ge \psi(x).
\end{align*}
For $\Omega\subset \RR^d$ open, $p>1$, $f \in L^1(\Omega)$, and $\psi \in W^{1,p}(\Omega)$, the weak form of this problem is well-posed over $\mathcal{X} = W_0^{1,p}(\Omega)$ \cite{KinderlehrerStampacchia1980}.  The same problem is to minimize the coercive functional $J[u] = \int_\Omega \frac{1}{p} |\grad u|^p - f u$ over the closed and convex admissible set $\mathcal{K}=\{u \in \mathcal{X}\,:\, u \ge \psi\}$.

To be specific, for $p=1.5,2,4$ we will solve the one-dimensional case
\begin{align}
-\left(|u'(x)|^{p-2} u'(x)\right)' &= f(x) = \begin{cases} \alpha, & |x| < 1 \\ -\alpha, & |x| > 1\end{cases} \label{concrete} \\
u(x) &\ge 0 \notag
\end{align}
on the interval $\Omega = (-3,3) \subset \RR^1$.  Later, $\alpha>0$ will chosen for convenience.

By symmetry and well-posedness, $u(x)$ is an even function and $u'(0)=0$.  We will see that the solution satisfies $u'(x)\le 0$ on $0 \le x \le 3$.  Therefore, integrating \eqref{concrete} once gives
    $$\left(-u'(x)\right)^{p-1} = \int_0^x f(t)\,dt = \begin{cases} \alpha x, & |x| < 1 \\ \alpha (1 - FIXME), & |x| > 1\end{cases}$$
so 

\begin{thebibliography}{1}

\bibitem{KinderlehrerStampacchia1980}
{\sc D.~Kinderlehrer and G.~Stampacchia}, {\em An {I}ntroduction to
  {V}ariational {I}nequalities and their {A}pplications}, Academic Press, New
  York, 1980.

\end{thebibliography}
\end{document}
