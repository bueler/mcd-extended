\documentclass[11pt]{amsart}
%prepared in AMSLaTeX, under LaTeX2e
\addtolength{\oddsidemargin}{-.9in} 
\addtolength{\evensidemargin}{-.9in}
\addtolength{\topmargin}{-.9in}
\addtolength{\textwidth}{1.5in}
\addtolength{\textheight}{1.5in}

\renewcommand{\baselinestretch}{1.02}

\usepackage{verbatim} % for "comment" environment

\usepackage{palatino}

\usepackage[final]{graphicx}

\usepackage{tikz}
\usetikzlibrary{positioning}

\usepackage{enumitem,xspace,fancyvrb}

\newtheorem*{thm}{Theorem}
\newtheorem*{defn}{Definition}
\newtheorem*{example}{Example}
\newtheorem*{problem}{Problem}
\newtheorem*{remark}{Remark}

\DefineVerbatimEnvironment{mVerb}{Verbatim}{numbersep=2mm,frame=lines,framerule=0.1mm,framesep=2mm,xleftmargin=4mm,fontsize=\footnotesize}

% macros
\usepackage{amssymb}
\newcommand{\bA}{\mathbf{A}}
\newcommand{\bB}{\mathbf{B}}
\newcommand{\bE}{\mathbf{E}}
\newcommand{\bF}{\mathbf{F}}
\newcommand{\bJ}{\mathbf{J}}

\newcommand{\bb}{\mathbf{b}}
\newcommand{\bi}{\mathbf{i}}
\newcommand{\bj}{\mathbf{j}}
\newcommand{\bk}{\mathbf{k}}
\newcommand{\br}{\mathbf{r}}
\newcommand{\bu}{\mathbf{u}}
\newcommand{\bv}{\mathbf{v}}
\newcommand{\bw}{\mathbf{w}}
\newcommand{\bx}{\mathbf{x}}

\newcommand{\ppr}[1]{\frac{\partial #1}{\partial r}}
\newcommand{\ppt}[1]{\frac{\partial #1}{\partial t}}
\newcommand{\ppx}[1]{\frac{\partial #1}{\partial x}}
\newcommand{\ppy}[1]{\frac{\partial #1}{\partial y}}
\newcommand{\ppz}[1]{\frac{\partial #1}{\partial z}}

\newcommand{\Div}{\ensuremath{\nabla\cdot}}
\newcommand{\Curl}{\ensuremath{\nabla\times}}

\newcommand{\eps}{\epsilon}
\newcommand{\grad}{\nabla}
\newcommand{\ip}[2]{\ensuremath{\left<#1,#2\right>}}
\newcommand{\lam}{\lambda}
\newcommand{\lap}{\triangle}

\newcommand{\RR}{\mathbb{R}}
\newcommand{\ZZ}{\mathbb{Z}}
\newcommand{\prob}[1]{\bigskip\noindent\textbf{#1.}\quad }

\newcommand{\Matlab}{\textsc{Matlab}\xspace}

\newcommand{\ds}{\displaystyle}

\begin{document}
\scriptsize \hfill \today

\Large
\bigskip
\centerline{\textbf{Exact solution to a 1D $p$-Laplacian obstacle problem}}
\bigskip

\normalsize

\thispagestyle{empty}

For the purposes of building a nonlinear variational inequality verification case, consider the $p$-Laplacian obstacle problem
    $$-\Div(|\grad u|^{p-2} \grad u) = f(x), \qquad u(x) \ge \psi(x)$$
over $\Omega\subset \RR^d$ open, with $p>1$, $f \in L^1(\Omega)$, and $\psi \in W^{1,p}(\Omega)$.  The weak form of this problem is well-posed over $\mathcal{X} = W_0^{1,p}(\Omega)$ \cite{KinderlehrerStampacchia1980}.  Equivalently one may minimize the coercive functional $J[u] = \int_\Omega \frac{1}{p} |\grad u|^p - f u$ over the closed and convex admissible set $\mathcal{K}=\{u \in \mathcal{X}\,:\, u \ge \psi\}$.

We will solve the one-dimensional case
\begin{equation}
-\left(|u'(x)|^{p-2} u'(x)\right)' = f(x) = \begin{cases} \alpha, & |x| < 1 \\ -\alpha, & |x| > 1\end{cases} \label{concrete}
\end{equation}
on the interval $\Omega = (-3,3) \subset \RR^1$, with obstacle $\psi(x)=0$, for any $\alpha>0$.

By symmetry and well-posedness, $u(x)$ is an even function and $u'(0)=0$.  Restricting attention to $x\ge 0$, we will see that $u'(x)\le 0$.  Integrating \eqref{concrete} once gives
    $$\left(-u'(x)\right)^{p-1} = \int_0^x f(t)\,dt = \begin{cases} \alpha x, & 0 \le x \le 1 \\ \alpha (2 - x), & x \ge 1\end{cases}$$
The function on the right is continuous, and nonnegative for $0 \le x \le 2$.  Solving for $u'(x)$ by raising to the $(p-1)^{-1} = q-1$ power, where $\frac{1}{p} + \frac{1}{q} = 1$ defines $q$, gives
    $$u'(x) = \begin{cases} - \alpha^{q-1} x^{q-1}, & 0 \le x \le 1 \\ -\alpha^{q-1} (2 - x)^{q-1}, & x \ge 1\end{cases}$$
Integrating again, and recalling that $u(x)$ has even symmetry, we have
\begin{align*}
u(x) &= \begin{cases} u(0) - \frac{\alpha^{q-1}}{q} |x|^q, & |x| \le 1 \\
                      u(0) - \frac{\alpha^{q-1}}{q} -\alpha^{q-1} \int_1^{|x|} (2 - y)^{q-1}\,dy, & |x| \ge 1\end{cases} \\
     &= \begin{cases} u(0) - \frac{\alpha^{q-1}}{q} |x|^q, & |x| \le 1 \\
                      u(0) + \frac{\alpha^{q-1}}{q} \left( (2 - |x|)^q - 2 \right), & |x| \ge 1\end{cases}
\end{align*}

Now, where is the free boundary?  It must occur at $x=\xi>0$ where simultaneously $u(\xi)=0$ and $u'(\xi)=0$.  However, $u'(2)=0$, which reflects the symmetry that the integral of $f(x)$ on $[0,1]$ is cancelled by its integral on $[1,2]$, and $u'(x)$ is otherwise nonzero for $|x|>0$.  Thus the free boundary must be at $\xi=2$.  This determines $u(0)$:
    $$0=u(2) = u(0) - 2 \frac{\alpha^{q-1}}{q} \qquad \iff \qquad u(0) = 2 \frac{\alpha^{q-1}}{q}.$$
In conclusion, our exact solution is
\begin{equation}
\boxed{u(x) = \frac{\alpha^{q-1}}{q} \begin{cases} 2 - |x|^q, & |x| \le 1 \\
                                                   (2 - |x|)^q, & |x| \ge 1.\end{cases}}
\end{equation}

We will numerically test powers $p=3/2,2,4$, corresponding to $q=3,2,4/3$, respectively, with parameter $\alpha=1$ for convenience.

\begin{thebibliography}{1}

\bibitem{KinderlehrerStampacchia1980}
{\sc D.~Kinderlehrer and G.~Stampacchia}, {\em An {I}ntroduction to
  {V}ariational {I}nequalities and their {A}pplications}, Academic Press, New
  York, 1980.

\end{thebibliography}
\end{document}
