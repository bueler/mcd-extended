\documentclass[letterpaper,final,12pt,reqno]{amsart}

\usepackage[total={6.3in,9.2in},top=1.1in,left=1.1in]{geometry}

\usepackage{times,bm,bbm,empheq,fancyvrb,graphicx,amsthm}
\usepackage[dvipsnames]{xcolor}
\usepackage{longtable}
\usepackage{booktabs}

\usepackage[within=section]{newfloat}

\usepackage{tikz}
\usetikzlibrary{decorations.pathreplacing}

\usepackage[kw]{pseudo}
\pseudoset{left-margin=15mm,topsep=5mm,idfont=\texttt}

% hyperref should be the last package we load
\usepackage[pdftex,
colorlinks=true,
plainpages=false, % only if colorlinks=true
linkcolor=blue,   % ...
citecolor=Red,    % ...
urlcolor=black    % ...
]{hyperref}

\renewcommand{\baselinestretch}{1.05}

\allowdisplaybreaks[1]  % allow display breaks in align environments, if they avoid major underfulls

\newtheoremstyle{cstyle}% name
  {5pt}% space above
  {5pt}% space below
  {\itshape}% body font
  {}% indent amount
  {\itshape}% theorem head font
  {.}% punctuation after theorem head
  {.5em}% space after theorem head
  {\thmname{#1}\thmnumber{ #2}\thmnote{ (#3)}}% theorem head spec
\theoremstyle{cstyle}

\newtheorem{theorem}{Theorem}
\newtheorem{lemma}[theorem]{Lemma}

\newtheoremstyle{dstyle}% name
  {5pt}% space above
  {5pt}% space below
  {}%{\itshape}% body font
  {}% indent amount
  {\itshape}% theorem head font
  {.}% punctuation after theorem head
  {.5em}% space after theorem head
  {\thmname{#1}\thmnumber{ #2}\thmnote{ (#3)}}% theorem head spec
\theoremstyle{dstyle}

\newtheorem{definition}[theorem]{Definition}
\newtheorem{example}[theorem]{Example}

% numbering
\numberwithin{equation}{section}
\numberwithin{figure}{section}
\numberwithin{table}{section}
\numberwithin{theorem}{section}

\newcommand{\eps}{\epsilon}
\newcommand{\RR}{\mathbb{R}}

\newcommand{\grad}{\nabla}
\newcommand{\Div}{\nabla\cdot}
\newcommand{\trace}{\operatorname{tr}}

\newcommand{\hbn}{\hat{\mathbf{n}}}

\newcommand{\bb}{\mathbf{b}}
\newcommand{\be}{\mathbf{e}}
\newcommand{\bbf}{\mathbf{f}}
\newcommand{\bg}{\mathbf{g}}
\newcommand{\bn}{\mathbf{n}}
\newcommand{\br}{\mathbf{r}}
\newcommand{\bu}{\mathbf{u}}
\newcommand{\bv}{\mathbf{v}}
\newcommand{\bw}{\mathbf{w}}
\newcommand{\bx}{\mathbf{x}}
\newcommand{\bF}{\mathbf{F}}
\newcommand{\bV}{\mathbf{V}}
\newcommand{\bX}{\mathbf{X}}
\newcommand{\bxi}{\bm{\xi}}
\newcommand{\bzero}{\bm{0}}

\newcommand{\cK}{\mathcal{K}}
\newcommand{\cV}{\mathcal{V}}

\newcommand{\rhoi}{\rho_{\text{i}}}

\newcommand{\ip}[2]{\left<#1,#2\right>}

\newcommand{\mR}{R^{\bm{\oplus}}}
\newcommand{\iR}{R^{\bullet}}

\newcommand{\nn}{{\text{n}}}
\newcommand{\pp}{{\text{p}}}
\newcommand{\qq}{{\text{q}}}
\newcommand{\rr}{{\text{r}}}

\DeclareFloatingEnvironment[name=Pseudocode]{pcode}


\begin{document}
\title[On MCD methods for nonlinear and nonlocal VIs]{On multilevel constraint decomposition methods \\ for nonlinear and nonlocal variational inequalities}

\author{Ed Bueler}

\date{\today}

\begin{abstract} FIXME
\end{abstract}

\maketitle

%\tableofcontents

\thispagestyle{empty}
%\bigskip

\section{Introduction} \label{sec:intro}

FIXME the goal is to generalize and improve the method of \cite{Tai2003}

\section{$p$-coercive variational inequalities} \label{sec:vi}

Suppose $\cV$ is a real, reflexive Banach space with norm $\|\cdot\|$ and topological dual space $\cV'$.  Denote the dual pairing of $z \in \cV'$ and $v\in \cV$ by $\ip{z}{v} = z(v)$.  Note that $\|z\|_{\cV'} = \sup_{\|v\|=1} |\ip{z}{v}|$ defines the norm on $\cV'$, which is also a Banach space.

Let $\cK \subset \cV$ be a nonempty closed and convex subset of $\cV$.  We consider the following \emph{variational inequality} (VI) for $u\in \cK$:
\begin{equation}
\ip{f(u)}{v-u} \ge 0 \qquad \text{for all } v\in \cK. \label{eq:vi}
\end{equation}
Informally, conceptualizing the dual pairing as an inner product, \eqref{eq:vi} would say that the angle between $f(u)$ and any arbitrary vector $v-u$ pointing from $u$ into $\cK$ is at most $90^\circ$.  That is, $f(u)$ points straight into $\cK$ if $u$ solves \eqref{eq:vi}, and if $u$ is in the interior of $\cK$ then \eqref{eq:vi} further implies $f(u)=0$.  Thus such a VI generalizes a system of equations.

The following definitions \cite{KinderlehrerStampacchia1980} are standard.

\begin{definition} A map $f:\cK \to \cV'$ is \emph{monotone} if
\begin{equation}
\ip{f(u)-f(v)}{u-v} \ge 0 \qquad \text{for all } u,v \in \cK, \label{eq:monotone}
\end{equation}
\emph{strictly monotone} if equality in \eqref{eq:monotone} implies $u=v$, and \emph{coercive} if there exists $w \in \cK$ so that
\begin{equation}
\frac{\ip{f(u)-f(w)}{u-w}}{\|u-w\|} \to +\infty \qquad \text{as } \|u\|\to +\infty.
\end{equation}
\end{definition}

It is well-known that if $f:\cK \to \cV'$ is continuous, monotone, and coercive then VI \eqref{eq:vi} has a solution \cite[Corollary III.1.8]{KinderlehrerStampacchia1980}, and that the solution is unique if $f$ is strictly monotone.  As in the calculus of variations \cite{Evans2010}, coercivity permits a (weak) compactness argument even for unbounded sets $\cK$.  The condition of continuity can be weakened to continuity on finite-dimensional subspaces \cite{KinderlehrerStampacchia1980}, but the stronger condition will apply in our examples.

Monotone VI \eqref{eq:vi} generalizes the problem of minimizing convex objective functionals.  In fact, suppose $F:\cV \to \RR$ is lower semi-continuous and (G\^ateau) differentiable with continuous derivative $F':\cV \to \cV'$.  Then $F$ is convex if and only if $F'$ is monotone \cite[Proposition I.5.5]{EkelandTemam1976}.  Furthermore, if $F$ is convex and $\cK$ is convex and closed then VI \eqref{eq:vi} holds for $f=F'$ and $u\in \cK$ if and only if
\begin{equation}
u = \operatorname{arg-min}_{v\in\cK} F(v) \label{eq:minimization}
\end{equation}
\cite[Proposition II.2.1]{EkelandTemam1976}.  Note that \cite{Tai2003} addresses MCD convergence for problem \eqref{eq:minimization}.

However, not all VIs correspond to optimization problems.  For example, consider the following problem of porous-medium type.

\begin{example}  Suppose $\Omega \subset \RR^d$ is bounded and open with polygonal boundary.  Let $\cV = H_0^1(\Omega)$ and consider $\cK = \{v\ge 0\} \subset \cV$.  For $g\in \cV'$ and $u$ in $\cK$ let
\begin{equation}
\ip{f(u)}{v} = \int_\Omega (1+e^{-u}) \grad u \cdot \grad v\,dx - \ip{g}{v}. \label{eq:nongradientexample}
\end{equation}
It is easy to see that $|\ip{f(u)}{v}| \le (c \|u\|_{\cV} + \|g\|_{\cV'}) \|v\|_{\cV}$, thus that $f:\cK \to \cV'$ is continuous.  FIXME IS IT $2$-coercive?  FIXME $f\ne F'$
\end{example}

However, the VIs solved in this paper satisfy a stronger form of coercivity which will be important in proving convergence of MCD methods.

\begin{definition}  $f:\cK \to \cV'$ is \emph{$p$-coercive} if there exists $\kappa>0$ such that
\begin{equation}
\ip{f(u)-f(v)}{u-v} \ge \kappa \|u-v\|^p \qquad \text{for all } u,v \in \cK. \label{eq:pcoercive}
\end{equation}
(Note \cite{Tai2003} uses ``coercive'' for $2$-coercive.)
\end{definition}

It is easy to see that if $f$ is $p$-coercive for $p>1$ then it is monotone, strictly monotone, and coercive, and thus the following result, which states that we are addressing well-posed problems, is immediate.

\begin{theorem}  \label{thm:viwellposed}  If $f:\cK \to \cV'$ is continuous and $p$-coercive for $p>1$ then there exists a unique $u\in \cK$ so that VI \eqref{eq:vi} holds.
\end{theorem}

\section{Multilevel constraint decomposition algorithms} \label{sec:algorithms}

FIXME state the decomposition, the basis algorithm, the FAS version which has $O(m)$ residual evaluation complexity on each level

\section{Convergence of the basic algorithm} \label{sec:convergence}

FIXME

\section{Results} \label{sec:results}

FIXME


\small

\bigskip
\bibliography{mcd2}
\bibliographystyle{siam}

\end{document}
