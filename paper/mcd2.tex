\documentclass[letterpaper,final,12pt,reqno]{amsart}

\usepackage[total={6.3in,9.2in},top=1.1in,left=1.1in]{geometry}

\usepackage{times,bm,bbm,empheq,fancyvrb,graphicx,amsthm}
\usepackage[dvipsnames]{xcolor}
\usepackage{longtable}
\usepackage{booktabs}

\usepackage[within=section]{newfloat}

\usepackage{tikz}
\usetikzlibrary{decorations.pathreplacing}

\usepackage[kw]{pseudo}
\pseudoset{left-margin=15mm,topsep=5mm,idfont=\texttt}

% hyperref should be the last package we load
\usepackage[pdftex,
colorlinks=true,
plainpages=false, % only if colorlinks=true
linkcolor=blue,   % ...
citecolor=Red,    % ...
urlcolor=black    % ...
]{hyperref}

\renewcommand{\baselinestretch}{1.05}

\allowdisplaybreaks[1]  % allow display breaks in align environments, if they avoid major underfulls

\newtheoremstyle{cstyle}% name
  {5pt}% space above
  {5pt}% space below
  {\itshape}% body font
  {}% indent amount
  {\itshape}% theorem head font
  {.}% punctuation after theorem head
  {.5em}% space after theorem head
  {\thmname{#1}\thmnumber{ #2}\thmnote{ (#3)}}% theorem head spec
\theoremstyle{cstyle}

\newtheorem{theorem}{Theorem}
\newtheorem{lemma}[theorem]{Lemma}

\newtheoremstyle{dstyle}% name
  {5pt}% space above
  {5pt}% space below
  {}%{\itshape}% body font
  {}% indent amount
  {\itshape}% theorem head font
  {.}% punctuation after theorem head
  {.5em}% space after theorem head
  {\thmname{#1}\thmnumber{ #2}\thmnote{ (#3)}}% theorem head spec
\theoremstyle{dstyle}

\newtheorem{definition}[theorem]{Definition}
\newtheorem{example}[theorem]{Example}

% numbering
\numberwithin{equation}{section}
\numberwithin{figure}{section}
\numberwithin{table}{section}
\numberwithin{theorem}{section}

\newcommand{\eps}{\epsilon}
\newcommand{\RR}{\mathbb{R}}

\newcommand{\grad}{\nabla}
\newcommand{\Div}{\nabla\cdot}
\newcommand{\trace}{\operatorname{tr}}

\newcommand{\hbn}{\hat{\mathbf{n}}}

\newcommand{\bb}{\mathbf{b}}
\newcommand{\be}{\mathbf{e}}
\newcommand{\bbf}{\mathbf{f}}
\newcommand{\bg}{\mathbf{g}}
\newcommand{\bn}{\mathbf{n}}
\newcommand{\br}{\mathbf{r}}
\newcommand{\bu}{\mathbf{u}}
\newcommand{\bv}{\mathbf{v}}
\newcommand{\bw}{\mathbf{w}}
\newcommand{\bx}{\mathbf{x}}
\newcommand{\bF}{\mathbf{F}}
\newcommand{\bV}{\mathbf{V}}
\newcommand{\bX}{\mathbf{X}}
\newcommand{\bxi}{\bm{\xi}}
\newcommand{\bzero}{\bm{0}}

\newcommand{\cK}{\mathcal{K}}
\newcommand{\cV}{\mathcal{V}}

\newcommand{\rhoi}{\rho_{\text{i}}}

\newcommand{\ip}[2]{\left<#1,#2\right>}

\newcommand{\mR}{R^{\bm{\oplus}}}
\newcommand{\iR}{R^{\bullet}}

\newcommand{\nn}{{\text{n}}}
\newcommand{\pp}{{\text{p}}}
\newcommand{\qq}{{\text{q}}}
\newcommand{\rr}{{\text{r}}}

\newcommand{\supp}{\operatorname{supp}}

\DeclareFloatingEnvironment[name=Pseudocode]{pcode}


\begin{document}
\title[On MCD methods for nonlinear and nonlocal VIs]{On multilevel constraint decomposition methods \\ for nonlinear and nonlocal variational inequalities}

\author{Ed Bueler}

\date{\today}

\begin{abstract} FIXME
\end{abstract}

\maketitle

%\tableofcontents

\thispagestyle{empty}
%\bigskip

\section{Introduction} \label{sec:intro}

FIXME the goal is to generalize and improve the method of \cite{Tai2003}


\section{Coercive variational inequalities and constraint decomposition} \label{sec:vi}

Suppose $\cV$ is a real, reflexive Banach space with norm $\|v\|$ for $v\in\cV$ and topological dual space $\cV'$.  Denote the dual pairing of $\phi \in \cV'$ and $v$ by $\ip{\phi}{v} = \phi(v)$.  Note that $\|\phi\|_{\cV'} = \sup_{\|v\|=1} |\ip{\phi}{v}|$ defines the norm on $\cV'$, which is also a Banach space.

Let $\cK \subset \cV$ be a nonempty closed and convex subset of $\cV$, the \emph{constraint set}.  For continuous $f:\cK \to \cV'$ we consider the following \emph{variational inequality} (VI) for $u\in \cK$:
\begin{equation}
\ip{f(u)}{v-u} \ge 0 \qquad \text{for all } v\in \cK. \label{eq:vi}
\end{equation}
Informally, if we conceptualize the dual pairing as an inner product then \eqref{eq:vi} says that the angle between $f(u)$ and any arbitrary vector $v-u$ pointing from $u$ into $\cK$ is at most $90^\circ$, that is, $f(u)$ points straight into $\cK$ if $u$ solves \eqref{eq:vi}.  Furthermore, if $u$ is in the interior of $\cK$ then \eqref{eq:vi} implies $f(u)=0$, so VI \eqref{eq:vi} generalizes such an equation (system of equations).

The following definitions are standard \cite{KinderlehrerStampacchia1980}.

\begin{definition} A map $f:\cK \to \cV'$ is \emph{monotone} if
\begin{equation}
\ip{f(u)-f(v)}{u-v} \ge 0 \qquad \text{for all } u,v \in \cK, \label{eq:monotone}
\end{equation}
\emph{strictly monotone} if equality in \eqref{eq:monotone} implies $u=v$, and \emph{coercive} if there exists $w \in \cK$ so that
\begin{equation}
\frac{\ip{f(u)-f(w)}{u-w}}{\|u-w\|} \to +\infty \qquad \text{as } \|u\|\to +\infty. \label{eq:coercive}
\end{equation}
\end{definition}

It is well-known that if $f:\cK \to \cV'$ is continuous, monotone, and coercive then VI \eqref{eq:vi} has a solution \cite[Corollary III.1.8]{KinderlehrerStampacchia1980}, and also that the solution is unique if $f$ is strictly monotone.  As is standard in the calculus of variations \cite{Evans2010}, coercivity permits a (weak) compactness argument even for unbounded sets $\cK$.  The condition of continuity can be weakened to continuity on finite-dimensional subspaces \cite{KinderlehrerStampacchia1980}, but the stronger condition will apply in our examples.

When $f$ is monotone, VI \eqref{eq:vi} generalizes the problem of minimizing convex functions.  Suppose $F:\cV \to \RR$ is lower semi-continuous and (G\^ateau) differentiable with continuous derivative $F':\cV \to \cV'$.  Then $F$ is convex if and only if $F'$ is monotone \cite[Proposition I.5.5]{EkelandTemam1976}.  Furthermore, if $F$ is convex and $\cK$ is closed and convex then \eqref{eq:vi} holds for $f=F'$ if and only if
\begin{equation}
u = \operatorname{arg-min}_{v\in\cK} F(v) \label{eq:minimization}
\end{equation}
\cite[Proposition II.2.1]{EkelandTemam1976}.  Note that reference \cite{Tai2003} addresses problem \eqref{eq:minimization}.

However, not all VIs correspond to optimization.  For example, consider the following steady advection-diffusion problem over $H_0^1(\Omega)$.  (We denote Sobolev spaces by $W^{k,p}(\Omega)$ if $k\ge 0$ and $p\ge 1$ \cite{Evans2010}, and $W^{k,2}=H^k$ in particular.)

\begin{example}  \label{ex:advectiondiffusion}  Suppose $\Omega \subset \RR^d$ is bounded and open with polygonal boundary.  Let $\cV = H_0^1(\Omega)$ and consider $\cK = \{v\ge 0\} \subset \cV$.  For $g\in L^2(\Omega)$, $\bX \in L^\infty(\Omega)^d$, and $u$ in $\cK$ let
\begin{equation}
\ip{f(u)}{v} = \int_\Omega \grad u \cdot \grad v\,dx - (\bX \cdot \grad u) v - gv\,dx. \label{eq:nongradientexample}
\end{equation}
FIXME OR DO I WANT
\begin{equation}
\ip{f(u)}{v} = \int_\Omega \grad u \cdot \grad v\,dx - u (\bX \cdot \grad v) - gv\,dx. \label{eq:ALTnongradientexample}
\end{equation}
FIXME POINT OUT THAT IS STEADY ADVECTION-DIFFUSION VI; IF $\bX=0$ THEN IT IS OPTIMIZATION BUT OTHERWISE NOT; ALWAYS CONTINUOUS; IF $\bX$ IS SMALL THEN $2$-coercive; It is easy to see that $|\ip{f(u)}{v}| \le (c \|u\|_{\cV} + \|g\|_{\cV'}) \|v\|_{\cV}$, thus that $f:\cK \to \cV'$ is continuous.
\end{example}

Other examples of VI problems which are not of optimization type are found in ice sheet models \cite{Calvoetal2002,JouvetBueler2012} and other geophysical fluids problems \cite{Bueler2021conservation}.

The VIs solved in this paper satisfy a stronger form of coercivity than \eqref{eq:coercive}.  This stronger assumption will be used in proving convergence of MCD methods.

\begin{definition}  Let $p>1$.  The map $f:\cK \to \cV'$ is \emph{$p$-coercive} if there exists $\kappa>0$ such that
\begin{equation}
\ip{f(u)-f(v)}{u-v} \ge \kappa \|u-v\|^p \qquad \text{for all } u,v \in \cK. \label{eq:pcoercive}
\end{equation}
(Note \cite{Tai2003} uses ``coercive'' for $2$-coercive.)
\end{definition}

It is easy to see that if $f$ is $p$-coercive then it is monotone, strictly monotone, and coercive, and thus the following result, which states that the VI problems considered in this paper are well-posed, is immediate.

\begin{theorem}  \label{thm:viwellposed}  If $f:\cK \to \cV'$ is continuous and $p$-coercive (for $p>1$) then there exists a unique $u\in \cK$ so that VI \eqref{eq:vi} holds.
\end{theorem}

Our multilevel algorithms (next section) are based on both subspace and constraint decompositions, as follows.  Suppose there are $m<\infty$ subspaces $\cV_i \subset \cV$ so that
\begin{equation}
\cV = \sum_{i=1}^m \cV_i \label{eq:subspacedecomp}
\end{equation}
holds in the sense that if $w \in \cV$ then there exist $w_i \in \cV_i$ so that $w = \sum_i w_i$ \cite{Xu1992}.  Suppose further that $\cK_i \subset \cV$ are nonempty, closed, and convex subsets such that
\begin{equation}
\cK_i \subset \cV_i \qquad \text{and} \qquad \cK = \sum_{i=1}^m \cK_i. \label{eq:constraintdecomp}
\end{equation}
The constraint decomposition \cite{Tai2003} sum in \eqref{eq:constraintdecomp} is required to hold in two senses: \emph{(i)}~if $w \in \cK$ then there exist $w_i \in \cK_i$ so that $w = \sum_i w_i$, and \emph{(ii)}~if $z_i \in \cK_i$ for each $i$ then $\sum_i z_i \in \cK$.  (The latter sense is automatic in \eqref{eq:subspacedecomp} because $\cV_i$ are subspaces.)  Note that neither decomposition is unique (in general), and that $\cK_i \not\subset \cK$ in many applications.

Following \cite{Tai2003}, for each $\cK_i$ we also assume a (generally) nonlinear restriction operator $R_i : \cK \to \cK_i$ such that
\begin{equation}
v = \sum_{i=1}^m R_i v \qquad \text{if } v \in \cK.  \label{eq:constraintrestrictionsum}
\end{equation}
An \emph{constraint decomposition} of the constraint set $\cK$ is a choice of $\cV_i,\cK_i,R_i$ satisfying the above hypotheses.

In the next section we will introduce FE spaces and describe practical algorithms, but, as noted in the Introduction, constraint decompositions may be defined at the level of the continuum problem.  The following examples illustrate constraint decompositions of \emph{obstacle problems} \cite{GraeserKornhuber2009}; Example \ref{ex:domaindecomposition} is an overlapping domain decomposition and \ref{ex:frequencydecomposition} is a nonoverlapping frequency decomposition.  The latter example can be compared to the multilevel FE decomposition in the next section.

\begin{example}  \label{ex:domaindecomposition}  Consider a bounded domain $\Omega \subset \RR^d$ with polygonal boundary, let $\cV = H_0^k(\Omega)$, $k\ge 1$, and suppose the obstacle $\psi \in H^k(\Omega)$ satisfies $\psi|_{\partial \Omega} \le 0$.  Let $\cK = \{v \ge \psi\} \subset \cV$.  Suppose further that $\{\phi_i\}_{i=1}^m$ is a smooth partition of unity on $\Omega$, i.e.~so that $0 \le \phi_i\le 1$ and $\sum_i \phi_i = 1$, and let $\Omega_i = \supp \phi_i$ (open support).  Let $\cV_i = \{w \in \cV:w|_{\Omega \setminus \Omega_i} =0 \}$, $\cK_i = \{v \in V: v \ge \phi_i \psi\}$, and $R_i(v) = \max\{v,\phi_i \psi\}$.  Then \eqref{eq:subspacedecomp}, \eqref{eq:constraintdecomp}, and \eqref{eq:constraintrestrictionsum} all hold.
\end{example}

\begin{example}  \label{ex:frequencydecomposition}  For simplicity suppose $\Omega = (0,a)^d \subset \RR^d$ for $a>0$, a cube, and let $\cV = H_{\text{per}}^k(\Omega)$, $k\ge 1$, be the periodic functions.  Suppose $\psi \in H^k(\Omega)$ and let $\cK = \{v \ge \psi\} \subset \cV$.  Without getting into any detailed notation of frequency representation, but noting that the frequencies are discrete, suppose $\cV_i \subset \cV$ are $m<\infty$ subspaces from an (nonoverlapping) partition by frequency, thus satisfying \eqref{eq:subspacedecomp}.  Suppose $P_i:\cV \to \cV_i$ are the corresponding orthogonal projections, satisfying $I = \sum_i P_i$.  Let $\cK_i = \{v \ge P_i \psi\} \subset \cV$ and $R_i(v) = \max\{v,P_i \psi\}$.  Then \eqref{eq:constraintdecomp} and \eqref{eq:constraintrestrictionsum} hold.
\end{example}

In Example \ref{ex:domaindecomposition}, note that $\cK_i \not\subset \cK$ in many cases, for example if $\psi$ is positive over portions of $\Omega$ where the decomposition into overlapping subdomains $\Omega_i$ is nontrivial.  A similar comment applies in Example \ref{ex:frequencydecomposition}.


FIXME state the basic ``sequential constraint decomposition'' and ``parallel constraint decomposition'' algorithms

In practice the space $\cV$ will be a finite-dimensional finite element (FE) space because we will solve VI \eqref{eq:vi} over a mesh.  We FIXME triangulation etc. Though our 

\begin{example}  FIXME $\cV_i$ are 1d spaces on a single level; $\psi,\cK,\cK_i$ for obstacle problem; note $\cK_i \not\subset \cK$ when $\psi>0$
\end{example}


\section{Multilevel constraint decomposition algorithms} \label{sec:multilevel}

FIXME state essentially Algorithm 4.7 \cite{GraeserKornhuber2009} but with $\text{V}(\nu_1,\nu_2)$ cycles which works for linear; observe that up-smoothing is more efficient; state the FAS version which has $O(m)$ residual evaluation complexity on each level


\section{Convergence of the basic algorithm} \label{sec:convergence}

FIXME


\section{Results} \label{sec:results}

FIXME


\small

\bigskip
\bibliography{mcd2}
\bibliographystyle{siam}

\end{document}
