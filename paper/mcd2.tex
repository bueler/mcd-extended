\documentclass[letterpaper,final,12pt,reqno]{amsart}

\usepackage[total={6.3in,9.2in},top=1.1in,left=1.1in]{geometry}

\usepackage{times,bm,bbm,empheq,fancyvrb,graphicx,amsthm,amssymb}
\usepackage[dvipsnames]{xcolor}
\usepackage{longtable}
\usepackage{booktabs}

\usepackage{tikz}
\usetikzlibrary{decorations.pathreplacing}

\usepackage[kw]{pseudo}
\pseudoset{left-margin=15mm,topsep=5mm,idfont=\texttt}

\usepackage{float}

% hyperref should be the last package we load
\usepackage[pdftex,
colorlinks=true,
plainpages=false, % only if colorlinks=true
linkcolor=blue,   % ...
citecolor=Red,    % ...
urlcolor=black    % ...
]{hyperref}

\renewcommand{\baselinestretch}{1.05}

\allowdisplaybreaks[1]  % allow display breaks in align environments, if they avoid major underfulls

\newtheoremstyle{cstyle}% name
  {5pt}% space above
  {5pt}% space below
  {\itshape}% body font
  {}% indent amount
  {\itshape}% theorem head font
  {.}% punctuation after theorem head
  {.5em}% space after theorem head
  {\thmname{#1}\thmnumber{ #2}\thmnote{ (#3)}}% theorem head spec
\theoremstyle{cstyle}

\newtheorem{theorem}{Theorem}
\newtheorem{lemma}[theorem]{Lemma}
\newtheorem{assumptions}[theorem]{Assumptions}

\newtheoremstyle{cstyle*}% name
  {5pt}% space above
  {5pt}% space below
  {\itshape}% body font
  {}% indent amount
  {\itshape}% theorem head font
  {.}% punctuation after theorem head
  {.5em}% space after theorem head
  {\thmname{#1}}% theorem head spec
\theoremstyle{cstyle*}
\newtheorem{assumptions*}{Assumptions}

\newtheoremstyle{dstyle}% name
  {5pt}% space above
  {5pt}% space below
  {}%{\itshape}% body font
  {}% indent amount
  {\itshape}% theorem head font
  {.}% punctuation after theorem head
  {.5em}% space after theorem head
  {\thmname{#1}\thmnumber{ #2}\thmnote{ (#3)}}% theorem head spec
\theoremstyle{dstyle}

\newtheorem{definition}[theorem]{Definition}
\newtheorem{example}[theorem]{Example}

% numbering
\numberwithin{equation}{section}
\numberwithin{figure}{section}
\numberwithin{table}{section}
\numberwithin{theorem}{section}

\newcommand{\eps}{\epsilon}
\newcommand{\RR}{\mathbb{R}}

\newcommand{\grad}{\nabla}
\newcommand{\Div}{\nabla\cdot}
\newcommand{\trace}{\operatorname{tr}}

\newcommand{\hbn}{\hat{\mathbf{n}}}

\newcommand{\bb}{\mathbf{b}}
\newcommand{\be}{\mathbf{e}}
\newcommand{\bbf}{\mathbf{f}}
\newcommand{\bg}{\mathbf{g}}
\newcommand{\bn}{\mathbf{n}}
\newcommand{\br}{\mathbf{r}}
\newcommand{\bu}{\mathbf{u}}
\newcommand{\bv}{\mathbf{v}}
\newcommand{\bw}{\mathbf{w}}
\newcommand{\bx}{\mathbf{x}}
\newcommand{\bF}{\mathbf{F}}
\newcommand{\bV}{\mathbf{V}}
\newcommand{\bX}{\mathbf{X}}
\newcommand{\bxi}{\bm{\xi}}
\newcommand{\bzero}{\bm{0}}

\newcommand{\cK}{\mathcal{K}}
\newcommand{\cV}{\mathcal{V}}

\newcommand{\rhoi}{\rho_{\text{i}}}

\newcommand{\ip}[2]{\left<#1,#2\right>}

\newcommand{\mR}{R^{\bm{\oplus}}}
\newcommand{\iR}{R^{\bullet}}

\newcommand{\nn}{{\text{n}}}
\newcommand{\pp}{{\text{p}}}
\newcommand{\qq}{{\text{q}}}
\newcommand{\rr}{{\text{r}}}

\newcommand{\supp}{\operatorname{supp}}
\newcommand{\Span}{\operatorname{span}}


\begin{document}
\title[On multilevel constraint decomposition methods]{On multilevel constraint decomposition methods \\ for nonlinear variational inequalities}

\author{Ed Bueler}

\date{\today}

\begin{abstract} FIXME
\end{abstract}

\maketitle

%\tableofcontents

\thispagestyle{empty}
%\bigskip

\newfloat{pseudofloat}{t}{xyz}[section]
\floatname{pseudofloat}{Algorithm}


\section{Introduction} \label{sec:intro}

The goal of this paper is to generalize the constraint decomposition (CD) method of X.-C.~Tai \cite{Tai2003} to nonlinear VI problems to which it has not been previously applied.  The convergence of this method, for several types of decompositions, was proven by Tai for coercive variational inequality (VI) problems which arise from minimization of a convex functional over a convex set.  In its multilevel form the method has been shown to have optimal complexity for elliptic, linear obstacle problems \cite[Subsection 5.4]{Tai2003}; see also \cite[Theorem 4.6 and Algorithm 4.7]{GraeserKornhuber2009}.  We construct a new proof of convergence which removes the assumption that the problem arises from (constrained) optimization.  Furthermore we address its finite element (FE) implementation in nonlinear cases, and we demonstrate the resulting performance.

The algorithms and theory presented in \cite{Tai2003} are extended in four particular directions:
\renewcommand{\labelenumi}{\emph{(\roman{enumi})}}
\begin{enumerate}
\item We do not assume that the continuum VI problem arises from optimization of a scalar objective.  Nonetheless we can prove convergence in $H^1$ norm at the same rate as the original method converges in energy (Section \ref{sec:convergence}); % HOPE
examples are explored in Sections \ref{sec:vi} and \ref{sec:results}.

\item We make the observation that, if implemented appropriately, multilevel ``up-smoothing'' is intrinsically more efficient than ``down-smoothing'' (Section \ref{sec:multilevel}).  This observation seems to be new; compare the comments on V(1,0) and V(1,1) cycles in \cite{GraeserKornhuber2009,Tai2003}.  A strong preference for up-smoothing is, apparently, special to multilevel CD methods and does not arise in corresponding unconstrained problems.

\item We make the multilevel CD algorithm more practical by addressing efficient storage of intermediate quantities.  In Section \ref{sec:results} we show results from a full approximation storage (FAS; see \cite{Brandt1977}) implementation for nonlinear operators, one which avoids global (Newton) linearization; compare \cite{GraeserKornhuber2009}.

\item For problems of porous-media type the nonlinear operator $f$ is not coercive (Appendix A), so the convergence theory in \emph{(i)} does not apply.  However, a full-cycle implementation of the multilevel CD method is demonstrated numerically.  By ``freezing' the solution-dependent coefficient, the operator is approximated by a coercive operator, nonlinear in general, for the duration of the V-cycle.  The resulting scheme is highly-effective for doubly-nonlinear diffusion operators (Section \ref{sec:results}).
\end{enumerate}

The iterates from the CD method are always admissible, and thus the operator need only be defined for admissible states; Section \ref{sec:results} includes a nontrivial example.  Admissible-iterate methods should permit direct solutions of certain VI problems, such as fluid-layer dynamics problems \cite{Bueler2021conservation,JouvetBueler2012}, for which non-admissible methods, such as semi-smooth methods \cite{BensonMunson2006}, require unnatural modifications of the operator formula.  The full-cycle scheme in \emph{(iv)} above, which applies to non-coercive VIs, is also designed for these problems, which are characterized by the solution of an auxiliary PDE on a domain determined inside the VI residual evaluation.

Especially for the classical obstacle problem with a Laplacian operator, certain multilevel techniques are known to improve performance relative to the multilevel CD method \cite{GraeserKornhuber2009}.  These improved methods either track the active set in the discretization or modify the nodal basis functions, and in this sense they are discrete algorithms.  By contrast the CD method applies at the level of the continuum problem (Sections \ref{sec:cd} and \ref{sec:convergence}), and thus it is a more fundamental method than the truncated monotone multigrid method \cite{Kornhuber1994}, for example.  In any case, acceleration of our nonlinear multilevel CD method via active-set and/or basis-level manipulations represents a potential extension of the method here, though as a topic for future research.

% A BRIDGE TOO FAR:  In one example at the end of this paper (Section \ref{sec:resultsnonlocal}) we consider a nonlocal residual functional, that is, one which is not a partial differential operator.  Each evaluation of this functional requires the solution of a Stokes problem for a layer of fluid \nocite{SayagWorster2013} on a substrate (which forms the obstacle), and thus the corresponding FE operator discretization is also not sparse.  In this case we cannot prove coercivity but we nonetheless succeed in demonstrating near-optimal complexity of the Section \ref{sec:multilevel} multilevel CD algorithm in practice.


\section{Coercive variational inequalities} \label{sec:vi}

Suppose $\cV$ is a real, reflexive Banach space with norm $\|\cdot\|$ and topological dual space $\cV'$.  Denote the dual pairing of $\phi \in \cV'$ and $v\in\cV$ by $\ip{\phi}{v} = \phi(v)$, and note that $\|\phi\|_{\cV'} = \sup_{\|v\|=1} |\ip{\phi}{v}|$ defines a (Banach space) norm on $\cV'$.

Let $\cK \subset \cV$ be a nonempty closed and convex subset, the \emph{constraint set}; elements of $\cK$ are said to be \emph{admissible}.  For a continuous, but generally nonlinear, operator $f:\cK \to \cV'$ and \emph{source} $g\in \cV'$ we consider the following \emph{variational inequality} (VI) for the (exact) solution $u^*\in \cK$, if it exists:
\begin{equation}
\ip{f(u^*)}{v-u^*} \ge \ip{g}{v-u^*} \qquad \text{for all } v\in \cK. \label{eq:vi}
\end{equation}
Because $f$ is a (generally) nonlinear map, $g$ is not strictly needed when posing this VI, but its presence is helpful to the algorithms of Section \ref{sec:multilevel}.  (Note that by redefining $f$ we may take $g=0$.)

VI \eqref{eq:vi} generalizes the nonlinear system of equations $f(u^*)=g$ from a vector space $\cV$ to the constrained case (over $\cK$).  Informally, if we conceptualize the dual pairing as an inner product then \eqref{eq:vi} says that the angle between $f(u^*)-g$ and any arbitrary vector $v-u$ pointing from $u$ into $\cK$ is at most $90^\circ$.  That is, \eqref{eq:vi} says that $f(u^*)-g$ may not be zero but it points directly into $\cK$.  In particular, if $u^* \in \cK^\circ$ (interior) then \eqref{eq:vi} implies $f(u^*)=g$.  The following definitions are also standard \cite{KinderlehrerStampacchia1980}.

\begin{definition} A map $f:\cK \to \cV'$ is \emph{monotone} if
\begin{equation}
\ip{f(u)-f(v)}{u-v} \ge 0 \qquad \text{for all } u,v \in \cK, \label{eq:monotone}
\end{equation}
\emph{strictly monotone} if equality in \eqref{eq:monotone} implies $u=v$, and \emph{coercive} if there exists $w \in \cK$ so that
\begin{equation}
\frac{\ip{f(u)-f(w)}{u-w}}{\|u-w\|} \to +\infty \qquad \text{as } \|u\|\to +\infty. \label{eq:coercive}
\end{equation}
We say VI \eqref{eq:vi} is \emph{monotone} if $f$ is monotone, and likewise for strictly monotone and coercive. \end{definition}

It is well-known that if $f:\cK \to \cV'$ is continuous, monotone, and coercive then VI \eqref{eq:vi} has a solution \cite[Corollary III.1.8]{KinderlehrerStampacchia1980}, and also that the solution $u^* \in \cK$ is unique when $f$ is strictly monotone.  As in the calculus of variations \cite{Evans2010}, coercivity permits a compactness argument for unbounded sets $\cK$; recall that the bounded, closed subsets of a reflexive Banach space are weakly compact.  The condition of continuity can be weakened to only apply on finite-dimensional subspaces \cite{KinderlehrerStampacchia1980}, but the stronger condition will apply in our examples.

The coercive VIs solved in this paper satisfy a stronger inequality than \eqref{eq:coercive}, and thus they are well-posed.

\begin{definition}  Let $p>1$.  The map $f:\cK \to \cV'$ is \emph{$p$-coercive} if there exists $\kappa>0$ such that
\begin{equation}
\ip{f(u)-f(v)}{u-v} \ge \kappa \|u-v\|^p \qquad \text{for all } u,v \in \cK. \label{eq:pcoercive}
\end{equation}
\end{definition}

Note \cite{Tai2003} uses ``coercive'' for $2$-coercive.  It is easy to see that if $f$ is $p$-coercive then it is monotone, strictly monotone, and coercive, and thus the following result holds.

\begin{theorem}  \label{thm:viwellposed}  If $f:\cK \to \cV'$ is continuous and $p$-coercive (for $p>1$) then there exists a unique $u\in \cK$ solving VI \eqref{eq:vi}.
\end{theorem}

When $f$ is monotone, VI \eqref{eq:vi} generalizes the problem of minimizing a convex function over $\cK$.  Suppose $F:\cK \to \RR$ is lower semi-continuous and (G\^ateau) differentiable with continuous derivative $F':\cK \to \cV'$.  Then $F$ is convex if and only if $F'$ is monotone \cite[Proposition I.5.5]{EkelandTemam1976}.  Furthermore, Proposition II.2.1 in \cite{EkelandTemam1976} shows that if $F$ is convex then \eqref{eq:vi} holds for $f=F'$ and $g=0$ if and only if
\begin{equation}
u^* = \operatorname{arg-min}_{v\in\cK} F(v). \label{eq:minimization}
\end{equation}
The CD methods of Tai \cite{Tai2003} address problem \eqref{eq:minimization} under the hypothesis that $F'$ is coercive.

From now on $\Omega \subset \RR^d$ denotes a bounded, open set with smooth or piecewise-smooth (e.g.~polygonal) boundary.  Also, Sobolev spaces \cite{Evans2010} are denoted by $W^{k,p}(\Omega)$, for integer $k$ and $1\le p \le \infty$, with $W^{k,2}$ denoted by $H^k$.

The following example includes the classical obstacle problem for the linear Laplacian \cite{GraeserKornhuber2009} and the obstacle problem for the $p$-Laplacian \cite{ChoeLewis1991}. 

\begin{example}  \label{ex:plaplacian}  Suppose $a\in L^\infty(\Omega)$ such that $a(x)\ge a_0$ a.e.~for some constant $a_0>0$, and $p\ge 2$.  For $u,v \in \cV = W^{1,p}_0(\Omega)$ define $f:\cV \to \cV'$ by
\begin{equation}
\ip{f(u)}{v} = \int_\Omega a(x) |\grad u|^{p-2} \grad u \cdot \grad v\,dx. \label{eq:plaplacian}
\end{equation}
Now, if $x,y\in\RR^d$ then $(|x|^{p-2} x - |y|^{p-2} y)\cdot (x-y) \ge 2^{2-p} |x-y|^p$ \cite[see Appendix A and references therein]{Bueler2021conservation}.  Thus it follows from the Poincar\'e inequality that
    $$\ip{f(u) - f(v)}{u-v} \ge 2^{2-p} a_0 \|\grad u - \grad v\|_p^p \ge 2^{2-p} a_0 C \|u-v\|^p$$
for some $C>0$, and thus $f$ is $p$-coercive.  (The map in \eqref{eq:plaplacian} is coercive if $1<p<2$, but the proof is somewhat different \cite[Theorem 4.4]{Bueler2021conservation}.  In Section \ref{sec:results} we need only the $p\ge 2$ case.)  For $g\in\cV'$ define
    $$F(u) = \int_\Omega \frac{a(x)}{p} |\grad u|^p\,dx - g(u).$$
Then $F'(u) = f(u) - g$, $F$ is a convex functional (since $f$ is coercive), and, for any closed and convex $\cK\subset \cV$, VI problem \eqref{eq:vi} for is equivalent to optimization problem \eqref{eq:minimization}.
\end{example}

However, not all VI problems arise from optimization.  We give two such examples next, first a coercive and linear advection-diffusion problem, and then a nonlinear porous-medium-type problem.  Each of these examples is important in applications.  The first is preceded by a lemma.

\begin{lemma}  \label{lem:advectionskew}  \cite{Elmanetal2014}\,  Suppose $\bX :\Omega \to \RR^d$ is a smooth and bounded vector field on $\Omega$ with zero divergence ($\Div \bX=0$).  For $u,v \in H^1(\Omega)$ let $b(u,v) = \int_\Omega (\bX \cdot \grad u) v\,dx$.  Then $b(u,u) = \frac{1}{2} \int_{\partial \Omega} u^2 \bX\cdot \bn\,dx$ where $\bn$ is the outward normal on $\partial \Omega$.
\end{lemma}

\begin{proof}
Integration by parts gives $b(u,v) = - b(v,u) + \int_{\partial \Omega} uv \bX\cdot \bn\,dx$, so the result follows.
\end{proof}

\begin{example}  \label{ex:advectiondiffusion}  Suppose $\partial\Omega$ is partitioned into Dirichlet and Neumann portions, i.e.~$\partial\Omega = \partial_D\Omega \cup \partial_N\Omega$, with $\partial_D\Omega$ of positive measure.  Let $\cV = H_0^1(\Omega)$ be the space of functions with value zero on $\partial_D\Omega$.  (For simplicity we consider zero boundary data.)  Consider a smooth and bounded velocity field $\bX$ on $\Omega$ such that $\Div \bX=0$, but additionally assume that the flow is outward on the Neumann boundary, i.e.~$\bX \cdot \bn \ge 0$ at points of $\partial_N\Omega$.  For $u,v \in \cV = H_0^1(\Omega)$ and $\eps>0$ define
\begin{equation}
\ip{f(u)}{v} = \int_\Omega \eps \grad u \cdot \grad v - (\bX \cdot \grad u) v\,dx. \label{eq:advectiondiffusion}
\end{equation}
Consider VI \eqref{eq:vi} for any closed and convex $\cK \subset \cV$ and any $g\in\cV'$.  It is easy to see that $|\ip{f(u)}{v}| \le (\eps + \|\bX\|_\infty) \|u\| \|v\|$, thus that $f:\cK \to \cV'$ is continuous (and Lipschitz).  On the other hand, Lemma \ref{lem:advectionskew} says that $s(u,v)$ is skew-symmetric up to a nonnegative term.  However, by the outward flow assumption and the Poincar\'e inequality,
\begin{align*}
\ip{f(u)-f(v)}{u-v} &= \eps \int_\Omega |\grad u - \grad v|^2\,dx + b(u-v,u-v) \\
                    &= \eps \int_\Omega |\grad u - \grad v|^2\,dx + \frac{1}{2} \int_{\partial_N\Omega} (u-v)^2 \bX\cdot\bn \ge \eps C \|u-v\|^2.
\end{align*}
Thus $f$ is 2-coercive, and so VI problem \eqref{eq:vi} is well-posed for any closed and convex $\cK \subset \cV$ and $g\in \cV'$.
\end{example}

If $\bX \ne 0$ then VI \eqref{eq:vi} for $f$ in \eqref{eq:advectiondiffusion} is not a (constrained) minimization problem.  Indeed, the Jacobian of $f$, namely the bilinear form $J_f(u,v) = \int_\Omega \eps \grad u \cdot \grad v\,dx - b(u,v)$, is not symmetric in that case.\footnote{Assume $\bX \ne 0$ is continuous for simplicity.  For $u,v$ which are zero on $\partial \Omega$, $J_f(u,v) - J_f(v,u) = -2 b(u,v)$.  There exist $u,v$ so $b(u,v)$ is nonzero; construct $u,v$ locally near some point where $\bX$ is nonzero.}  Thus $f$ in \eqref{eq:advectiondiffusion} cannot be the Hessian of a scalar objective $F$.  References \cite{Bueler2021conservation,ChangNakshatrala2017} consider such advection-diffusion VI problems over $\cK = \{v\ge 0\}$.

\begin{example}  \label{ex:porous}  FIXME
\begin{equation}
\ip{f(u)}{v} = \int_\Omega \phi(u) \grad u \cdot \grad v\,dx \label{eq:porous}
\end{equation}
\end{example}

Numerical solver performance for Examples \ref{ex:plaplacian}, \ref{ex:advectiondiffusion}, and \ref{ex:porous} will be considered in Section \ref{sec:results}.  Further examples of nonlinear VI problems appear in ice sheet models \cite{Calvoetal2002,JouvetBueler2012} and other geophysical fluids \cite{Bueler2021conservation}.


\section{Constraint decomposition, the basic algorithm} \label{sec:cd}

Suppose there are $m<\infty$ closed subspaces $\cV_i \subset \cV$ so that the sum
\begin{equation}
\cV = \sum_{i=0}^{m-1} \cV_i \label{eq:subspacedecomp}
\end{equation}
holds in the sense that if $w \in \cV$ then there exist $w_i \in \cV_i$ so that $w = \sum_i w_i$; this is called a \emph{subspace decomposition} \cite{Xu1992}.  Suppose further that $\cK_i \subset \cV_i$ are nonempty, closed, and convex subsets such that
\begin{equation}
\cK = \sum_{i=0}^{m-1} \cK_i. \label{eq:constraintdecomp}
\end{equation}
The sum in \eqref{eq:constraintdecomp} is required to hold in two senses: \emph{(i)}~if $w \in \cK$ then there exist $w_i \in \cK_i$ so that $w = \sum_i w_i$, and \emph{(ii)}~if $z_i \in \cK_i$ for each $i$ then $\sum_i z_i \in \cK$.\footnote{Sense \emph{(ii)} for \eqref{eq:subspacedecomp} is automatic because the $\cV_i$ are subspaces.}

Note that neither decomposition \eqref{eq:subspacedecomp} or \eqref{eq:constraintdecomp} is required to be unique.  Also, $\cK_i \not\subset \cK$ in many applications; see the cartoon in Figure \ref{fig:cartoon}.

Finally, for each $\cK_i$ we assume that there are bounded, (generally) nonlinear restriction operators $R_i : \cK \to \cK_i$ such that if $v \in \cK$ then
\begin{equation}
v = \sum_{i=0}^{m-1} R_i v.  \label{eq:constraintrestrictionsum}
\end{equation}
A \emph{constraint decomposition} (CD) of $\cK$ is a choice of $\cV_i,\cK_i,R_i$ satisfying \eqref{eq:subspacedecomp}--\eqref{eq:constraintrestrictionsum} \cite{Tai2003}.

\begin{figure}[ht]
\includegraphics[width=0.55\textwidth]{genfigs/cartoon.pdf}
\caption{A constraint decomposition (CD) in a one-sided obstacle problem ($\mathcal{K}=\{v\ge \psi\}$) might look like this if $\mathcal{V}$ were 2-dimensional.}
\label{fig:cartoon}
\end{figure}

In Section \ref{sec:multilevel} we will introduce discretizations and describe practical algorithms, but the fundamental CD concept applies even at the level of the continuum problem.  The following two examples illustrate this for obstacle problems \cite{GraeserKornhuber2009}.  First we consider an overlapping domain decomposition.

\begin{example}  \label{ex:domaindecomposition}  Consider a bounded domain $\Omega \subset \RR^d$, let $\cV = W_0^{k,p}(\Omega)$ for $k\ge 0$ and $p\ge 1$, and suppose the obstacle $\psi \in W^{k,p}(\Omega)$ satisfies $\psi|_{\partial \Omega} \le 0$.  Let $\cK = \{v \ge \psi\} \subset \cV$.  Suppose further that $\{\phi_i\}_{i=0}^{m-1}$ is a smooth partition of unity on $\Omega$, satisfying $0 \le \phi_i\le 1$ and $\sum_i \phi_i = 1$, and let $\Omega_i$ be the support of $\phi_i$.  Let $\cV_i = \{w \in \cV:w|_{\Omega \setminus \Omega_i} =0 \}$, $\cK_i = \{v \in \cV_i: v \ge \phi_i \psi\}$, and $R_i(v) = \phi_i v$.  Then \eqref{eq:subspacedecomp}, \eqref{eq:constraintdecomp}, and \eqref{eq:constraintrestrictionsum} all hold.
\end{example}

Our second example is a disjoint frequency decomposition.  A multilevel FE CD, e.g.~the one proposed in Section \ref{sec:multilevel}, approximates such a frequency decomposition.

\begin{example}  \label{ex:frequencydecomposition}  For simplicity suppose $\Omega = (0,a)^d \subset \RR^d$ is a cube, and let $\cV = H_{\text{per}}^k(\Omega)$, $k\ge 0$, be the periodic functions.  Suppose $\psi \in H_{\text{per}}^k(\Omega)$ and let $\cK = \{v \ge \psi\} \subset \cV$.  Without using any detailed notation for Fourier representation, but noting that the frequencies are discrete, suppose $\{\cV_i\}$ are $m<\infty$ subspaces of $\cV$ defined by an (nonoverlapping) partition by frequency, thus satisfying \eqref{eq:subspacedecomp} as an orthogonal decomposition.  Suppose $P_i:\cV \to \cV_i$ are the corresponding orthogonal projections, satisfying $I = \sum_i P_i$.  Let $\cK_i = \{v \ge P_i \psi\} \subset \cV_i$ and $R_i = P_i$.  Then \eqref{eq:constraintdecomp} and \eqref{eq:constraintrestrictionsum} also hold.
\end{example}

Note that $\cK_i \not\subset \cK$ in most cases.  In Example \ref{ex:domaindecomposition}, if $\psi$ is positive over portions of $\Omega$ where the decomposition into overlapping subdomains $\Omega_i$ is nontrivial, then $\cK_i \not\subset \cK$, and similarly for Example \ref{ex:frequencydecomposition}.  Instead, the important inclusion is $\cK_i \subset \cV_i$.

Algorithm \ref{alg:basiccd} below states the basic CD algorithm as an iteration which solves smaller VI problems over each set $\cK_i$.  (By contrast, the algorithms in \cite{Tai2003} solve optimization problems over each $\cK_i$.)  The algorithm starts from a current iterate $u \in \cK$ and computes a new iterate $w\in\cK$, an improved approximation of the solution $u^* \in \cK$ of \eqref{eq:vi}, using a damping parameter $0<\alpha\le 1$.  There are parallel (additive) and successive (multiplicative) versions of the algorithm, generalizing the Jacobi and Gauss-Seidel iterations \cite{Greenbaum1997}, respectively.  For the parallel version the \textbf{for} loop can be computed in any order.

\begin{pseudofloat}[H]
\begin{pseudo*}
\pr{cditeration}(u\in\cK)\text{:} \\+
    if \pr{parallel}: \\+
        for $i \in \{0,\dots,m-1\}$: \\+
            $\hat w_i\in \cK_i$: \\+
                 $\boxed{\ip{f(u - R_i u + \hat w_i)}{v_i-\hat w_i} \ge g(v_i-\hat w_i)} \quad \forall v_i\in \cK_i$ \\---
    else: \\+
        for $i = 0,\dots,m-1$: \\+
            $\hat w_i\in \cK_i$: \\+
                $\displaystyle \boxed{\ip{f\Big(\sum_{j<i} w_j + \hat w_i + \sum_{j>i} R_j u\Big)}{v_i-\hat w_i} \ge g(v_i-\hat w_i)} \quad \forall v_i\in \cK_i$ \\-
            $w_i = (1-\alpha) R_i u + \alpha \hat w_i\in\cK_i$ \\--
    $\hat w = \sum_i \hat w_i\in\cK$ \\
    return $w=(1-\alpha) u + \alpha \hat w\in\cK$
\end{pseudo*}
\caption{The basic constraint decomposition (CD) algorithm for VI problem \eqref{eq:vi}.}
\label{alg:basiccd}
\end{pseudofloat}

The reader may confirm that, inside each boxed VI in Algorithm \ref{alg:basiccd}, the argument of $f$ is an element of $\cK$.  While the difference $v_i - \hat w_i \in \cV_i$ appears, note that by \eqref{eq:constraintdecomp} and \eqref{eq:constraintrestrictionsum} one may write it as a difference of admissible vectors (i.e.~from $\cK$), namely
\begin{align*}
[u - R_i u + v_i] - [u - R_i u + \hat w_i] &= v_i - \hat w_i, \label{eq:admissibledifference} \\
\left[\sum_{j<i} w_j + v_i + \sum_{j>i} R_j u\right] - \left[\sum_{j<i} w_j + \hat w_i + \sum_{j>i} R_j u\right] &= v_i - \hat w_i,  \notag
\end{align*}
for the two versions, respectively.

In other words, we solve VI \eqref{eq:vi} over perturbations from the sets $\cK_i$.  Define
\begin{equation}
e_i = \hat w_i - R_i u \in \cV_i \label{eq:ithupdate}
\end{equation}
as the $i$th update from a pass through either \textbf{for} loop in Algorithm \ref{alg:basiccd}.  Then $\hat w = u + \sum_{i} e_i$ and $w = u + \alpha \sum_i e_i$.  Easy identities $u - R_i u + \hat w_i = u + e_i$ and $\hat w = u^* + \sum_i \hat w_i - R_i u^*$ also hold.  We will use these identities in Section \ref{sec:convergence}.

The convergence results in Section \ref{sec:convergence} require substantial damping when applying the parallel algorithm, i.e.~$\alpha \le 1/m$, the same as in \cite{Tai2003}, but the successive version can be shown to converge without damping ($\alpha=1$).  In Section \ref{sec:results} we demonstrate practical convergence for a larger range of $\alpha$ than suggested by the theory.  % HOPE

In order for Algorithm \ref{alg:basiccd} to have a practical and efficient FE implementation (Section \ref{sec:multilevel}), we need additional notions when $f$ has any of the following properties: \emph{(i)} $f$ is defined only on $\cK$, or \emph{(ii)} $f$ is nonlinear, or \emph{(iii)} $f$ is non-local.  The issue of practical implementation for any such $f$ seems not to have been addressed in the CD literature, which is why we call Algorithm \ref{alg:basiccd} the ``basic'' algorithm.  In particular  \cite{GraeserKornhuber2009,Tai2003} only apply the CD method to the classical obstacle problem.  The following example explains the simplifications available in that easy case.

\begin{example}  \label{ex:fnice} Suppose $f:\cV \to \cV'$ is linear and defined on all of $\cV$.  Furthermore suppose $f$ is local in the sense that a basis $\{\phi_i\}$ of $\cV_i$ exists, with each support small in $\Omega$, such that $\ip{f(\phi_i)}{z}$ can be computed by an integral over the support of $\phi_i$.  (When $\cV$ is an FE space then this is the usual case for evaluating PDE weak forms over a basis of hat functions; see Section \ref{sec:multilevel}.  Constrast integral operators where $\ip{f(\phi_i)}{z}$ requires an integral over $\Omega$ even if the support of $\phi_i$ is small.)  Considering only the parallel version of Algorithm \ref{alg:basiccd} for simplicity, the boxed VI over $\cK_i$ can be written as
\begin{equation}
\ip{f(e_i)}{v_i-\hat w_i} \ge \tilde g(v_i-\hat w_i) \label{eq:linearlocalvi}
\end{equation}
where $\tilde g(z) = g(z) - \ip{f(u)}{z}$.  Noting $e_i = \hat w_i - R_i u \in \cV_i \notin \cK$, in general, VIs \eqref{eq:linearlocalvi} will only make sense because $f$ is defined over $\cV$.  Each problem \eqref{eq:linearlocalvi} can be solved using a stored residual $\ip{f(u)}{\cdot}$ (already included into the source term).  The VI \eqref{eq:vi} is then approximately solved in any incremental and efficient manner, by computations over the basis supports, during one application of \pr{cditeration}.
\end{example}

In other words, when $f$ has all the nice properties proposed in Example \ref{ex:fnice} then the decomposed VI problems can be computed from inexpensive data.  Note that the actual solution method for VI \eqref{eq:linearlocalvi} is not the concern here; we are observing that the data of problem \eqref{eq:linearlocalvi}, and the cost of residual evaluation, is small in practice.  A solver which would work for Example \ref{ex:fnice} can thus have implementation efficiencies which are unavailable in general.

Additional ideas, like those in Section \ref{sec:multilevel}, are needed for practical application of CD methods to more difficult problems.  In particular, we will extend the basic Algorithm \ref{alg:basiccd} to nonlinear $f$ by applying the full approximation storage (FAS) idea of Brandt \cite{Brandt1977}, in which case the decomposed problems only evaluate $f$ over (admissible) elements of $\mathcal{K}$, and we propose restrictions which allow inexpensive approximations of the decomposed problems.  However, extending the algorithm to non-local residual functionals is a topic for future research.


\section{Convergence of the basic algorithm} \label{sec:convergence}

We now seek a proof of the convergence of Algorithm \ref{alg:basiccd}.  This will be possible if we restrict to $2$-coercive operators and make certain assumptions as in \cite{Tai2003}.  First we define a new quantity related to VI problem \eqref{eq:vi}.

\begin{definition} Suppose $f:\cK \to \cV'$ and $g \in \cV'$.  For $u,v \in \cK$ let
\begin{equation}
  E(v,u) = \ip{f(v)}{v-u} - g(v-u).  \label{eq:normlikedefn}
\end{equation}
\end{definition}

If $u^*$ solves \eqref{eq:vi} then $E(v,u^*)$ is somewhat like an ``merit function,'' something used in the context of solving nonlinear equations \cite{NocedalWright2006} to replace an objective (scalar) functional.  (Note that $E(v,u^*)$ is \emph{not} a residual for \eqref{eq:vi}.)  The next lemma, which follows directly from $p$-coercivity, shows $E(v,u^*)$ is bounded below.

\begin{lemma} \label{lem:normlike}  Suppose $f:\mathcal{K} \to \mathcal{V}'$ is $p$-coercive and $u^* \in \mathcal{K}$ solves \eqref{eq:vi}.  For $v \in \mathcal{K}$,
\begin{equation}
  E(v,u^*) \ge \kappa \|v-u^*\|^p.  \label{eq:normlikebound}
\end{equation}
\end{lemma}

\begin{proof}
\begin{align*}
E(v,u^*) &= \ip{f(v)}{v-u^*} - \ip{f(u^*)}{v-u^*} + \ip{f(u^*)}{v-u^*} - g(v-u^*) \\
   &\ge \kappa \|v-u^*\|^p + 0.  \qedhere
\end{align*}
\end{proof}

Next we make two assumptions which are essentially the same as (7) and (8) in \cite{Tai2003}.

\begin{assumptions*}  There exists a constant $C_1>0$ so that
\begin{equation}
\left(\sum_{i=0}^{m-1} \|R_i u - R_i v\|^2\right)^{1/2} \le C_1 \|u-v\| \label{as:lipschitzrestrictions}
\end{equation}
for all $u,v\in\cK$.  Furthermore, there exists $C_2>0$ so that for all $v_i \in \cV_i$ and $y_j \in \cV_j$,
\begin{equation}
\sum_{i=0}^{m-1} \sum_{j=0}^{m-1} \left|\ip{f(w_{ij} + v_i) - f(w_{ij})}{y_j}\right| \le C_2 \left(\sum_{i=0}^{m-1} \|v_i\|^2\right)^{1/2} \left(\sum_{j=0}^{m-1} \|y_j\|^2\right)^{1/2} \label{as:lipschitzresidual}
\end{equation}
for all $w_{ij} \in \cK$ such that $w_{ij} + v_i \in \cK$ for all $i$ and $j$.
\end{assumptions*}

These inequalities might be described as ``totally Lipschitz'' requirements for the maps $R_i$ and $f$, respectively.  We make the following observations:
\begin{itemize}
\item Assumption \eqref{as:lipschitzrestrictions} addresses only the CD, and not the map $f$.
\item It is sometimes the case that \eqref{as:lipschitzresidual} addresses only a subspace decomposition property of $f$ (and not the CD itself).  For example, in the classical obstacle problem one may verify \eqref{as:lipschitzresidual} over all $w_{ij} \in \cV$.
\item The unconstrained versions of \eqref{as:lipschitzrestrictions} and \eqref{as:lipschitzresidual} are (13), (14) in \cite{TaiXu2002}, respectively.
\item If we assume $f$ itself is Lipschitz then the existence of $C_2$ is clear \cite{TaiXu2002}.  However, our bound for the convergence rate for Algorithm \ref{alg:basiccd} is improved when $C_1,C_2$ are made smaller.
\item Tai \cite{Tai2003} gives values of the constants $C_1$ in \eqref{as:lipschitzrestrictions} for obstacle problem constraint decompositions using $P_1$ FE spaces over shape-regular and quasi-uniform triangulations.  In particular, constants $C_1$ are known for overlapping domain decompositions (with or without an additional coarse mesh) and standard multilevel hierarchies.  Thus the cases we will demonstrate in Section \ref{sec:results} are already covered.
\item The constants $C_1,C_2$ generally depend on $m$.  However, regarding the convergence proof below, \cite{Tai2003} observes that if repeated application of Algorithm \ref{alg:basiccd} generates a bounded sequence of iterates in $\mathcal{K}$ then constants $C_1,C_2$ would be permitted to depend on $u,v,w_{ij},v_i,y_i$ as well.
\end{itemize}

From the above definitions and assumptions we have the following estimate for the parallel version of Algorithm \ref{alg:basiccd}.  Our proof follows \cite{Tai2003}.

\begin{lemma} \label{lem:core}  Suppose $f$ is $2$-coercive and $u^* \in \mathcal{K}$ solves \eqref{eq:vi}.  Recall $E(v,u^*)$ is defined in \eqref{eq:normlikedefn}.  Suppose $\mathcal{V}_i$, $\mathcal{K}_i$, $R_i$, and $f$ satisfy \eqref{as:lipschitzrestrictions} and \eqref{as:lipschitzresidual}.  Suppose $\hat w$ is computed from $u$ by the parallel version of Algorithm \ref{alg:basiccd}, and recall $e_i = \hat w_i - R_i u$.  Then
\begin{equation}
   E(\hat w,u^*) \le C_2 \sum_{i=0}^{m-1} \|e_i\|^2 + \kappa^{-1} C_1 C_2 \left(\sum_{i=0}^{m-1} \|e_i\|^2\right)^{1/2} E(u,u^*)^{1/2}. \label{eq:core}
\end{equation}
\end{lemma}

\begin{proof}  Recalling that $\hat w = u^* + \sum_i \hat w_i - R_i u^*$, by the (parallel version) boxed VI in Algorithm \ref{alg:basiccd} with $v_i = R_i u^*$ we get
\begin{align}
E(\hat w,u^*) &= \sum_{i=0}^{m-1} \ip{f(\hat w)}{\hat w_i - R_i u^*} - g(\hat w_i - R_i u^*) \label{eq:startcore} \\
    &\le \sum_{i=0}^{m-1} \ip{f(\hat w)}{\hat w_i - R_i u^*} + \ip{f(u + e_i)}{R_i u^* - \hat w_i} \notag \\
    &= \sum_{i=0}^{m-1} \ip{f(\hat w) - f(u + e_i)}{\hat w_i - R_i u^*}. \notag
\end{align}
For $i\in \{0,1,\dots,m-1\}$ use wrapped indices to define elements $\phi_{i,j}$:
\begin{align}
\phi_{i,0} &= u + e_i, \label{eq:gridcore} \\
\phi_{i,1} &= u + e_i + e_{i+1}, \notag \\
  &\vdots \notag \\
\phi_{i,m-1} &= u + e_i + e_{i+1} + \dots + e_{m-1} + e_0 + \dots + e_{i-1}, \notag
\end{align}
and $\phi_{i,j} \in \cK$ by \eqref{eq:constraintdecomp} and \eqref{eq:constraintrestrictionsum}.  Note that $\phi_{i,m-1} = u + \sum_i e_i = \hat w$.  From \eqref{eq:startcore}, apply a telescoping sum using the elements in \eqref{eq:gridcore}:
\begin{align}
E(\hat w,u^*) &\le \sum_{i=0}^{m-1} \sum_{j=1}^{m-1} \ip{f(\phi_{i,j}) - f(\phi_{i,j-1})}{\hat w_i - R_i u^*} \label{eq:nextcore} \\
  &\le \sum_{i=0}^{m-1} \sum_{j=0}^{m-1} \left|\ip{f(\phi_{i,j}) - f(\phi_{i,j-1})}{\hat w_i - R_i u^*}\right|. \notag
\end{align}
By an easy renumbering of the indices $i,j$ we may apply assumption \eqref{as:lipschitzresidual}, then the triangle inequality, and then assumption \eqref{as:lipschitzrestrictions}; we denote $\left(\sum_{i=0}^{m-1} \|e_i\|^2\right)^{1/2}$ by $Z$:
\begin{align}
E(\hat w,u^*) &\le C_2 Z \left(\sum_{i=0}^{m-1} \|\hat w_i - R_i u^*\|^2\right)^{1/2} = C_2 Z \left(\sum_{i=0}^{m-1} \|\hat w_i - R_i u + R_i u - R_i u^*\|^2\right)^{1/2} \label{eq:nextnextcore} \\
  &\le C_2 Z \left(Z + \left(\sum_{i=0}^{m-1} \|R_i u - R_i u^*\|^2\right)^{1/2}\right) \le C_2 Z \left(Z + C_1 \|u-u^*\|\right). \notag
\end{align}
Finally apply \eqref{eq:normlikebound} to give \eqref{eq:core}.
\end{proof}

FIXME NOW THE ENTIRE BATTLE IS TO GET A CONVEXITY RESULT $E(w,u) \ge C \sum_{j=0}^{m-1} \|e_j\|^2$ for some reasonable $C$

\section{Finite elements and multilevel constraint decomposition} \label{sec:multilevel}

In practice we will solve VI \eqref{eq:vi} over a finite-dimensional space $\cV$ based on a choice of a mesh over $\Omega$ and a finite element (FE) space.

FIXME triangulation

\begin{example}  FIXME if $\cV_i=\Span\{\phi_i\}$ for $\phi_i\in\cV$ are 1d spaces and if $\cK = \{v \ge \psi\} \subset \cV$ (obstacle problem), and if $R_i : \cK \to \cK_i$ for obstacle problem; note $\cK_i \not\subset \cK$ when $\psi>0$
\end{example}

FIXME state essentially Algorithm 4.7 \cite{GraeserKornhuber2009} but with $\text{V}(\nu_1,\nu_2)$ cycles which works for linear; observe that up-smoothing is more efficient; state the FAS version which has $O(m)$ residual evaluation complexity on each level


\section{Results for local variational inequalities} \label{sec:results}

FIXME


% A BRIDGE TOO FAR:  \section{Results for a nonlocal variational inequality} \label{sec:resultsnonlocal}



\small
\bibliography{mcd2}
\bibliographystyle{siam}

\normalsize
\appendix

\section{Porous-medium problems are never monotone}

Based on the examples in Section \ref{sec:vi}, the property of $p$-coercivity would seem to be a mild generalization of uniform ellipticity (including its quasi-linear meaning \cite[Section 8.3]{Evans2010}).  Indeed, weak-form operators with certain lower-order terms (Example \ref{ex:advectiondiffusion}) or degenerate quasi-linear type (Example \ref{ex:plaplacian}) can be monotone and coercive.  However, even the weaker property of monotonicity is ``fragile'' in a pointwise sense, as explained next, while ellipticity is more robust.  In particular, we show that no nontrivial porous-medium type functional is monotone, nor $p$-coercive.  We have not found any such explanation in the literature, but the observation of non-monotonicity in \cite{RocknerWang2008} is related.

Consider the expression $(Ax - By)\cdot (x-y)$ where $A,B \in \RR^{d\times d}$ are symmetric matrices and $x,y\in \RR^d$.  This expression is the quadratic form for a certain symmetric block matrix.  In fact, regarding $x,y$ as column vectors, we observe that
\begin{equation}
(Ax - By)\cdot (x-y) = \begin{bmatrix} x^\top & y^\top \end{bmatrix} \begin{bmatrix}
                        A & -\frac{1}{2}(A+B) \\
                        -\frac{1}{2}(A+B) & B \end{bmatrix} \begin{bmatrix} x \\ y \end{bmatrix}. \label{eq:blockmatrix}
\end{equation}
Let $M_{A,B}$ denote the symmetric $2d \times 2d$ matrix in \eqref{eq:blockmatrix}.  Clearly the expression is only bounded below if $M_{A,B}$ is nonnegative definite.  Stated as the following Lemma, the important point here is that exact equality $A=B$ is required for nonnegativity of the expression; knowing $A$ is close to $B$ is not helpful.

\begin{lemma}  Suppose $A,B \in \RR^{d\times d}$ are symmetric matrices which commute.  Then $(Ax - By)\cdot (x-y)\ge 0$ for all $x,y\in \RR^d$ if and only if $A=B$ is nonnegative definite. \end{lemma}

\newcommand{\sbvec}[2]{\left[\begin{smallmatrix} #1 \\ #2 \end{smallmatrix}\right]}
\newcommand{\sbmat}[4]{\left[\begin{smallmatrix} #1 & #2 \\ #3 & #4 \end{smallmatrix}\right]}

\begin{proof}
If $A=B$ is nonnegative definite then nonnegativity of the expression follows immediately.  Conversely, since $A$ is symmetric there exists an orthogonal basis $\{x_i\}$ of eigenvectors, with $Ax_i = \lambda_i x$.  By commutativity these are also eigenvectors of $B$, with $Bx_i = \mu_i x_i$.  Let $\big\{\sbvec{x_1}{x_1}$, $\sbvec{x_1}{-x_1}$, $\dots$, $\sbvec{x_d}{x_d}$, $\sbvec{x_d}{-x_d}\big\}$ be an ordered basis of $\RR^{2d}$.  It is easy to confirm that in this basis $M_{A,B}$ is block diagonal with $2\times 2$ diagonal blocks $\sbmat{0}{-\frac{1}{2}(\lambda_i+\mu_i)}{-\frac{1}{2}(\lambda_i+\mu_i)}{\lambda_i+\mu_i}$.  Thus the eigenvalues of $M_{A,B}$ come in pairs $\frac{1}{2}(\lambda_i + \mu_i) \pm \sqrt{\frac{1}{2}(\lambda_i^2 + \mu_i^2)}$ for $i=1,\dots,d$.  By the strict concavity of the square root, $\sqrt{\frac{1}{2}(\lambda_i^2 + \mu_i^2)} > \frac{1}{2}(|\lambda_i| + |\mu_i|)$ if $\lambda_i\ne \mu_i$.  Thus if any eigenvalues of $A$ differ from those of $B$, i.e.~$\lambda_i\ne \mu_i$ for any $i$, then there exists a negative eigenvalue of $M_{A,B}$.
\end{proof}

It is not clear if the hypotheses of symmetry and commutativity can be removed.  However, when the matrices are scalar multiples of the identity then the Lemma asserts that $(ax-by)\cdot(x-y) \ge 0$ for all $x,y\in\RR^d$ if and only if $a=b\ge 0$.  This scalar case suffices for the following result.

\begin{theorem}
Suppose $\phi:[0,\infty) \to (0,\infty)$ is continuous.  For $\Omega \subset \RR^d$ open and nonempty, $\cV = H_0^1(\Omega)$, and $g\in \cV'$, consider the following functional on $\cK = \{v\ge 0\} \subset \cV$:
\begin{equation}
\ip{f(u)}{v} = \int_\Omega \phi(u) \grad u\cdot \grad v\,dx - g(v).  \label{eq:porousagain}
\end{equation}
If $\phi$ is not constant then $f$ is not monotone.
\end{theorem}

\begin{proof}
Suppose $\phi(\alpha)\ne \phi(\beta)$ for $\alpha > 0$ and $\beta \ge 0$.  (That is, suppose $\phi$ is not constant.)  For $x\in\Omega$, construct $u\in \cK$ smooth so that $u(x)=\alpha$ and $\xi = \grad u(x) \ne 0$.  Using the Lemma, choose $\eta \in \RR^d$ so that $(\phi(\alpha) \xi - \phi(\beta) \eta) \cdot (\xi - \eta) < 0$.  FIXME NOT SURE IF THIS IS GOING TO WORK
\end{proof}

The main idea is that even the slightest variation in the coefficient $\phi(u)$ violates monotonicity.  A strong and tight ellipticity hypothesis, e.g.~the existence of $c_0>0$ and $\eps>0$ so that $c_0 \le \phi(u) \le (1+\eps)c_0$, will not imply monotonicity.  Contrast the nontrivial coefficient dependence on $\grad u$ in the $p$-coercive $p$-Laplacian Example \ref{ex:plaplacian} (with $p>2)$.

No nontrivial functional \eqref{eq:porousagain} is monotone, nor is it ever $p$-coercive for any $p>1$.
\end{document}

