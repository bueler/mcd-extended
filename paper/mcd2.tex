\documentclass[letterpaper,final,12pt,reqno]{amsart}

\usepackage[total={6.3in,9.2in},top=1.1in,left=1.1in]{geometry}

\usepackage{times,bm,bbm,empheq,fancyvrb,graphicx,amsthm,amssymb}
\usepackage[dvipsnames]{xcolor}
\usepackage{longtable}
\usepackage{booktabs}

\usepackage{tikz}
\usetikzlibrary{decorations.pathreplacing}

\usepackage[kw]{pseudo}
\pseudoset{left-margin=15mm,topsep=5mm,idfont=\texttt}

\usepackage{float}

% hyperref should be the last package we load
\usepackage[pdftex,
colorlinks=true,
plainpages=false, % only if colorlinks=true
linkcolor=blue,   % ...
citecolor=Red,    % ...
urlcolor=black    % ...
]{hyperref}

\renewcommand{\baselinestretch}{1.05}

\allowdisplaybreaks[1]  % allow display breaks in align environments, if they avoid major underfulls

\newtheoremstyle{cstyle}% name
  {5pt}% space above
  {5pt}% space below
  {\itshape}% body font
  {}% indent amount
  {\itshape}% theorem head font
  {.}% punctuation after theorem head
  {.5em}% space after theorem head
  {\thmname{#1}\thmnumber{ #2}\thmnote{ (#3)}}% theorem head spec
\theoremstyle{cstyle}

\newtheorem{theorem}{Theorem}
\newtheorem{lemma}[theorem]{Lemma}
\newtheorem{assumptions}[theorem]{Assumptions}

\newtheoremstyle{cstyle*}% name
  {5pt}% space above
  {5pt}% space below
  {\itshape}% body font
  {}% indent amount
  {\itshape}% theorem head font
  {.}% punctuation after theorem head
  {.5em}% space after theorem head
  {\thmname{#1}}% theorem head spec
\theoremstyle{cstyle*}
\newtheorem{assumptions*}{Assumptions}

\newtheoremstyle{dstyle}% name
  {5pt}% space above
  {5pt}% space below
  {}%{\itshape}% body font
  {}% indent amount
  {\itshape}% theorem head font
  {.}% punctuation after theorem head
  {.5em}% space after theorem head
  {\thmname{#1}\thmnumber{ #2}\thmnote{ (#3)}}% theorem head spec
\theoremstyle{dstyle}

\newtheorem{definition}[theorem]{Definition}
\newtheorem{example}[theorem]{Example}

% numbering
\numberwithin{equation}{section}
\numberwithin{figure}{section}
\numberwithin{table}{section}
\numberwithin{theorem}{section}

\newcommand{\eps}{\epsilon}
\newcommand{\RR}{\mathbb{R}}

\newcommand{\grad}{\nabla}
\newcommand{\Div}{\nabla\cdot}
\newcommand{\trace}{\operatorname{tr}}

\newcommand{\hbn}{\hat{\mathbf{n}}}

\newcommand{\bb}{\mathbf{b}}
\newcommand{\be}{\mathbf{e}}
\newcommand{\bbf}{\mathbf{f}}
\newcommand{\bg}{\mathbf{g}}
\newcommand{\bn}{\mathbf{n}}
\newcommand{\br}{\mathbf{r}}
\newcommand{\bu}{\mathbf{u}}
\newcommand{\bv}{\mathbf{v}}
\newcommand{\bw}{\mathbf{w}}
\newcommand{\bx}{\mathbf{x}}
\newcommand{\bF}{\mathbf{F}}
\newcommand{\bV}{\mathbf{V}}
\newcommand{\bX}{\mathbf{X}}
\newcommand{\bxi}{\bm{\xi}}
\newcommand{\bzero}{\bm{0}}

\newcommand{\cK}{\mathcal{K}}
\newcommand{\cV}{\mathcal{V}}

\newcommand{\rhoi}{\rho_{\text{i}}}

\newcommand{\ip}[2]{\left<#1,#2\right>}

\newcommand{\mR}{R^{\bm{\oplus}}}
\newcommand{\iR}{R^{\bullet}}

\newcommand{\nn}{{\text{n}}}
\newcommand{\pp}{{\text{p}}}
\newcommand{\qq}{{\text{q}}}
\newcommand{\rr}{{\text{r}}}

\newcommand{\supp}{\operatorname{supp}}
\newcommand{\Span}{\operatorname{span}}


\begin{document}
\title[On multilevel constraint decomposition methods]{On multilevel constraint decomposition methods \\ for nonlinear variational inequalities}

\author{Ed Bueler}

\date{\today}

\begin{abstract} FIXME
\end{abstract}

\maketitle

%\tableofcontents

\thispagestyle{empty}
%\bigskip

\newfloat{pseudofloat}{t}{xyz}[section]
\floatname{pseudofloat}{Algorithm}


\section{Introduction} \label{sec:intro}

The goal of this paper is to extend the constraint decomposition (CD) method of X.-C.~Tai \cite{Tai2003} to nonlinear VI problems to which it has not been applied.  The convergence of this method, for several types of decompositions, was proven by Tai for coercive variational inequality (VI) problems which arise from minimization of a convex functional over a convex set.  In its multilevel form the method has been shown to have optimal complexity for elliptic, linear obstacle problems \cite[Subsection 5.4]{Tai2003}; see also \cite[Theorem 4.6 and Algorithm 4.7]{GraeserKornhuber2009}.

We construct a new proof of convergence which removes the assumption that the problem arises from optimization.  Furthermore we address finite element (FE) implementation in nonlinear cases, and we demonstrate the resulting performance.  In fact, the theory presented in \cite{Tai2003} is extended in four particular directions:
\renewcommand{\labelenumi}{\emph{(\roman{enumi})}}
\begin{enumerate}
\item We do not assume that the continuum VI problem arises from optimization of a scalar objective; examples are explored in Sections \ref{sec:vi} and \ref{sec:results}.  We prove convergence in $H^1$ norm at the same rate as the original method converges in energy (Section \ref{sec:convergence}). % THIS IS THE HOPE

\item We make the observation that, if implemented appropriately, multilevel ``up-smoothing'' is intrinsically more efficient than ``down-smoothing'' (Section \ref{sec:multilevel}).  This observation seems to be new; compare the comments on V(1,0) and V(1,1) cycles in \cite{GraeserKornhuber2009,Tai2003}.  A strong preference for up-smoothing is, apparently, special to multilevel CD methods and does not arise in corresponding unconstrained problems.

\item We make the multilevel CD algorithm more practical by addressing storage of intermediate quantities.  In Section \ref{sec:results} we show results from a full approximation storage (FAS; see \cite{Brandt1977}) implementation for nonlinear operators, one which avoids global (Newton) linearization; compare \cite{GraeserKornhuber2009}.

\item For problems of porous-media type the nonlinear operator $f$ is not known to be coercive, so the convergence theory in \emph{(i)} does not apply.  However, a full-cycle implementation of the multilevel CD method is demonstrated numerically.  By ``freezing' the solution-dependent coefficient, the operator is approximated by a coercive operator, nonlinear in general, for the duration of the V-cycle.  The resulting scheme is highly-effective for doubly-nonlinear diffusion operators (Section \ref{sec:results}).
\end{enumerate}

The iterates from a CD method are always admissible, and thus the operator need only be defined for admissible states; Section \ref{sec:results} includes a nontrivial example.  Admissible-iterate methods should permit direct solutions of certain VI problems, such as fluid-layer dynamics problems \cite{Bueler2021conservation,JouvetBueler2012}, for which non-admissible methods, such as semi-smooth methods \cite{BensonMunson2006}, at least require unnatural modifications of the operator formula.  The full-cycle scheme in \emph{(iv)} above is also designed for these problems, which are characterized by the solution of an auxiliary PDE on a domain determined inside the VI residual evaluation.

For the classical obstacle problem with a Laplacian operator, certain multilevel techniques are known to improve performance relative to the multilevel CD method \cite{GraeserKornhuber2009}.  These improved methods either track the active set in the discretization or modify the nodal basis functions, and in this sense they are discrete algorithms.  By contrast the CD method applies at the level of the continuum problem (Sections \ref{sec:cd} and \ref{sec:convergence}); compare the truncated monotone multigrid method \cite{Kornhuber1994}, for example.  Acceleration of our nonlinear multilevel CD method via active-set and/or basis-level manipulations represents a potential extension of the method here, and is a topic for future research.

% A BRIDGE TOO FAR:  In one example at the end of this paper (Section \ref{sec:resultsnonlocal}) we consider a nonlocal residual functional, that is, one which is not a partial differential operator.  Each evaluation of this functional requires the solution of a Stokes problem for a layer of fluid \nocite{SayagWorster2013} on a substrate (which forms the obstacle), and thus the corresponding FE operator discretization is also not sparse.  In this case we cannot prove coercivity but we nonetheless succeed in demonstrating near-optimal complexity of the Section \ref{sec:multilevel} multilevel CD algorithm in practice.


\section{Coercive variational inequalities} \label{sec:vi}

Suppose $\cV$ is a real, reflexive Banach space with norm $\|\cdot\|$ and topological dual space $\cV'$.  Denote the dual pairing of $\phi \in \cV'$ and $v\in\cV$ by $\ip{\phi}{v} = \phi(v)$, and note that $\|\phi\|_{\cV'} = \sup_{\|v\|=1} |\ip{\phi}{v}|$ defines a (Banach space) norm on $\cV'$.

Let $\cK \subset \cV$ be a nonempty closed and convex subset, the \emph{constraint set}; elements of $\cK$ are said to be \emph{admissible}.  For a continuous, but generally nonlinear, operator $f:\cK \to \cV'$ and \emph{source functional} $g\in \cV'$ we consider the following \emph{variational inequality} (VI) for the (exact) solution $u^*\in \cK$, if it exists:
\begin{equation}
\ip{f(u^*)}{v-u^*} \ge \ip{g}{v-u^*} \qquad \text{for all } v\in \cK. \label{eq:vi}
\end{equation}
Because $f$ is (generally) nonlinear, $g$ is not strictly needed for posing this VI, and by redefining $f$ we may take $g=0$, but the presence of $g$ is helpful to algorithms in Section \ref{sec:multilevel}.

VI \eqref{eq:vi} generalizes the nonlinear system of equations $f(u^*)=g$ to the constraint set $\cK$.  Informally, if we conceptualize the dual pairing as an inner product then \eqref{eq:vi} says that the angle between $f(u^*)-g$ and any arbitrary vector $v-u$ pointing from $u$ into $\cK$ is at most $90^\circ$.  That is, \eqref{eq:vi} says that $f(u^*)-g$ may not be zero but it points directly into $\cK$.  In particular, if $u^*$ is in the interior $\cK^\circ$ then \eqref{eq:vi} implies $f(u^*)=g$.

\begin{definition} The following definitions are standard \cite{KinderlehrerStampacchia1980}.  A map $f:\cK \to \cV'$ is \emph{monotone} if
\begin{equation}
\ip{f(u)-f(v)}{u-v} \ge 0 \qquad \text{for all } u,v \in \cK, \label{eq:monotone}
\end{equation}
\emph{strictly monotone} if equality in \eqref{eq:monotone} implies $u=v$, and \emph{coercive} if there exists $w \in \cK$ so that
\begin{equation}
\frac{\ip{f(u)-f(w)}{u-w}}{\|u-w\|} \to +\infty \qquad \text{as } \|u\|\to +\infty. \label{eq:coercive}
\end{equation}
We say VI \eqref{eq:vi} is \emph{monotone} if $f$ is monotone, and likewise for strictly monotone and coercive. \end{definition}

It is well-known that if $f:\cK \to \cV'$ is continuous, monotone, and coercive then VI \eqref{eq:vi} has a solution \cite[Corollary III.1.8]{KinderlehrerStampacchia1980}, and also that the solution $u^* \in \cK$ is unique when $f$ is strictly monotone.  As in the calculus of variations \cite{Evans2010}, coercivity permits a compactness argument for unbounded sets $\cK$; recall that the bounded, closed subsets of a reflexive Banach space are weakly compact.  The condition of continuity can be weakened to only apply on finite-dimensional subspaces \cite{KinderlehrerStampacchia1980}, but the stronger condition will apply in our examples.

The coercive VIs solved in this paper satisfy a stronger inequality than \eqref{eq:coercive}, and thus they are well-posed.

\begin{definition}  Let $p>1$.  The map $f:\cK \to \cV'$ is \emph{$p$-coercive} if there exists $\kappa>0$ such that
\begin{equation}
\ip{f(u)-f(v)}{u-v} \ge \kappa \|u-v\|^p \qquad \text{for all } u,v \in \cK. \label{eq:pcoercive}
\end{equation}
\end{definition}

Note that Tai \cite{Tai2003} uses ``coercive'' for what we wil call $2$-coercive.  It is easy to see that if $f$ is $p$-coercive then it is monotone, strictly monotone, and coercive, and thus the following result holds.

\begin{theorem}  \label{thm:viwellposed}  If $f:\cK \to \cV'$ is continuous and $p$-coercive then there exists a unique $u^*\in \cK$ solving VI \eqref{eq:vi}.
\end{theorem}

When $f$ is monotone, \eqref{eq:vi} generalizes the problem of minimizing a convex function over $\cK$.  Suppose $F:\cK \to \RR$ is lower semi-continuous and (G\^ateau) differentiable with continuous derivative $F':\cK \to \cV'$.  Then $F$ is convex if and only if $F'$ is monotone \cite[Proposition I.5.5]{EkelandTemam1976}.  Furthermore, Proposition II.2.1 in \cite{EkelandTemam1976} shows that if $F$ is convex then \eqref{eq:vi} holds for $f=F'$ and $g=0$ if and only if
\begin{equation}
u^* = \operatorname{arg-min}_{v\in\cK} F(v). \label{eq:minimization}
\end{equation}
The CD methods of Tai \cite{Tai2003} address problem \eqref{eq:minimization} under the hypothesis that $F'$ is coercive.

From now on $\Omega \subset \RR^d$ denotes a bounded, open set with smooth or piecewise-smooth (e.g.~polygonal) boundary.  Sobolev spaces \cite{Evans2010} are denoted by $W^{k,p}(\Omega)$, for integer $k$ and $1\le p \le \infty$, with $W^{k,2}$ denoted $H^k$.  The following example includes the classical obstacle problem for the linear Laplacian \cite{GraeserKornhuber2009} and the $p$-Laplacian for $p\ge 2$ \cite{ChoeLewis1991}.

\begin{example}  \label{ex:plaplacian}  Suppose $a\in L^\infty(\Omega)$ such that $a(x)\ge a_0$ a.e.~for some constant $a_0>0$, and $p\ge 2$.  For $u,v \in \cV = W^{1,p}_0(\Omega)$ define $f:\cV \to \cV'$ by
\begin{equation}
\ip{f(u)}{v} = \int_\Omega a(x) |\grad u|^{p-2} \grad u \cdot \grad v\,dx. \label{eq:plaplacian}
\end{equation}
Now, if $x,y\in\RR^d$ then $(|x|^{p-2} x - |y|^{p-2} y)\cdot (x-y) \ge 2^{2-p} |x-y|^p$ \cite[see Appendix A and references therein]{Bueler2021conservation}.  Thus it follows from the Poincar\'e inequality that
    $$\ip{f(u) - f(v)}{u-v} \ge 2^{2-p} a_0 \|\grad u - \grad v\|_p^p \ge 2^{2-p} a_0 C \|u-v\|^p$$
for some $C>0$, and thus $f$ is $p$-coercive.  On the other hand, for $g\in\cV'$ define
    $$F(v) = \int_\Omega \frac{a(x)}{p} |\grad v|^p\,dx - \ip{g}{v}.$$
Then $F'(v) = f(v) - g$, $F$ is a convex functional (since $f$ is coercive), and, for any closed and convex $\cK\subset \cV$, VI problem \eqref{eq:vi} for is equivalent to optimization problem \eqref{eq:minimization}.\end{example}

The map in \eqref{eq:plaplacian} is coercive if $1<p<2$, but the proof is somewhat different \cite[Theorem 4.4]{Bueler2021conservation}.  In Section \ref{sec:results} we need only the $p\ge 2$ case.

Not all VI problems arise as in Example \ref{ex:plaplacian}, from optimization.  We give two such examples next, first a coercive and linear advection-diffusion problem, and then a nonlinear porous-medium-type problem; each is important in applications.  The first is preceded by a lemma.

\begin{lemma}  \label{lem:advectionskew}  \cite{Elmanetal2014}\,  Suppose $\bX :\Omega \to \RR^d$ is a bounded and boundedly-differentiable vector field on $\Omega$ with zero divergence ($\Div \bX=0$).  For $u,v \in H^1(\Omega)$ let $b(u,v) = \int_\Omega (\bX \cdot \grad u) v\,dx$.  Then $b(u,u) = \frac{1}{2} \int_{\partial \Omega} u^2 \bX\cdot \bn\,dx$ where $\bn$ is the outward normal on $\partial \Omega$.
\end{lemma}

\begin{proof}
Integration by parts gives $b(u,v) = - b(v,u) + \int_{\partial \Omega} uv \bX\cdot \bn\,dx$, so the result follows.
\end{proof}

\begin{example}  \label{ex:advectiondiffusion}  Suppose $\partial\Omega$ is partitioned into Dirichlet and Neumann portions, i.e.~$\partial\Omega = \partial_D\Omega \cup \partial_N\Omega$, with $\partial_D\Omega$ of positive measure.  Let $\cV = H_0^1(\Omega)$ be the space of functions which are zero along $\partial_D\Omega$.  Consider a divergence-free velocity field $\bX$ on $\Omega$ satisfying the conditions of Lemma \ref{lem:advectionskew}, but additionally assume that the flow is outward on the Neumann boundary, $\bX \cdot \bn \ge 0$ on $\partial_N\Omega$.  For $u,v \in \cV = H_0^1(\Omega)$ and $\eps>0$ define
\begin{equation}
\ip{f(u)}{v} = \eps \left(\grad u, \grad v\right)_{L^2(\Omega)} - b(u,v). \label{eq:advectiondiffusion}
\end{equation}
Consider VI \eqref{eq:vi} for any closed and convex $\cK \subset \cV$ and $g\in\cV'$.  It is easy to see that $|\ip{f(u)}{v}| \le (\eps + \|\bX\|_\infty) \|u\| \|v\|$, thus that $f:\cK \to \cV'$ is continuous.  Lemma \ref{lem:advectionskew} says that the bilinear form $s(u,v)$ is skew-symmetric up to a nonnegative term.  By the outward flow assumption and the Poincar\'e inequality,
\begin{align*}
\ip{f(u)-f(v)}{u-v} &= \eps \int_\Omega |\grad u - \grad v|^2\,dx + b(u-v,u-v) \\
                    &= \eps \int_\Omega |\grad u - \grad v|^2\,dx + \frac{1}{2} \int_{\partial_N\Omega} (u-v)^2 \bX\cdot\bn \ge \eps C \|u-v\|^2.
\end{align*}
Thus $f$ is 2-coercive, and so VI problem \eqref{eq:vi} is well-posed.
\end{example}

References \cite{Bueler2021conservation,ChangNakshatrala2017} consider advection-diffusion VI problems like Example \ref{ex:advectiondiffusion}, specifically over the set $\cK = \{v\ge 0\}$.  If $\bX \ne 0$ then VI \eqref{eq:vi} for $f$ in \eqref{eq:advectiondiffusion} does not correspond to a minimization problem.  Indeed, $\ip{f(u)}{v}$ is not symmetric in that case,\footnote{Assume $\bX \ne 0$ is continuous for simplicity.  For $u,v$ which are zero on $\partial \Omega$, note $\ip{f(u)}{v} - \ip{f(v)}{u} = -2 b(u,v)$.  By constructing $u,v$ locally near some point where $\bX$ is nonzero. one may show $b(u,v)\ne 0$.} so $f$ cannot be the gradient of a scalar objective.

\begin{example}  \label{ex:porous}  Suppose $\phi:[0,\infty) \to [\phi_0,\infty)$ is continuous for some constant $\phi_0>0$.  Then the following operator, of porous medium type, is not known to be monotone (or coercive) when $\phi$ is not constant:
\begin{equation}
\ip{f(u)}{v} = \int_\Omega \phi(u) \grad u \cdot \grad v\,dx. \label{eq:porous}
\end{equation}
As with Example \ref{ex:advectiondiffusion}, this form does not have the symmetry necessary to be the gradient of a scalar objective.
\end{example}

Numerical solver performance for Examples \ref{ex:plaplacian}, \ref{ex:advectiondiffusion}, and \ref{ex:porous} will be considered in Section \ref{sec:results}.  Further examples of nonlinear VI problems appear in ice sheet models and other geophysical fluids \cite{Bueler2021conservation,Calvoetal2002,JouvetBueler2012}.


\section{Constraint decomposition, the basic algorithm} \label{sec:cd}

Suppose there are $m<\infty$ closed subspaces $\cV_i \subset \cV$ so that the sum
\begin{equation}
\cV = \sum_{i=0}^{m-1} \cV_i \label{eq:subspacedecomp}
\end{equation}
holds in the sense that if $w \in \cV$ then there exist $w_i \in \cV_i$ so that $w = \sum_i w_i$.  Equation \eqref{eq:subspacedecomp} is called a \emph{subspace decomposition} \cite{Xu1992}.  Suppose further that $\cK_i \subset \cV_i$ are nonempty, closed, and convex subsets such that
\begin{equation}
\cK = \sum_{i=0}^{m-1} \cK_i. \label{eq:constraintdecomp}
\end{equation}
The sum in \eqref{eq:constraintdecomp} is required to hold in two senses \cite{TaiTseng2002}: \emph{(i)}~if $w \in \cK$ then there exist $w_i \in \cK_i$ so that $w = \sum_i w_i$, and \emph{(ii)}~if $z_i \in \cK_i$ for each $i$ then $\sum_i z_i \in \cK$.  (Sense \emph{(ii)} is automatic for \eqref{eq:subspacedecomp} because the $\cV_i$ are subspaces.)  Note that neither decomposition \eqref{eq:subspacedecomp} or \eqref{eq:constraintdecomp} is required to be unique.  Also, $\cK_i \not\subset \cK$ in many applications; see the cartoon in Figure \ref{fig:cartoon}.

\begin{figure}[ht]
\includegraphics[width=0.55\textwidth]{genfigs/cartoon.pdf}
\caption{Suppose $\mathcal{V}$ is the space of real functions on a two-point set $\Omega=\{x_1,x_2\}$.  A constraint decomposition (CD) for a one-sided obstacle problem with $\mathcal{K}=\{v\ge \psi\}$ looks like this.}
\label{fig:cartoon}
\end{figure}

For each $\cK_i$ we will also assume that there are bounded, generally-nonlinear restriction operators $R_i : \cK \to \cK_i$ such that if $v \in \cK$ then
\begin{equation}
v = \sum_{i=0}^{m-1} R_i v;  \label{eq:constraintrestrictionsum}
\end{equation}
see Figure \ref{fig:cartoon}.  Clearly \eqref{eq:constraintrestrictionsum} implies sense \emph{(i)} for \eqref{eq:constraintdecomp}. A \emph{constraint decomposition} (CD) of $\cK$ is a choice of $\cV_i,\cK_i,R_i$ satisfying \eqref{eq:subspacedecomp}--\eqref{eq:constraintrestrictionsum} \cite{Tai2003}.

In Section \ref{sec:multilevel} we will introduce discretizations and practical algorithms, but we observe here that the CD concept applies at the level of the continuum problem.  The following two examples illustrate this for obstacle problems \cite{GraeserKornhuber2009}.

\begin{example}  \label{ex:domaindecomposition}  Consider an overlapping domain decomposition as follows.  For a bounded domain $\Omega \subset \RR^d$, let $\cV = W_0^{k,p}(\Omega)$ for $k\ge 0$ and $p\ge 1$, and suppose the obstacle $\psi \in W^{k,p}(\Omega)$ satisfies $\psi|_{\partial \Omega} \le 0$.  Let $\cK = \{v \ge \psi\} \subset \cV$.  Suppose further that $\{\phi_i\}_{i=0}^{m-1}$ is a smooth partition of unity on $\Omega$, satisfying $0 \le \phi_i\le 1$ and $\sum_i \phi_i = 1$, and let $\Omega_i$ be the support of $\phi_i$.  Let $\cV_i = \{w \in \cV:w|_{\Omega \setminus \Omega_i} =0 \}$, $\cK_i = \{v \in \cV_i: v \ge \phi_i \psi\}$, and $R_i(v) = \phi_i v$.  Then \eqref{eq:subspacedecomp}, \eqref{eq:constraintdecomp}, and \eqref{eq:constraintrestrictionsum} all hold.
\end{example}

Our second example is a disjoint frequency decomposition.  A multilevel FE CD, e.g.~the one proposed in Section \ref{sec:multilevel}, approximates such a frequency decomposition.

\begin{example}  \label{ex:frequencydecomposition}  For simplicity suppose $\Omega = (0,a)^d \subset \RR^d$ is a cube, and let $\cV = H_{\text{per}}^k(\Omega)$, $k\ge 0$, be the periodic functions.  Suppose $\psi \in H_{\text{per}}^k(\Omega)$ and let $\cK = \{v \ge \psi\} \subset \cV$.  Without using any detailed notation for Fourier representation, but noting that the frequencies are discrete, suppose $\{\cV_i\}$ are $m<\infty$ subspaces of $\cV$ defined by an (nonoverlapping) partition by frequency, thus satisfying \eqref{eq:subspacedecomp} as an orthogonal decomposition.  Suppose $P_i:\cV \to \cV_i$ are the corresponding orthogonal projections, satisfying $I = \sum_i P_i$.  Let $\cK_i = \{v \ge P_i \psi\} \subset \cV_i$ and $R_i = P_i$.  Then \eqref{eq:constraintdecomp} and \eqref{eq:constraintrestrictionsum} also hold.
\end{example}

Note that $\cK_i \not\subset \cK$ in many cases.  Specifically in Example \ref{ex:domaindecomposition}, if $\psi$ is positive over portions of $\Omega$ where the decomposition into overlapping subdomains $\Omega_i$ is nontrivial, then $\cK_i \not\subset \cK$, and similarly for Example \ref{ex:frequencydecomposition}.  The important inclusion is $\cK_i \subset \cV_i$.

Algorithms \ref{alg:basiccd-add} and \ref{alg:basiccd-mult} below state the basic CD method as iterations which improve a current iterate $u \in \cK$ by solving ``smaller'' VI problems over each subset $\cK_i$.  The output $w\in\cK$ should be closer to the solution $u^* \in \cK$ of \eqref{eq:vi}.

\begin{pseudofloat}[H]
\begin{pseudo*}
\pr{cd-add}(u\in\cK)\text{:} \\+
    for $i \in \{0,\dots,m-1\}$: \\+
        \rm{find} $\hat w_i\in \cK_i$ \rm{so that for all} $v_i\in \cK_i$ \rm{we have} \\+
            $\boxed{{\large \strut} \ip{f(u - R_i u + \hat w_i)}{v_i-\hat w_i} \ge g(v_i-\hat w_i)}$ \\--
    $\hat w = \sum_i \hat w_i\in\cK$ \\
    return $w=(1-\alpha) u + \alpha \hat w\in\cK$
\end{pseudo*}
\caption{Additive constraint decomposition for VI problem \eqref{eq:vi}.}
\label{alg:basiccd-add}
\end{pseudofloat}

The \textbf{for} loop in Algorithm \ref{alg:basiccd-add}, the additive or parallel CD method, can be computed in any order.  Note that $u-R_iu+\hat w_i$ involves removing the part of $u$ which lies in $\mathcal{K}_i$ before adding-back the improved subset solution $\hat w_i$.

The following multiplicative (successive) version orders the sets $\mathcal{K}_i$, and each subset solution updates the global iterate.  While Algorithm \ref{alg:basiccd-add} generalizes a Jacobi iteration, this one is like Gauss-Seidel.  The sum $\sum_{j>i} R_j u$ should be read as ``keep the parts of the current iterate which we have not yet improved.''

\begin{pseudofloat}[H]
\begin{pseudo*}
\pr{cd-mult}(u\in\cK)\text{:} \\+
    for $i = 0,\dots,m-1$: \\+
        \rm{find} $\hat w_i\in \cK_i$ \rm{so that for all} $v_i\in \cK_i$ \rm{we have} \\+
            $\displaystyle \boxed{\ip{f\Big(\sum_{j<i} w_j + \hat w_i + \sum_{j>i} R_j u\Big)}{v_i-\hat w_i} \ge g(v_i-\hat w_i)}$ \\-
            $w_i = (1-\alpha) R_i u + \alpha \hat w_i\in\cK_i$ \\-
    $\hat w = \sum_i \hat w_i\in\cK$ \\
    return $w=(1-\alpha) u + \alpha \hat w\in\cK$
\end{pseudo*}
\caption{Multiplicative constraint decomposition for VI problem \eqref{eq:vi}.}
\label{alg:basiccd-mult}
\end{pseudofloat}

Note the use of a damping parameter $0<\alpha\le 1$.  In Algorithm \ref{alg:basiccd-add} its role is obvious, as we keep part of the current iterate if $\alpha<1$, but in Algorithm \ref{alg:basiccd-mult} we only update the iterate ``so far'', namely $w_i$, using the same damping that will be applied in generating the final output $w$.

In each boxed VI above the argument of $f$ is an element of $\cK$, as the reader should confirm.  Likewise, by \eqref{eq:constraintdecomp} and \eqref{eq:constraintrestrictionsum} one may write the test element $v_i - \hat w_i \in \cV_i$ as a difference of admissible vectors, from $\cK$, namely
\begin{align*}
[u - R_i u + v_i] - [u - R_i u + \hat w_i] &= v_i - \hat w_i, \label{eq:admissibledifference} \\
\left[\sum_{j<i} w_j + v_i + \sum_{j>i} R_j u\right] - \left[\sum_{j<i} w_j + \hat w_i + \sum_{j>i} R_j u\right] &= v_i - \hat w_i,  \notag
\end{align*}
for the two Algorithms respectively.

Define
\begin{equation}
e_i = \hat w_i - R_i u \in \cV_i \label{eq:ithupdate}
\end{equation}
as the $i$th subset update in either Algorithm.  Then, as the reader may check, $\hat w = u + \sum_{i} e_i$ and $w = u + \alpha \sum_i e_i$.  The easy identity $\hat w = u^* + \sum_i \hat w_i - R_i u^*$ also holds.  We will use these identities in Section \ref{sec:convergence}.

% FOLLOWING CONTAINS ASPIRATIONS
The convergence results in Section \ref{sec:convergence} require substantial damping when applying additive Algorithm \ref{alg:basiccd-add}, namely we require $\alpha \le 1/m$ as does \cite{Tai2003}.  Algorithm \ref{alg:basiccd-mult} can be shown to converge without damping ($\alpha=1$).  In Section \ref{sec:results} we demonstrate practical convergence for larger ranges of $\alpha$ than suggested by the theory.

We refer to Algorithms \ref{alg:basiccd-add} and \ref{alg:basiccd-mult} as the \emph{basic CD method}.  To construct efficient FE implementations (Section \ref{sec:multilevel}) we will need additional notions.  While the above Algorithms are indeed meaningful even when $f$ is nonlinear, non-local, or defined only on $\cK$, practical implementation for such general problems seems not to have been addressed in the literature.  In particular, references \cite{GraeserKornhuber2009,Tai2003} only apply the CD method to the classical obstacle problem.  The following example explains the implementation-relevant simplifications available in that case.

\begin{example}  \label{ex:fnice} Suppose $f:\cV \to \cV'$ is linear and defined on all of $\cV$.  Furthermore suppose $f$ is local in the sense that a basis $\{\phi_i\}$ of $\cV_i$ exists, with $\supp \phi_i$ compactly-contained in $\Omega$, such that for any $z\in\mathcal{V}$ the value $\ip{f(\phi_i)}{z}$ can be computed by an integral over the support of $\phi_i$.  Considering only additive Algorithm \ref{alg:basiccd-add} for simplicity, by linearity the boxed VI over $\cK_i$ can be written as
\begin{equation}
\ip{f(e_i)}{v_i-\hat w_i} \ge \tilde g(v_i-\hat w_i) \label{eq:linearlocalvi}
\end{equation}
where $\tilde g(z) = g(z) - \ip{f(u)}{z}$.  Note $e_i = \hat w_i - R_i u \in \cV_i \notin \cK$, in general, but VIs \eqref{eq:linearlocalvi} make sense because $f$ is defined over $\cV$.  Each problem \eqref{eq:linearlocalvi} can be solved using a stored residual $\ip{f(u)}{\cdot}$, namely as included here into the source term.  The VI \eqref{eq:vi} is then approximately solved in any incremental and efficient manner, by computations over the basis supports.
\end{example}

The assumption that $\ip{f(\phi_i)}{z}$ can be evaluated over the support of $\phi_i$ is satisfied when $f$ is a linear differential operator, $\cV$ is an FE space, and $\phi_i$ are hat functions; see Section \ref{sec:multilevel}.  Constrast non-local integral operators where $\ip{f(\phi_i)}{z}$ requires an integral over $\Omega$ even if $\phi_i$ is is a hat function with small support.

When $f$ has all the nice properties listed in Example \ref{ex:fnice} then a solver can take advantage of implementation efficiencies which are unavailable in general.  The actual solution process for VI \eqref{eq:linearlocalvi} is not the concern here; we are observing that the data of problem \eqref{eq:linearlocalvi}, and the cost of residual evaluation, are smaller than in the general case.  For more difficult problems additional ideas, illustrated in Section \ref{sec:multilevel}, are needed for practical implementation.  We will extend the basic method to nonlinear $f$ by applying the full approximation storage (FAS) idea of Brandt \cite{Brandt1977}, and the decomposed problems will only evaluate $f$ over (admissible) elements of $\mathcal{K}$.  However, extending the algorithm to non-local functionals $f$ is a topic for future research.


\section{Convergence of the basic algorithm} \label{sec:convergence}

We now seek a proof of the convergence of Algorithm \ref{alg:basiccd}.  This will be possible if we restrict to $2$-coercive operators and make certain assumptions as in \cite{Tai2003}.  First we define a new quantity related to VI problem \eqref{eq:vi}.

\begin{definition} Suppose $f:\cK \to \cV'$ and $g \in \cV'$.  For $u,v \in \cK$ let
\begin{equation}
  E(v,u) = \ip{f(v)}{v-u} - g(v-u).  \label{eq:normlikedefn}
\end{equation}
\end{definition}

If $u^*$ solves \eqref{eq:vi} then $E(v,u^*)$ is somewhat like an ``merit function,'' something used in the context of solving nonlinear equations \cite{NocedalWright2006} to replace an objective (scalar) functional.  (Note that $E(v,u^*)$ is \emph{not} a residual for \eqref{eq:vi}.)  The next lemma, which follows directly from $p$-coercivity, shows $E(v,u^*)$ is bounded below.

\begin{lemma} \label{lem:normlike}  Suppose $f:\mathcal{K} \to \mathcal{V}'$ is $p$-coercive and $u^* \in \mathcal{K}$ solves \eqref{eq:vi}.  For $v \in \mathcal{K}$,
\begin{equation}
  E(v,u^*) \ge \kappa \|v-u^*\|^p.  \label{eq:normlikebound}
\end{equation}
\end{lemma}

\begin{proof}
\begin{align*}
E(v,u^*) &= \ip{f(v)}{v-u^*} - \ip{f(u^*)}{v-u^*} + \ip{f(u^*)}{v-u^*} - g(v-u^*) \\
   &\ge \kappa \|v-u^*\|^p + 0.  \qedhere
\end{align*}
\end{proof}

Next we make two assumptions which are essentially the same as (7) and (8) in \cite{Tai2003}.

\begin{assumptions*}  There exists a constant $C_1>0$ so that
\begin{equation}
\left(\sum_{i=0}^{m-1} \|R_i u - R_i v\|^2\right)^{1/2} \le C_1 \|u-v\| \label{as:lipschitzrestrictions}
\end{equation}
for all $u,v\in\cK$.  Furthermore, there exists $C_2>0$ so that for all $v_i \in \cV_i$ and $y_j \in \cV_j$,
\begin{equation}
\sum_{i=0}^{m-1} \sum_{j=0}^{m-1} \left|\ip{f(w_{ij} + v_i) - f(w_{ij})}{y_j}\right| \le C_2 \left(\sum_{i=0}^{m-1} \|v_i\|^2\right)^{1/2} \left(\sum_{j=0}^{m-1} \|y_j\|^2\right)^{1/2} \label{as:lipschitzresidual}
\end{equation}
for all $w_{ij} \in \cK$ such that $w_{ij} + v_i \in \cK$ for all $i$ and $j$.
\end{assumptions*}

These inequalities might be described as ``totally Lipschitz'' requirements for the maps $R_i$ and $f$, respectively.  We make the following observations:
\begin{itemize}
\item Assumption \eqref{as:lipschitzrestrictions} addresses only the CD, and not the map $f$.
\item It is sometimes the case that \eqref{as:lipschitzresidual} addresses only a subspace decomposition property of $f$ (and not the CD itself).  For example, in the classical obstacle problem one may verify \eqref{as:lipschitzresidual} over all $w_{ij} \in \cV$.
\item The unconstrained versions of \eqref{as:lipschitzrestrictions} and \eqref{as:lipschitzresidual} are (13), (14) in \cite{TaiXu2002}, respectively.
\item If we assume $f$ itself is Lipschitz then the existence of $C_2$ is clear \cite{TaiXu2002}.  However, our bound for the convergence rate for Algorithm \ref{alg:basiccd} is improved when $C_1,C_2$ are made smaller.
\item Tai \cite{Tai2003} gives values of the constants $C_1$ in \eqref{as:lipschitzrestrictions} for obstacle problem constraint decompositions using $P_1$ FE spaces over shape-regular and quasi-uniform triangulations.  In particular, constants $C_1$ are known for overlapping domain decompositions (with or without an additional coarse mesh) and standard multilevel hierarchies.  Thus the cases we will demonstrate in Section \ref{sec:results} are already covered.
\item The constants $C_1,C_2$ generally depend on $m$.  However, regarding the convergence proof below, \cite{Tai2003} observes that if repeated application of Algorithm \ref{alg:basiccd} generates a bounded sequence of iterates in $\mathcal{K}$ then constants $C_1,C_2$ would be permitted to depend on $u,v,w_{ij},v_i,y_i$ as well.
\end{itemize}

From the above definitions and assumptions we have the following estimate for the parallel version of Algorithm \ref{alg:basiccd}.  Our proof follows \cite{Tai2003}.

\begin{lemma} \label{lem:core}  Suppose $f$ is $2$-coercive and $u^* \in \mathcal{K}$ solves \eqref{eq:vi}.  Recall $E(v,u^*)$ is defined in \eqref{eq:normlikedefn}.  Suppose $\mathcal{V}_i$, $\mathcal{K}_i$, $R_i$, and $f$ satisfy \eqref{as:lipschitzrestrictions} and \eqref{as:lipschitzresidual}.  Suppose $\hat w$ is computed from $u$ by the parallel version of Algorithm \ref{alg:basiccd}, and recall $e_i = \hat w_i - R_i u$.  Then
\begin{equation}
   E(\hat w,u^*) \le C_2 \sum_{i=0}^{m-1} \|e_i\|^2 + \kappa^{-1} C_1 C_2 \left(\sum_{i=0}^{m-1} \|e_i\|^2\right)^{1/2} E(u,u^*)^{1/2}. \label{eq:core}
\end{equation}
\end{lemma}

\begin{proof}  Recalling that $\hat w = u^* + \sum_i \hat w_i - R_i u^*$, by the (parallel version) boxed VI in Algorithm \ref{alg:basiccd} with $v_i = R_i u^*$ we get
\begin{align}
E(\hat w,u^*) &= \sum_{i=0}^{m-1} \ip{f(\hat w)}{\hat w_i - R_i u^*} - g(\hat w_i - R_i u^*) \label{eq:startcore} \\
    &\le \sum_{i=0}^{m-1} \ip{f(\hat w)}{\hat w_i - R_i u^*} + \ip{f(u + e_i)}{R_i u^* - \hat w_i} \notag \\
    &= \sum_{i=0}^{m-1} \ip{f(\hat w) - f(u + e_i)}{\hat w_i - R_i u^*}. \notag
\end{align}
For $i\in \{0,1,\dots,m-1\}$ use wrapped indices to define elements $\phi_{i,j}$:
\begin{align}
\phi_{i,0} &= u + e_i, \label{eq:gridcore} \\
\phi_{i,1} &= u + e_i + e_{i+1}, \notag \\
  &\vdots \notag \\
\phi_{i,m-1} &= u + e_i + e_{i+1} + \dots + e_{m-1} + e_0 + \dots + e_{i-1}, \notag
\end{align}
and $\phi_{i,j} \in \cK$ by \eqref{eq:constraintdecomp} and \eqref{eq:constraintrestrictionsum}.  Note that $\phi_{i,m-1} = u + \sum_i e_i = \hat w$.  From \eqref{eq:startcore}, apply a telescoping sum using the elements in \eqref{eq:gridcore}:
\begin{align}
E(\hat w,u^*) &\le \sum_{i=0}^{m-1} \sum_{j=1}^{m-1} \ip{f(\phi_{i,j}) - f(\phi_{i,j-1})}{\hat w_i - R_i u^*} \label{eq:nextcore} \\
  &\le \sum_{i=0}^{m-1} \sum_{j=0}^{m-1} \left|\ip{f(\phi_{i,j}) - f(\phi_{i,j-1})}{\hat w_i - R_i u^*}\right|. \notag
\end{align}
By an easy renumbering of the indices $i,j$ we may apply assumption \eqref{as:lipschitzresidual}, then the triangle inequality, and then assumption \eqref{as:lipschitzrestrictions}; we denote $\left(\sum_{i=0}^{m-1} \|e_i\|^2\right)^{1/2}$ by $Z$:
\begin{align}
E(\hat w,u^*) &\le C_2 Z \left(\sum_{i=0}^{m-1} \|\hat w_i - R_i u^*\|^2\right)^{1/2} = C_2 Z \left(\sum_{i=0}^{m-1} \|\hat w_i - R_i u + R_i u - R_i u^*\|^2\right)^{1/2} \label{eq:nextnextcore} \\
  &\le C_2 Z \left(Z + \left(\sum_{i=0}^{m-1} \|R_i u - R_i u^*\|^2\right)^{1/2}\right) \le C_2 Z \left(Z + C_1 \|u-u^*\|\right). \notag
\end{align}
Finally apply \eqref{eq:normlikebound} to give \eqref{eq:core}.
\end{proof}

FIXME NOW THE ENTIRE BATTLE IS TO GET A CONVEXITY RESULT $E(w,u) \ge C \sum_{j=0}^{m-1} \|e_j\|^2$ for some reasonable $C$

\section{Finite elements and multilevel constraint decomposition} \label{sec:multilevel}

In practice we will solve VI \eqref{eq:vi} over a finite-dimensional space $\cV$ based on a choice of a mesh over $\Omega$ and a finite element (FE) space.

FIXME triangulation

\begin{example}  FIXME if $\cV_i=\Span\{\phi_i\}$ for $\phi_i\in\cV$ are 1d spaces and if $\cK = \{v \ge \psi\} \subset \cV$ (obstacle problem), and if $R_i : \cK \to \cK_i$ for obstacle problem; note $\cK_i \not\subset \cK$ when $\psi>0$
\end{example}

FIXME state essentially Algorithm 4.7 \cite{GraeserKornhuber2009} but with $\text{V}(\nu_1,\nu_2)$ cycles which works for linear; observe that up-smoothing is more efficient; state the FAS version which has $O(m)$ residual evaluation complexity on each level


\section{Results for local variational inequalities} \label{sec:results}

FIXME


% A BRIDGE TOO FAR:  \section{Results for a nonlocal variational inequality} \label{sec:resultsnonlocal}


\bibliography{mcd2}
\bibliographystyle{siam}

\end{document}

