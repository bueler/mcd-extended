\documentclass[letterpaper,final,12pt,reqno]{amsart}

\usepackage[total={6.3in,9.2in},top=1.1in,left=1.1in]{geometry}

\usepackage{times,bm,bbm,empheq,fancyvrb,graphicx,amsthm,amssymb}
\usepackage[dvipsnames]{xcolor}
\usepackage{longtable}
\usepackage{booktabs}

\usepackage{tabto}
\TabPositions{1.5cm}

\usepackage{tikz}
\usetikzlibrary{decorations.pathreplacing}

\usepackage[kw]{pseudo}
\pseudoset{%
left-margin=15mm,%
topsep=5mm,%
label=\footnotesize\arabic*,%
idfont=\texttt,%
ctfont=\textsl,%
ct-left=\qquad\qquad(,%
ct-right=),%
}

\usepackage{float}

% hyperref should be the last package we load
\usepackage[pdftex,
colorlinks=true,
plainpages=false, % only if colorlinks=true
linkcolor=blue,   % ...
citecolor=Red,    % ...
urlcolor=black    % ...
]{hyperref}

\renewcommand{\baselinestretch}{1.05}

\allowdisplaybreaks[1]  % allow display breaks in align environments, if they avoid major underfulls

\newtheoremstyle{cstyle}% name
  {5pt}% space above
  {5pt}% space below
  {\itshape}% body font
  {}% indent amount
  {\itshape}% theorem head font
  {.}% punctuation after theorem head
  {.5em}% space after theorem head
  {\thmname{#1}\thmnumber{ #2}\thmnote{ (#3)}}% theorem head spec
\theoremstyle{cstyle}

\newtheorem{theorem}{Theorem}
\newtheorem{lemma}[theorem]{Lemma}
\newtheorem{assumptions}[theorem]{Assumptions}

\newtheoremstyle{cstyle*}% name
  {5pt}% space above
  {5pt}% space below
  {\itshape}% body font
  {}% indent amount
  {\itshape}% theorem head font
  {.}% punctuation after theorem head
  {.5em}% space after theorem head
  {\thmname{#1}}% theorem head spec
\theoremstyle{cstyle*}
\newtheorem{assumptions*}{Assumptions}

\newtheoremstyle{dstyle}% name
  {5pt}% space above
  {5pt}% space below
  {}%{\itshape}% body font
  {}% indent amount
  {\itshape}% theorem head font
  {.}% punctuation after theorem head
  {.5em}% space after theorem head
  {\thmname{#1}\thmnumber{ #2}\thmnote{ (#3)}}% theorem head spec
\theoremstyle{dstyle}

\newtheorem{definition}[theorem]{Definition}
\newtheorem{example}[theorem]{Example}

% numbering
\numberwithin{equation}{section}
\numberwithin{figure}{section}
\numberwithin{table}{section}
\numberwithin{theorem}{section}

\newcommand{\eps}{\epsilon}
\newcommand{\RR}{\mathbb{R}}

\newcommand{\grad}{\nabla}
\newcommand{\Div}{\nabla\cdot}
\newcommand{\trace}{\operatorname{tr}}

\newcommand{\hbn}{\hat{\mathbf{n}}}

\newcommand{\bb}{\mathbf{b}}
\newcommand{\be}{\mathbf{e}}
\newcommand{\bbf}{\mathbf{f}}
\newcommand{\bg}{\mathbf{g}}
\newcommand{\bn}{\mathbf{n}}
\newcommand{\br}{\mathbf{r}}
\newcommand{\bu}{\mathbf{u}}
\newcommand{\bv}{\mathbf{v}}
\newcommand{\bw}{\mathbf{w}}
\newcommand{\bx}{\mathbf{x}}
\newcommand{\by}{\mathbf{y}}
\newcommand{\bz}{\mathbf{z}}

\newcommand{\bF}{\mathbf{F}}
\newcommand{\bV}{\mathbf{V}}
\newcommand{\bX}{\mathbf{X}}

\newcommand{\bxi}{\bm{\xi}}
\newcommand{\bzero}{\bm{0}}

\newcommand{\cK}{\mathcal{K}}
\newcommand{\cV}{\mathcal{V}}

\newcommand{\rhoi}{\rho_{\text{i}}}

\newcommand{\ip}[2]{\left<#1,#2\right>}

\newcommand{\maxR}{R^{\bm{\oplus}}}
\newcommand{\minR}{R^{\bm{\ominus}}}
\newcommand{\iR}{R^{\bullet}}

\newcommand{\nn}{{\text{n}}}
\newcommand{\pp}{{\text{p}}}
\newcommand{\qq}{{\text{q}}}
\newcommand{\rr}{{\text{r}}}

\newcommand{\supp}{\operatorname{supp}}
\newcommand{\Span}{\operatorname{span}}


\begin{document}
\title[Multilevel constraint decomposition methods]{Multilevel constraint decomposition methods \\ for nonlinear variational inequalities}

\author{Ed Bueler}

\date{\today}

\begin{abstract} FIXME
\end{abstract}

\maketitle

%\tableofcontents

\thispagestyle{empty}
%\bigskip

\newfloat{pseudofloat}{t}{xyz}[section]
\floatname{pseudofloat}{Algorithm}


\section{Introduction} \label{sec:intro}

We extend the constraint decomposition (CD) method of X.-C.~Tai \cite{Tai2003} to nonlinear variational inequality (VI) problems to which it has not been applied.  The convergence of this method, for several types of decompositions, was proven by Tai for coercive VI problems which arise from minimization of a convex functional over a convex set.  In its multilevel form the method has been shown to have optimal complexity for elliptic, linear obstacle problems \cite[Subsection 5.4]{Tai2003}; see also \cite[Theorem 4.6 and Algorithm 4.7]{GraeserKornhuber2009}.

The multilevel CD algorithms in \cite{GraeserKornhuber2009} and \cite{Tai2003} are extended in three practical ways:
\renewcommand{\labelenumi}{\emph{(\roman{enumi})}}
\begin{enumerate}
\item We make finite element (FE) implementations of multilevel CD algorithm more effective for nonlinear problems by applying a full approximation storage (FAS; see \cite{Brandt1977,Bruneetal2015}) approach.  Results from the resulting \emph{nonlinear multilevel constraint decomposition} (NMCD) algorithm, which avoids global (Newton) linearization (compare \cite{GraeserKornhuber2009}), are shown in Section \ref{sec:results} for several nonlinear problems.
\item Our implementation permits ``box'' constraints, namely upper and lower obstacles.
\item In Sections \ref{sec:multilevel} and \ref{sec:vcycle} we explain why, given the manner in which constraint sets are decomposed in the multilevel context \cite{GraeserKornhuber2009}, multilevel ``up-smoothing'' is more efficient than ``down-smoothing''.  This observation seems to be new; compare the comments on V(1,0) and V(1,1) cycles in \cite{GraeserKornhuber2009,Tai2003}.  A strong preference for up-smoothing is special to multilevel CD methods and does not arise in unconstrained problems.
\end{enumerate}

The iterates from a CD method are always admissible, and thus the operator can be defined only for admissible states.  Admissible-iterate methods should permit direct solutions of certain VI problems, such as fluid-layer dynamics problems \cite{Bueler2021conservation,JouvetBueler2012}, for which non-admissible methods, including semi-smooth methods \cite{BensonMunson2006}, require unnatural modifications of the operator formula.  Our NMCD scheme is also designed for these problems, some of which require the solution of an auxiliary PDE on a domain determined inside the VI residual evaluation, as discussed in \cite{Bueler2021conservation}.

One of our examples demonstrates the application of the NMCD method to a problem of porous-media type, to which the extended convergence theory, \emph{(i)} above, does not apply because the nonlinear operator is not known to be coercive.  By ``freezing' the solution-dependent coefficient, the operator is approximated by a coercive operator for the duration of the V-cycle.  The resulting scheme is highly-effective for doubly-nonlinear diffusion operators, another type of problem to which the extended theory does not apply (Section \ref{sec:results}).

For the classical obstacle problem with a Laplacian operator, certain multilevel techniques are known to improve performance relative to the multilevel CD method \cite{GraeserKornhuber2009}, such as the truncated monotone multigrid method \cite{Kornhuber1994}.  These improved methods either track the active set in the discretization or modify the nodal basis functions, and in this sense they are discrete algorithms, while by contrast the CD method applies essentially at the level of the continuum problem; see Section \ref{sec:cd}.  Acceleration of our NMCD method via active-set and/or basis-level manipulations represents a potential extension of the method here, and is a topic for future research.


\section{Coercive variational inequalities} \label{sec:vi}

Suppose $\cV$ is a real and reflexive Banach space with norm $\|\cdot\|$, and denote its topological dual space by $\cV'$.  Denote the dual pairing of $\phi \in \cV'$ and $v\in\cV$ by $\ip{\phi}{v} = \phi(v)$, and note that $\|\phi\|_{\cV'} = \sup_{\|v\|=1} |\ip{\phi}{v}|$ defines a (Banach space) norm on $\cV'$.

Let $\cK \subset \cV$ be a nonempty closed and convex subset, the \emph{constraint set}; elements of $\cK$ are said to be \emph{admissible}.  For a continuous, but generally nonlinear, operator $f:\cK \to \cV'$ and a linear \emph{source functional} $\ell\in \cV'$ we consider the following \emph{variational inequality} (VI) for the (exact) solution $u^*\in \cK$, if it exists:
\begin{equation}
\ip{f(u^*)}{v-u^*} \ge \ip{\ell}{v-u^*} \qquad \text{for all } v\in \cK. \label{eq:vi}
\end{equation}
Because $f$ is (generally) nonlinear, the source term $\ell$ is not strictly needed, and indeed by redefining $f$ we may take $\ell=0$, but its presence clarifies the algorithm in Section \ref{sec:vcycle}.

VI \eqref{eq:vi} generalizes the nonlinear system of equations $f(u^*)=\ell$ to problems where $u^*$ is also required to be in a constraint set $\cK$.  Informally, if we conceptualize the dual pairing as an inner product then \eqref{eq:vi} says that the angle between $f(u^*)-\ell$ and any arbitrary vector $v-u$ pointing from $u$ into $\cK$ is at most $90^\circ$.  That is, while $f(u^*)-\ell$ may not be zero, it points directly into $\cK$.  However, if $u^*$ is in the interior of the constraint set, $u^*\in\cK^\circ$, then \eqref{eq:vi} implies $f(u^*)=\ell$.

\begin{definition} The following definitions are standard \cite{KinderlehrerStampacchia1980}.  A map $f:\cK \to \cV'$ is \emph{monotone} if
\begin{equation}
\ip{f(u)-f(v)}{u-v} \ge 0 \qquad \text{for all } u,v \in \cK, \label{eq:monotone}
\end{equation}
\emph{strictly monotone} if equality in \eqref{eq:monotone} implies $u=v$, and \emph{coercive} if there exists $w \in \cK$ so that
\begin{equation}
\frac{\ip{f(u)-f(w)}{u-w}}{\|u-w\|} \to +\infty \qquad \text{as } \|u\|\to +\infty. \label{eq:coercive}
\end{equation}
\end{definition}

It is well-known that if $f:\cK \to \cV'$ is continuous, monotone, and coercive then VI \eqref{eq:vi} has a solution \cite[Corollary III.1.8]{KinderlehrerStampacchia1980}, uniquely so when $f$ is strictly monotone.  As in the calculus of variations \cite{Evans2010}, coercivity permits a compactness argument for unbounded sets $\cK$.  (Recall that the bounded, closed subsets of a reflexive Banach space are weakly compact.)  The condition of continuity can be weakened to only apply on finite-dimensional subspaces \cite{KinderlehrerStampacchia1980}, but the stronger condition will apply in our examples.

The coercive VIs solved in this paper satisfy a stronger inequality than \eqref{eq:monotone} and \eqref{eq:coercive}.

\begin{definition}  Let $p>1$.  The map $f:\cK \to \cV'$ is \emph{$p$-coercive} if there exists $\kappa>0$ such that
\begin{equation}
\ip{f(u)-f(v)}{u-v} \ge \kappa \|u-v\|^p \qquad \text{for all } u,v \in \cK. \label{eq:pcoercive}
\end{equation}
\end{definition}

It is easy to see that if $f$ is $p$-coercive then it is monotone, strictly monotone, and coercive.

\begin{theorem}  \label{thm:viwellposed}  If $f:\cK \to \cV'$ is continuous and $p$-coercive then there exists a unique $u^*\in \cK$ solving VI \eqref{eq:vi}.
\end{theorem}

When $f$ is monotone, \eqref{eq:vi} generalizes the problem of minimizing a convex function over $\cK$.  In fact, suppose $F:\cK \to \RR$ is lower semi-continuous and (G\^ateau) differentiable with continuous derivative $F':\cK \to \cV'$.  Then $F$ is convex if and only if $F'$ is monotone \cite[Proposition I.5.5]{EkelandTemam1976}.  Furthermore, Proposition II.2.1 in \cite{EkelandTemam1976} shows that if $F$ is convex then \eqref{eq:vi} holds for $f=F'$ and $\ell=0$ if and only if
\begin{equation}
u^* = \operatorname{arg-min}_{v\in\cK} F(v). \label{eq:minimization}
\end{equation}
The CD methods of Tai \cite{Tai2003} address problem \eqref{eq:minimization} under the hypothesis that $F'$ is 2-coercive.  (Note that Tai \cite{Tai2003} uses ``coercive'' for what we call $2$-coercive.)

From now on $\Omega \subset \RR^d$ denotes a bounded, open set with smooth or piecewise-smooth (e.g.~polygonal) boundary.  Sobolev spaces \cite{Evans2010} are denoted by $W^{k,p}(\Omega)$, for integer $k$ and $1\le p \le \infty$.  The following example includes the classical obstacle problem for the linear Laplacian \cite{GraeserKornhuber2009} and the $p$-Laplacian for $p\ge 2$ \cite{ChoeLewis1991}.

\begin{example}  \label{ex:plaplacian}  Suppose $a\in L^\infty(\Omega)$ such that $a(x)\ge a_0$ a.e.~for some constant $a_0>0$, and $p\ge 2$.  For $u,v \in \cV = W^{1,p}_0(\Omega)$ define $f:\cV \to \cV'$ by
\begin{equation}
\ip{f(u)}{v} = \int_\Omega a(x) |\grad u|^{p-2} \grad u \cdot \grad v\,dx, \label{eq:plaplacian}
\end{equation}
a continuous map \cite[Theorem A.0.6]{Peral1997}.  Now, if $x,y\in\RR^d$ then $(|x|^{p-2} x - |y|^{p-2} y)\cdot (x-y) \ge 2^{2-p} |x-y|^p$ \cite[see Appendix A and references therein]{Bueler2021conservation}.  Thus it follows from the Poincar\'e inequality that
    $$\ip{f(u) - f(v)}{u-v} \ge 2^{2-p} a_0 \|\grad u - \grad v\|_p^p \ge 2^{2-p} a_0 C \|u-v\|^p$$
for some $C>0$, and thus $f$ is $p$-coercive.  On the other hand, for $\ell\in\cV'$ define
    $$F(v) = \int_\Omega \frac{a(x)}{p} |\grad v|^p\,dx - \ip{\ell}{v}.$$
Then $F'(v) = f(v) - \ell$ and $F$ is a convex functional (since $f$ is coercive).  For any closed and convex $\cK\subset \cV$, VI problem \eqref{eq:vi} for is equivalent to optimization problem \eqref{eq:minimization}.\end{example}

The map in \eqref{eq:plaplacian} is also coercive if $1<p<2$, but the proof is somewhat different \cite[Theorem 4.4]{Bueler2021conservation}.  In Section \ref{sec:results} we consider only the $p\ge 2$ case.

However, not all VI problems arise from optimization.  We give two such examples next, first a coercive and linear advection-diffusion problem, and then a nonlinear porous-medium-type problem; each is important in applications.  The first is preceded by a lemma.

\begin{lemma}  \label{lem:advectionskew}  \cite{Elmanetal2014}\,  Suppose $\bX :\Omega \to \RR^d$ is a bounded and boundedly-differentiable vector field on $\Omega$ with zero divergence ($\Div \bX=0$).  For $u,v \in W^{1,2}(\Omega)$ let $b(u,v) = \int_\Omega (\bX \cdot \grad u) v\,dx$.  Then $b(u,u) = \frac{1}{2} \int_{\partial \Omega} u^2 \bX\cdot \bn\,dx$ where $\bn$ is the outward normal on $\partial \Omega$.
\end{lemma}

\begin{proof}
Integration by parts gives $b(u,v) = - b(v,u) + \int_{\partial \Omega} uv \bX\cdot \bn\,dx$, so the result follows.
\end{proof}

\begin{example}  \label{ex:advectiondiffusion}  Suppose $\partial\Omega$ is partitioned into Dirichlet and Neumann portions, i.e.~$\partial\Omega = \partial_D\Omega \cup \partial_N\Omega$, with $\partial_D\Omega$ of positive measure.  Let $\cV = W_0^{1,2}(\Omega)$ be the space of functions which are zero along $\partial_D\Omega$.  Consider a divergence-free velocity field $\bX$ on $\Omega$ satisfying the conditions of Lemma \ref{lem:advectionskew}, but additionally assume that the flow is outward on the Neumann boundary, $\bX \cdot \bn \ge 0$ on $\partial_N\Omega$.  For $u,v \in \cV = W_0^{1,2}(\Omega)$ and $\eps>0$ define
\begin{equation}
\ip{f(u)}{v} = \eps \left(\grad u, \grad v\right)_{L^2(\Omega)} - b(u,v). \label{eq:advectiondiffusion}
\end{equation}
Consider VI \eqref{eq:vi} for any closed and convex $\cK \subset \cV$ and $g\in\cV'$.  It is easy to see that $|\ip{f(u)}{v}| \le (\eps + \|\bX\|_\infty) \|u\| \|v\|$, thus that $f:\cK \to \cV'$ is continuous.  Lemma \ref{lem:advectionskew} says that the bilinear form $s(u,v)$ is skew-symmetric up to a nonnegative term.  By the outward flow assumption and the Poincar\'e inequality,
\begin{align*}
\ip{f(u)-f(v)}{u-v} &= \eps \int_\Omega |\grad u - \grad v|^2\,dx + b(u-v,u-v) \\
                    &= \eps \int_\Omega |\grad u - \grad v|^2\,dx + \frac{1}{2} \int_{\partial_N\Omega} (u-v)^2 \bX\cdot\bn \ge \eps C \|u-v\|^2.
\end{align*}
Thus $f$ is 2-coercive, and so VI problem \eqref{eq:vi} is well-posed.
\end{example}

References \cite{Bueler2021conservation,ChangNakshatrala2017} consider advection-diffusion VI problems like Example \ref{ex:advectiondiffusion}, specifically over the set $\cK = \{v\ge 0\}$.  If $\bX \ne 0$ then VI \eqref{eq:vi} for $f$ in \eqref{eq:advectiondiffusion} does not correspond to a minimization problem.  Indeed, $\ip{f(u)}{v}$ is not symmetric in that case,\footnote{Assume $\bX \ne 0$ is continuous for simplicity.  For $u,v$ which are zero on $\partial \Omega$, note $\ip{f(u)}{v} - \ip{f(v)}{u} = -2 b(u,v)$.  By constructing $u,v$ locally near some point where $\bX$ is nonzero. one may show $b(u,v)\ne 0$.} so $f$ cannot be the gradient of a scalar objective.

\begin{example}  \label{ex:porous}  Suppose $\phi:[0,\infty) \to [\phi_0,\infty)$ is continuous for some constant $\phi_0>0$.  Then the following operator, of porous medium type, is not known to be monotone (or coercive) when $\phi$ is not constant:
\begin{equation}
\ip{f(u)}{v} = \int_\Omega \phi(u) \grad u \cdot \grad v\,dx. \label{eq:porous}
\end{equation}
As with Example \ref{ex:advectiondiffusion}, this form does not have the symmetry necessary to be the gradient of a scalar objective.
\end{example}

Numerical solver performance for Examples \ref{ex:plaplacian}, \ref{ex:advectiondiffusion}, and \ref{ex:porous} will be considered in Section \ref{sec:results}.  Further examples of nonlinear VI problems appear in ice sheet models \cite{Calvoetal2002,JouvetBueler2012} and other geophysical fluids.  In the above Examples $f(u)$ may be defined over all of $\mathcal{V}$, but in some geophysical fluid problems the value of $f(u)$ is only defined for admissible $u\in \mathcal{K}$ because $u$ defines the fluid layer thickness \cite{Bueler2021conservation}.


\section{Constraint decomposition (CD)} \label{sec:cd}

Suppose there are $m<\infty$ closed subspaces $\cV_i \subset \cV$ so that the sum
\begin{equation}
\cV = \sum_{i=0}^{m-1} \cV_i \label{eq:subspacedecomp}
\end{equation}
holds in the sense that if $w \in \cV$ then there exist $w_i \in \cV_i$ so that $w = \sum_i w_i$.  Equation \eqref{eq:subspacedecomp} is called a \emph{subspace decomposition} \cite{Xu1992}.

Suppose further that $\cK_i \subset \cV_i$ are nonempty, closed, and convex subsets such that
\begin{equation}
\cK = \sum_{i=0}^{m-1} \cK_i. \label{eq:constraintdecomp}
\end{equation}
The sum in \eqref{eq:constraintdecomp} is required to hold in two senses \cite{TaiTseng2002}: \emph{(i)}~if $w \in \cK$ then there exist $w_i \in \cK_i$ so that $w = \sum_i w_i$, and \emph{(ii)}~if $z_i \in \cK_i$ for each $i$ then $\sum_i z_i \in \cK$.  Note that sense \emph{(ii)} is automatic for \eqref{eq:subspacedecomp}, because the $\cV_i$ are subspaces, and that neither decomposition \eqref{eq:subspacedecomp} or \eqref{eq:constraintdecomp} is required to be unique.  Also, $\cK_i \not\subset \cK$ in many applications; see the cartoon in Figure \ref{fig:cartoon}.

\begin{figure}[ht]
\includegraphics[width=0.6\textwidth]{genfigs/cartoon.pdf}
\caption{Suppose $\mathcal{V}$ is the space of real functions on a two-point set $\Omega=\{x_1,x_2\}$.  A constraint decomposition (CD) for a unilateral obstacle problem with $\mathcal{K}=\{v\ge \psi\}$ might look like this.}
\label{fig:cartoon}
\end{figure}

Finally, for each $\cK_i$ we assume there are bounded, generally-nonlinear \emph{decomposition operators}\footnote{Denoted ``$R_i$'' in \cite{Tai2003}.  Here we avoid a notational conflict with multilevel transfer operators (Section \ref{sec:multilevel}).} $\Pi_i : \cK \to \cK_i$ such that if $v \in \cK$ then
\begin{equation}
v = \sum_{i=0}^{m-1} \Pi_i v;  \label{eq:constraintrestrictionsum}
\end{equation}
see Figure \ref{fig:cartoon}.  Clearly \eqref{eq:constraintrestrictionsum} implies sense \emph{(i)} for \eqref{eq:constraintdecomp}. A \emph{constraint decomposition} (CD) of $\cK$ is a choice of $\cV_i,\cK_i,\Pi_i$ satisfying \eqref{eq:subspacedecomp}--\eqref{eq:constraintrestrictionsum} \cite{Tai2003}.

In Sections \ref{sec:multilevel} and \ref{sec:vcycle} we will introduce discretizations and a practical algorithms for finite-dimensional VI problems, but observe here that this CD concept applies at the level of the continuum problem.  The following two obstacle problem examples illustrate this.  The first is an overlapping domain decomposition.

\begin{example}  \label{ex:domaindecomposition}  For a bounded domain $\Omega \subset \RR^d$, let $\cV = W_0^{k,p}(\Omega)$ for $k\ge 0$ and $p\ge 1$.  Suppose an obstacle $\psi \in W^{k,p}(\Omega)$ satisfies $\psi|_{\partial \Omega} \le 0$, and let $\cK = \{v \ge \psi\} \subset \cV$.  Suppose further that $\{\phi_i\}_{i=0}^{m-1}$ is a smooth partition of unity on $\Omega$, satisfying $0 \le \phi_i\le 1$ and $\sum_i \phi_i = 1$, and let $\Omega_i$ be the open support of $\phi_i$.  Let $\cV_i = \{w \in \cV:w|_{\Omega \setminus \Omega_i} =0 \}$, $\cK_i = \{v \in \cV_i: v \ge \phi_i \psi\}$, and $\Pi_i(v) = \phi_i v$.  Then \eqref{eq:subspacedecomp}, \eqref{eq:constraintdecomp}, and \eqref{eq:constraintrestrictionsum} all hold.
\end{example}

Our second example is a disjoint frequency decomposition.  A multilevel CD, e.g.~the one proposed in Section \ref{sec:multilevel}, approximates such a frequency decomposition.

\begin{example}  \label{ex:frequencydecomposition}  For simplicity suppose $\Omega = (0,a)^d \subset \RR^d$ is a cube, and let $\cV = W_{\text{per}}^{p,k}(\Omega)$, for $p\ge 2$ and $k\ge 0$, be the periodic functions.  Suppose $\psi \in W_{\text{per}}^{p,k}(\Omega)$ and let $\cK = \{v \ge \psi\} \subset \cV$.  Without using any detailed notation for Fourier representation, but noting that the frequencies are discrete because $\Omega$ is compact, suppose $\{\cV_i\}$ are $m<\infty$ subspaces of $\cV$ defined by an (nonoverlapping) partition by frequency, thus satisfying \eqref{eq:subspacedecomp} as an orthogonal decomposition.  Suppose $P_i:\cV \to \cV_i$ are the corresponding orthogonal projections, satisfying $I = \sum_i P_i$.  Let $\cK_i = \{v \ge P_i \psi\} \subset \cV_i$ and $\Pi_i = P_i$.  Then \eqref{eq:constraintdecomp} and \eqref{eq:constraintrestrictionsum} also hold.
\end{example}

Again note that $\cK_i \not\subset \cK$ in these cases.  Specifically in Example \ref{ex:domaindecomposition}, if $\psi$ is positive over portions of $\Omega$ where the decomposition into overlapping subdomains $\Omega_i$ is nontrivial, then $\cK_i \not\subset \cK$, and similarly for Example \ref{ex:frequencydecomposition}.  The important inclusion is $\cK_i \subset \cV_i$.


\section{Basic CD iterations} \label{sec:cditers}

Algorithms \ref{alg:basiccd-add} and \ref{alg:basiccd-mult} below record basic CD iterative methods as functions which improve a current iterate $u \in \cK$ by solving ``smaller'' VI problems over each subset $\cK_i$.   Their outputs $w\in\cK$ should be closer to the solution $u^* \in \cK$ of \eqref{eq:vi}.

\begin{pseudofloat}[H]
\begin{pseudo*}
\pr{cd-add}(u)\text{:} \\+
    for all $i \in \{0,\dots,m-1\}$: \\+
        \rm{find} $\hat w_i\in \cK_i$ \rm{so that for all} $v_i\in \cK_i$, \\+
            $\boxed{\ip{f(u - \Pi_i u + \hat w_i)}{v_i-\hat w_i} \ge \ip{\ell}{v_i-\hat w_i}}$ \\--
    $\hat w = \sum_i \hat w_i\in\cK$ \\
    return $w=(1-\alpha) u + \alpha \hat w$
\end{pseudo*}
\caption{One additive CD iteration for VI problem \eqref{eq:vi}.}
\label{alg:basiccd-add}
\end{pseudofloat}

In Algorithm \ref{alg:basiccd-add}, the additive or parallel CD method, the \textbf{for all} can be computed in any order.  Note that $u-\Pi_iu+\hat w_i$ removes the part of $u$ which lies in $\mathcal{K}_i$ before adding-back the improved subset solution $\hat w_i$, also from $\mathcal{K}_i$.  By contrast, in the following multiplicative (successive) Algorithm \ref{alg:basiccd-mult} each subset solution updates the global iterate before proceeding to the next subset.  The sets $\mathcal{K}_i$ are addressed in a fixed order, and the sum $\sum_{j>i} \Pi_j u$ should be read as ``keep those parts of the input $u$ which we have not yet improved.''

\begin{pseudofloat}[H]
\begin{pseudo*}
\pr{cd-mult}(u)\text{:} \\+
    for $i = 0,\dots,m-1$: \\+
        \rm{find} $\hat w_i\in \cK_i$ \rm{so that for all} $v_i\in \cK_i$, \\+
            $\displaystyle \boxed{\ip{f\Big(\sum_{j<i} w_j + \hat w_i + \sum_{j>i} \Pi_j u\Big)}{v_i-\hat w_i} \ge \ip{\ell}{v_i-\hat w_i}}$ \\-
            $w_i = (1-\alpha) \Pi_i u + \alpha \hat w_i\in\cK_i$ \\-
    $\hat w = \sum_i \hat w_i\in\cK$ \\
    return $w=(1-\alpha) u + \alpha \hat w$
\end{pseudo*}
\caption{One multiplicative CD iteration for VI problem \eqref{eq:vi}.}
\label{alg:basiccd-mult}
\end{pseudofloat}

While Algorithm \ref{alg:basiccd-add} generalizes a Jacobi iteration, Algorithm \ref{alg:basiccd-mult} generalizes Gauss-Seidel.  Note the use of a damping parameter $0<\alpha\le 1$ in both Algorithms.  In Algorithm \ref{alg:basiccd-mult} we must update the in-progress iterate $w_i$ using the same damping that is applied to the final output $w$.

In each boxed VI the argument of $f$ is an element of $\cK$, as the reader should confirm.  On the other hand, by \eqref{eq:constraintdecomp} and \eqref{eq:constraintrestrictionsum} one may also write the test vector $v_i - \hat w_i \in \cV_i$ as a difference of admissible vectors from $\cK$.  For the two Algorithms respectively:
\begin{align*}
[u - \Pi_i u + v_i] - [u - \Pi_i u + \hat w_i] &= v_i - \hat w_i, \label{eq:admissibledifference} \\
\left[\sum_{j<i} w_j + v_i + \sum_{j>i} \Pi_j u\right] - \left[\sum_{j<i} w_j + \hat w_i + \sum_{j>i} \Pi_j u\right] &= v_i - \hat w_i.  \notag
\end{align*}

For either Algorithm we define
\begin{equation}
e_i = \hat w_i - \Pi_i u \in \cV_i, \label{eq:ithupdate}
\end{equation}
the $i$th \emph{subset update}.  As the reader may check, $\hat w = u + \sum_{i} e_i$ and $w = u + \alpha \sum_i e_i$.  Note $\hat w = u^* + \sum_i \hat w_i - \Pi_i u^*$ also holds.

We refer to Algorithms \ref{alg:basiccd-add} and \ref{alg:basiccd-mult} as the \emph{basic CD iterations}.  These basic iterations are meaningful even when $f$ is nonlinear, non-local, or defined only on $\cK$.  However, practical and efficient FE implementation for such general problems seems not to have been addressed in the literature, and will need additional notions.  In particular, references \cite{GraeserKornhuber2009,Tai2003} only apply the CD method to the classical obstacle problem for which $f$ is linear, local, and defined for all inputs.  The following example exposes the implementation simplifications available in that case.

\begin{example}  \label{ex:fnice} Suppose $f:\cV \to \cV'$ is linear and defined on all of $\cV$.  Furthermore suppose $f$ is a differential operator, thus local in the sense that for any $z\in\mathcal{V}$ the value $\ip{f(\phi)}{z}$ can be computed by an integral only over the support of $\phi \in \mathcal{V}$.  Considering additive Algorithm \ref{alg:basiccd-add} for concreteness, by linearity the boxed VI can be written as
\begin{equation}
\ip{f(e_i)}{v_i-\hat w_i} \ge \ip{\tilde\ell}{v_i-\hat w_i}, \label{eq:linearlocalvi}
\end{equation}
for all $v_i \in \mathcal{K_i}$, where $\tilde\ell = \ell - f(u)$.  Note $e_i = \hat w_i - \Pi_i u \in \cV_i$ is not generally admissible ($e_i \notin \cK$), but VIs \eqref{eq:linearlocalvi} make sense because $f$ is defined over $\cV$.  Each problem \eqref{eq:linearlocalvi} can be solved using a stored residual $f(u)$, here included into a source term.  If a hat function basis exists for $\cV_i$, namely of small-support functions, then by linearity of $f$ a solver for VI \eqref{eq:linearlocalvi} can be implemented using only integrals over the hat function supports.
\end{example}

When $f$ has all the nice properties listed in Example \ref{ex:fnice} then a solver can take advantage of implementation efficiencies which are unavailable in general.  The FE approximation of VI \eqref{eq:vi}---see Sections \ref{sec:multilevel} and \ref{sec:vcycle}---can be solved in any incremental and efficient manner following the techniques in \cite{GraeserKornhuber2009,Tai2003}, namely by computations over hat function supports.  The actual solution process for VI \eqref{eq:linearlocalvi} is not really the concern here; we are observing that the storage needed for problem \eqref{eq:linearlocalvi}, and the cost of residual evaluation, are smaller than in the general case.  For more difficult problems certain additional ideas, illustrated in Section \ref{sec:vcycle}, are needed for practical implementation.  Specifically, we will extend a multilevel CD iteration to a method for nonlinear functions $f:\cK\to\cV'$, i.e.~defined only on the constraint set $\cK$, by applying the full approximation storage (FAS) idea of Brandt \cite{Brandt1977}.

One may also contrast Example \ref{ex:fnice} with the case of a non-local integral operator $f$ \cite{Bueler2021conservation}, where $\ip{f(\phi)}{z}$ requires an integral over $\Omega$ even if $\phi$ is a function with small support such as a hat function.  Extension of the multilevel CD method to such non-local functionals $f$ is a topic for future research.


\section{Multilevel finite element CDs} \label{sec:multilevel}

We will approximately solve VI \eqref{eq:vi} for certain nonlinear obstacle problems, over two-dimensional computational domains $\Omega \subset \RR^2$, and the Sobolev space $\mathcal{V}=W_0^{1,p}(\Omega)$, using piecewise-linear finite element (FE) spaces over triangular meshes.  The targeted mesh will be the finest level in a hierarchy of meshes constructed by refinement of a coarsest mesh.  In this Section we construct the corresponding multilevel CD.  Its constraint sets are designed to allow multigrid coarser-level corrections (next Section) in a manner which does not violate the inequality constraints, and which also does not introduce unnecessary high frequencies into the coarser-level problems.

The mesh hierarchy and FE spaces are constructed in a standard manner \cite{Elmanetal2014}.  Suppose $\Omega$ is polygonal and triangulated by $\mathcal{T}^0$, the \emph{coarsest mesh}, a finite set of non-overlapping triangles with union $\overline{\Omega}$ and characteristic size (e.g.~maximum edge length) $h_0>0$.  Starting with $j=1$, let $\mathcal{T}^j$ be the uniform refinement of $\mathcal{T}^{j-1}$ by edge bisection, thus each $\mathcal{T}^{j-1}$ triangle becomes four similar triangles in $\mathcal{T}^j$ with characteristic size $h_j = 2^{-j} h_0$.  Refinement proceeds to level $J\ge 0$, the \emph{depth} of the hierarchy, giving $\mathcal{T}^J$ as the \emph{finest mesh}.  Let $m_j$ be the number of interior nodes in $\mathcal{T}^j$, the \emph{degrees of freedom} at the $j$th level, and denote $h=h_J=2^{-J} h_0$ and $m=m_J$ for the finest level.  Now let $\mathcal{V}^j$ be the continuous, piecewise-linear FE space $P_1$ \cite{Elmanetal2014} over $\mathcal{T}^j$ with zero boundary values, and observe the following nesting:
\begin{equation}
\mathcal{V}^0 \subset \mathcal{V}^1 \subset \dots \subset \mathcal{V}^J \subset \mathcal{V}.  \label{eq:fe:nestedspaces}
\end{equation}

In the above subspace construction we have chosen triangles and the $P_1$ FE space.  Quadrilaterals and the $Q_1$ FE space will work just as well in what follows.  However, use of higher-order Lagrange elements $P_k$ and $Q_k$ would imply a ``variational crime'' \cite[Chapter 10]{BrennerScott2007} specific to inequality constraints in FE schemes.  Our FE construction of constraint sets requires a property we call \emph{nodal monotonicity}, namely that for all FE functions $w,z \in \mathcal{V}^j$,
\begin{equation}
\bw \ge \bz \quad \implies \quad w \ge z \label{eq:nodalmonotonicity}
\end{equation}
where $\bw,\bz$ denote the vectors of nodal values.  For $k\ge 2$ there are hat functions in $P_k$ and $Q_k$, i.e.~functions with nodal value one at a single node, and otherwise zero nodal values, which take on negative values \cite[Figure 1.7]{Elmanetal2014}, whereas this does not occur for $k\le 1$.  It is easy to show that \eqref{eq:nodalmonotonicity} is equivalent to the non-existence of such sometimes-negative hat functions.

Next we define ``box constraints'' \cite{BensonMunson2006}.  Let $\tilde{\mathcal{V}}^j$ denote the set of functions on the $\mathcal{T}^j$ nodes with values in the extended real line $\tilde{\RR} = [-\infty,+\infty]$.  At the finest-level, suppose that the \emph{lower} and \emph{upper obstacles} $\underline{\gamma}^J, \overline{\gamma}^J \in \tilde{\mathcal{V}}^J$ satisfy
\begin{equation}
\underline{\gamma}^J \le \overline{\gamma}^J, \qquad \underline{\gamma}^J < +\infty, \qquad -\infty < \overline{\gamma}^J, \qquad \text{and} \qquad \underline{\gamma}^J|_{\partial\Omega} \le 0 \le \overline{\gamma}^J|_{\partial\Omega}. \label{eq:fe:boxconstraintrequirements}
\end{equation}
Where $\underline{\gamma}^J=-\infty$ there will no lower constraint, and similarly where $\overline{\gamma}^J=+\infty$ no upper constraint.  Using nodal monotonicity, define
\begin{equation}
\mathcal{K}^J = \left\{v \in \mathcal{V}^J\,:\,\underline{\gamma}^J \le v \le \overline{\gamma}^J\right\}, \label{eq:fineconstraintset}
\end{equation}
the finest-level \emph{constraint set} of \emph{admissible} FE functions, a nonempty, closed, and convex subset of $\mathcal{V}^J$.  As $\mathcal{V}^J$ functions never take values $\pm\infty$, inequality in \eqref{eq:fineconstraintset} is strict where $\underline{\gamma}^J, \overline{\gamma}^J = \pm \infty$.

Let $f^J:\mathcal{K}^J \to (\mathcal{V}^J)'$ be a discretization of $f$ in \eqref{eq:vi}.  Noting that the functional $\ell$ in \eqref{eq:vi} is already defined and continuous over $\mathcal{V}^J$, we denote $\ell^J \in (\mathcal{V}^J)'$ for its FE representation.  We seek the finest-level FE solution $u^J \in \mathcal{K}^J$ of the finite-dimensional, nonlinear VI
\begin{equation}
\ip{f^J(u^J)}{v-u^J} \ge \ip{\ell^J}{v-u^J} \qquad \text{for all } v\in \cK^J. \label{eq:fe:vi}
\end{equation}
As for \eqref{eq:vi}, if $f^J$ is continuous and $p$-coercive on $\mathcal{K}^J$, in the sense that \eqref{eq:pcoercive} holds for all $u,v \in \mathcal{K}^J$, then \eqref{eq:fe:vi} is well-posed.

While \eqref{eq:fe:vi} approximates \eqref{eq:vi}, it may not technically be conforming for \eqref{eq:vi}.  That is, even if quadrature and other implementation details for $f^J$ are ignored, and despite the vector space inclusion $\mathcal{V}^J \subset \mathcal{V}$, we do not require $\mathcal{K}^J \subset \mathcal{K}$.  The continuum constraint set $\mathcal{K}$ remains abstract in applications where only the fine-level FE obstacles $\underline{\gamma}^J, \overline{\gamma}^J$ are defined.  In fact, even if smooth continuum obstacles exist, admissibility with respect to their FE interpolants might not imply continuum admissiblity.  These caveats specifically apply in geophysical situations where the obstacle is given by meshed elevation data, the continuum limit of which is undefined \cite{Bueler2016}.

The multilevel algorithm in the next section, a solver for VI problem \eqref{eq:fe:vi}, requires transfer operators between meshes.  We use five such operators, three of which are standard, linear maps used for multigrid methods \cite{Trottenbergetal2001}.  Table \ref{tab:transfers} will help the reader keep track.

\begin{table}
\begin{tabular}{llc}
\emph{name}  & \emph{action}  & \emph{linear?} \\ \hline
canonical prolongation        & $P:\mathcal{V}^{j-1}\to\mathcal{V}^j$ & \,\checkmark \\
canonical (dual) restriction  & $R:(\mathcal{V}^j)'\to(\mathcal{V}^{j-1})'$ & \,\checkmark \\
(nodal) injection             & $\iR:\mathcal{V}^j\to\mathcal{V}^{j-1}$ & \,\checkmark \\
maximum restriction           & $\maxR:\tilde{\mathcal{V}}^j\to\tilde{\mathcal{V}}^{j-1}$ & \\
minimum restriction           & $\minR:\tilde{\mathcal{V}}^j\to\tilde{\mathcal{V}}^{j-1}$ & 
\end{tabular}

\medskip
\caption{Transfer operators used in this paper.  The FE vector space $\mathcal{V}^j$ has zero boundary values while $\tilde{\mathcal{V}}^j$ has $\tilde \RR$ values, and arbitrary boundary values.}
\label{tab:transfers}
\end{table}

Canonical prolongation $P:\mathcal{V}^{j-1}\to\mathcal{V}^j$ acts as the identity on coarser-mesh functions in $\mathcal{V}^{j-1}$; it is the linear vector space inclusion $\mathcal{V}^{j-1} \hookrightarrow \mathcal{V}^j$.  Canonical (dual) restriction $R:(\mathcal{V}^j)'\to(\mathcal{V}^{j-1})'$ also acts as an identity in the sense that if $\sigma \in (\mathcal{V}^j)'$ then $(R\sigma)[z] = \sigma[z]$ for all $z \in \mathcal{V}^{j-1} \subset \mathcal{V}^j$.  While these canonical transfers do not lose information, they typically involve a change of FE representation.  For example, the practical formula for canonical restriction, derived by writing coarser-mesh hat functions in terms of finer-mesh ones, is a ``full-weighting'' average \cite{Trottenbergetal2001}, and its transpose is canonical prolongation.

To define injection $\iR:\mathcal{V}^j\to\mathcal{V}^{j-1}$, observe that for any given values at the (interior) nodes of $\mathcal{T}^j$ there is a unique function in $\mathcal{V}^j$ with those values.  Also observe that, by our refinement construction, the nodes of $\mathcal{T}^{j-1}$ are nodes of $\mathcal{T}^j$.  For $z\in\mathcal{V}^j$ we define the injection $\iR z$ as the element of $\mathcal{V}^{j-1}$ which has the same nodal values as $z$ at the nodes of $\mathcal{T}^{j-1}$.  Note that injection is monotonic, that is, if $w \le z$ then $\iR w \le \iR z$.

Finally, the monotone restrictions $\maxR,\minR$ are nonlinear maps which maximize and minimize, respectively, nodal values over the supports of coarser-mesh hat functions.  To make the definitions precise, first denote the \emph{nodes} of triangulation $\mathcal{T}^j$ by $x_q^j$ for $q=0,\dots,\tilde{m}_j-1$, and then the corresponding hat functions \cite{Elmanetal2014} by $\psi_q^j$.  (Boundary nodes are included here.)  The monotone restrictions of $z\in\tilde{\mathcal{V}}^j$ are
\begin{align}
\maxR z &= \sum_{p=1}^{\tilde m_{j-1}} \max \{z(x_q^j) \,:\, \psi_p^{j-1}(x_q^j) > 0\}\, \psi_p^{j-1}, \label{eq:fe:monotonerestrictions} \\
\minR z &= \sum_{p=1}^{\tilde m_{j-1}} \min \{z(x_q^j) \,:\, \psi_p^{j-1}(x_q^j) > 0\}\, \psi_p^{j-1}, \notag
\end{align}
with the result in $\tilde{\mathcal{V}}^{j-1}$.  The usual ordering rules for the extended real line $\tilde\RR$ are applied.  Note that if $z\in\tilde{\mathcal{V}}^j$ then $\minR z \le z \le \maxR z$.

To define the constraint subsets, the sets generically denoted $\cK_i$ in Section \ref{sec:cd}, in the current multilevel FE context, suppose $w^J \in \cK^J$ is an admissible fine-level iterate which approximates the solution $u^J$ of VI \eqref{eq:fe:vi}.  From this iterate we define the finest-level \emph{lower and upper defect constraints} \cite{GraeserKornhuber2009}
\begin{equation}
\underline{\chi}^J = \underline{\gamma}^J - w^J, \qquad \overline{\chi}^J = \overline{\gamma}^J - w^J, \label{eq:fe:defectconstraints}
\end{equation}
functions in $\tilde{\mathcal{V}}^J$ satisfying $\underline{\chi}^J \le 0 \le \overline{\chi}^J$.  The motivation here is that a corrected iterate is admissible, $\underline{\gamma}^J \le w^J + z^J \le \overline{\gamma}^J$, if and only if the perturbation $z^J$ is between the defect constraints, $\underline{\chi}^J \le z^J \le \overline{\chi}^J$.

For $j=J,J-1,\dots,1$ we apply monotone restrictions to define the \emph{level defect constraints},
\begin{equation}
\underline{\chi}^{j-1} = \maxR \underline{\chi}^j, \qquad \overline{\chi}^{j-1} = \minR \overline{\chi}^j. \label{eq:fe:chilevels}
\end{equation}
Observe that
\begin{equation}
\underline{\chi}^{J} \le \dots \le \underline{\chi}^0 \le 0 \le \overline{\chi}^0 \le \dots \le \overline{\chi}^J. \label{eq:fe:chiordering}
\end{equation}
A $J=4$ example, with no upper obstacles ($\overline{\chi}^j=+\infty$), is shown in Figure \ref{fig:chiphilevels}.

%REGENERATE using https://github.com/bueler/mg-glaciers:
%$ cd mg-glaciers/py/
%$ [modify decomposition() inside visualize.py to turn off axes and print \gamma for obstacle]
%$ ./obstacle.py -J 5 -jcoarse 1 -random -randommodes 8 -diagnostics -o defect.pdf
%fine level 5 (m=63): using 20 V(1,0) cycles -> 38.750 WU
\begin{figure}
\includegraphics[width=0.65\textwidth]{fixfigs/chiphilevels.pdf}
\caption{The lower defect constraint $\underline{\chi}^J = \underline{\gamma}^J - w^J$ is decomposed using maximum restriction: $\underline{\chi}^{j-1} = \maxR \underline{\chi}^j$.}
\label{fig:chiphilevels}
\end{figure}

We also compute the differences between the level defect constraints:
\begin{equation}
\underline{\phi}^j = \underline{\chi}^j - \underline{\chi}^{j-1}, \qquad \overline{\phi}^j = \overline{\chi}^j - \overline{\chi}^{j-1},  \label{eq:fe:philevels}
\end{equation}
using $\underline{\chi}^{-1}=\overline{\chi}^{-1}=0$.  Again these constraints bracket zero,
\begin{equation}
\underline{\phi}^j \le 0 \le \overline{\phi}^j,  \label{eq:fe:phibrackets}
\end{equation}
but these functions are not ordered as in \eqref{eq:fe:chiordering}, but telescoping sums hold for $j=0,1,\dots,J$:
\begin{equation}
\sum_{i=0}^j \underline{\phi}^i = \underline{\chi}^j, \qquad \sum_{i=0}^j \overline{\phi}^i = \overline{\chi}^j.  \label{eq:fe:telescoping}
\end{equation}

A multilevel V-cycle is defined in the next Section.  For the downward direction we will use the differences $\underline{\phi}^j,\overline{\phi}^j$ as constraints, and for the upward part the defect constraints $\underline{\chi}^j,\overline{\chi}^j$.  That is,
\begin{align}
\mathcal{D}^j = \{v \in \mathcal{V}^j \,:\, \underline{\phi}^j \le v \le \overline{\phi}^j\}, \label{eq:fe:constraintsets} \\
\mathcal{U}^j = \{v \in \mathcal{V}^j \,:\, \underline{\chi}^j \le v \le \overline{\chi}^j\}, \notag
\end{align}
are the \emph{downward} and \emph{upward constraint subsets} of the FE subspaces $\mathcal{V}^j$, respectively.  Observe that the upward sets are larger, $\mathcal{D}^j \subset \mathcal{U}^j$, and in fact by \eqref{eq:fe:telescoping} the set decompositions
\begin{equation}
\mathcal{U}^j = \sum_{i=0}^j \mathcal{D}^i \label{eq:fe:constraintdecomp}
\end{equation}
apply in the same sense as \eqref{eq:constraintdecomp}.  Also note that the finest-level upward constraint set is equivalent to the original constraint set $\mathcal{K}^J$; for all $z^J \in \mathcal{V}^J$,
\begin{equation}
z^J \in \mathcal{U}^J \quad \iff \quad w^J+z^J \in \mathcal{K}^J. \label{eq:fe:finestlevelequivalent}
\end{equation}

FIXME what are the CDs?  what are the decomposition operators?


\section{The nonlinear multilevel CD V-cycle} \label{sec:vcycle}

Our \emph{nonlinear multilevel constraint decomposition} (NMCD) V-cycle algorithm in this section will use the two types of constraint subsets in \eqref{eq:fe:constraintsets}, with distinct sets for the downward and upward parts of the cycle.  We also diverge from the multilevel CD algorithms by Tai \cite{Tai2003} and Gr\"aser \& Kornhuber \cite[Algorithm 4.7]{GraeserKornhuber2009} by following a nonlinear \emph{full approximation scheme} (FAS) philosophy \cite{BrandtLivne2011}.  In the FAS approach here we must coarsen the solution iterate itself, using the injection restriction, in constrast to classical linear multigrid, wherein only the residual and correction are coarsened, and in constrast to the implemented examples in \cite{GraeserKornhuber2009,Tai2003} in which $f$ is linear.

FIXME from here

We will correct the finest-level iterate $w^J\in \mathcal{K}^J$ by adding modifications from each coarser subspace $\mathcal{V}^j$.  By nesting \eqref{eq:fe:nestedspaces} these corrections live in the fine-level subspace $\mathcal{V}^J$, but the corrected iterate needs to be admissible as well.  The accumulated corrections, from the constraint subsets $\mathcal{D}^j$ and $\mathcal{U}^j$, over the whole multigrid V-cycle, compute a new iterate in $\mathcal{K}^J$.

In detail, suppose $y^{j+1} \in \mathcal{D}^{j+1}, \dots, y^J \in \mathcal{D}^J$ are already-computed corrections during the downward part of the V-cycle (Figure \ref{fig:nmcdvcycle}).  The so-far corrected iterate is admissible, namely $w^J + y^J + \dots + y^{j+1} \in \mathcal{K}^J$, because the sum of corrections is in the (finest) defect constraint set, i.e.~$y^J + \dots + y^{j+1} \in \mathcal{U}^J = \{v\ge \chi^J\}$.

\begin{figure}[ht]
\begin{center}
\input{tikz/nmcdvcycle.tex}
\end{center}
\caption{An NMCD V-cycle computes downward corrections $y_j \in \mathcal{D}^j$, but the upward corrections $z_j$ may expand into larger constraint subsets $\mathcal{U}^j$.}
\label{fig:nmcdvcycle}
\end{figure}

When computing the next-coarser correction one must use the coarser-mesh representation of the problem.  As in nonlinear FAS multigrid schemes \cite{BrandtLivne2011,Bruneetal2015,Trottenbergetal2001}, this requires that we coarsen the iterate as well as the residual and the correction, and here injection is used \cite[section 5.3]{Trottenbergetal2001}.  Define
\begin{equation}
g^j = \begin{cases} w^J, & j=J \\
                    \iR(g^{j+1} + y^{j+1}), & j < J.
      \end{cases}  \label{eq:fe:defineg}
\end{equation}
Function $g^j$ is the $j$th-level mesh approximation to the solution of \eqref{eq:fe:vi}, before smoothing and correction at that level.\footnote{If we were to inject the finest-mesh obstacle downward to coarser meshes, $\gamma^{j-1} = \iR \gamma^j$, then, because $\iR$ preserves inequalities and $y^j \in \mathcal{D}^j$, it would follow that $g^j \ge \gamma^j$.  In this sense $g^j$ is admissible on the $j$th-level mesh.  However, following \cite{GraeserKornhuber2009}, our use of defect constraints $\chi^j$ avoids the need to generate the $\gamma^j$.}

Down-smoothing makes progress toward solving the following VI for $y^j \in \mathcal{D}^j$:
\begin{equation}
\ip{f^j(g^j + y^j)}{v-y^j} \ge \ip{\ell^j}{v-y^j} \qquad \text{for all } v\in \mathcal{D}^j. \label{eq:fe:downvi}
\end{equation}
Here $f^j$ is a discretization of $f$ in \eqref{eq:vi},\footnote{Again we diverge from the algorithm by Tai \cite{Tai2003}, who essentially assumes that ``$f(g^j + y^j)$'' is used here.  If $f$ is linear then efficient implementation via that formula is possible, but otherwise one must apply $f^j$.} using quadrature etc.~as addressed for $f^J$ in \eqref{eq:fe:vi}.

For $j<J$ the source function $\ell^j$ is determined by an FAS-type formula, as follows.  First note that the $j+1$ level version of \eqref{eq:fe:downvi} is not solved exactly by the correction $y^{j+1}$.  Fixing $y^{j+1} \in \mathcal{D}^{j+1}$, we would want to find a further correction $y$ so that $g^{j+1}+y^{j+1}+y$ is still admissible, and so that
\begin{equation}
\ip{f^{j+1}(g^{j+1}+y^{j+1}+y)}{v-y^{j+1}-y} \ge \ip{\ell^{j+1}}{v-y^{j+1}-y} \qquad \text{\emph{(notional)}} \label{eq:fe:downvinotional}
\end{equation}
held exactly.  However, in a multilevel method we actually compute this next correction $y$ on the next-coarser level $j$.  To do this we modify the notional finer-level VI \eqref{eq:fe:downvinotional} in five steps which mimic the standard argument for the FAS correction equations for PDEs \cite{BrandtLivne2011,Trottenbergetal2001}:
\begin{enumerate}
\item group $v-y^{j+1}$ as a coarser-level test function $v\in \mathcal{D}^j$,
\item subtract the known $f^{j+1}(g^{j+1}+y^{j+1}) \in (\mathcal{V}^{j+1})'$ from both sides,
\item replace the residual $\ell^{j+1}-f^{j+1}(g^{j+1}+y^{j+1})$ by its restriction,
\item replace $g^{j+1}+y^{j+1}$ where it appears by its restriction,
\item and replace $f^{j+1}$ by its coarser rediscretization $f^j$ where it appears on the left.
\end{enumerate}
Steps \emph{(iii)}--\emph{(v)} are justified, as usual, by the assumption that the residual and the correction should be smooth as a result of the progress made by the smoother on the $j+1$ level.  These steps now yield the precise $j$th-level VI problem, in cluttered form but the same as \eqref{eq:fe:downvi}:
\begin{align}
&\ip{f^j\left(\iR(g^{j+1}+y^{j+1})+y^j\right)}{v-y^j} - \ip{f^j\left(\iR(g^{j+1}+y^{j+1})\right)}{v-y^j} \label{eq:fe:downvicluttered} \\
&\qquad \ge \ip{R\left(\ell^{j+1}-f^{j+1}(g^{j+1}+y^{j+1})\right)}{v-y^j} \notag
\end{align}

Thus we define $\ell^j \in (\mathcal{V}^j)'$ in \eqref{eq:fe:downvi} by the following recursive definition, starting from $\ell^J$ already defined for the finest-level problem \eqref{eq:fe:vi}, and incorporating definition \eqref{eq:fe:defineg}:
\begin{equation}
\ell^j = f^j(g^j) + R\left(\ell^{j+1}-f^{j+1}(g^{j+1}+y^{j+1})\right). \label{eq:fe:levelsource}
\end{equation}
This is identical, even if the notation differs, to the standard FAS formula for a PDE source term, for example equation (8.5b) in \cite{BrandtLivne2011} or equation (5.3.14) in \cite{Trottenbergetal2001}.

At each level in the downward part of the V-cycle (Figure \ref{fig:nmcdvcycle}) we apply a smoother, an iterative partial solver of \eqref{eq:fe:downvi} which rapidly reduces the high-frequencies in the error.  This smoother yields a correction $y^j \in \mathcal{D}^j$.  It is important that we can initialize this iteration using an admissible first guess.  However, by the monotone restriction construction the down-constraint is nonpositive, $\phi^j\le 0$ so $0\in \mathcal{D}^j$, and thus an initial value of $y^j=0$ is admissible.

At the coarsest $j=0$ level we solve \eqref{eq:fe:downvi}, accurately if possible, yielding a correction $y^0$ in the coarsest defect constraint set $\mathcal{D}^0=\mathcal{U}^0 = \{v \ge \chi^0\}$.  Note that the coarsest-level solver may be the same as the other smoothers.

In the upward part of the V-cycle the smoother starts from an initial iterate which has accumulated the coarser-level corrections.  Define $z^0 = y^0$ and
\begin{equation}
z^j = P z^{j-1} + y^j  \label{eq:fe:upwardaccumulation}
\end{equation}
where $P$ is canonical prolongation; stated simply, $z^j = y^j + \dots + y^0$.  By the decompositions $\mathcal{U}^j = \sum_{i=0}^j \mathcal{D}^i$, noted earlier, $z^j \in \mathcal{U}^j$.  This initial $z^j$ is smoothed to generate a result, also denoted $z^j \in \mathcal{U}^j$.  This result approximately solves the same VI as the downward version \eqref{eq:fe:downvi}, except now over a larger constraint set:
\begin{equation}
\ip{f^j(g^j + z^j)}{v-z^j} \ge \ip{\ell^j}{v-z^j} \qquad \text{for all } v\in \mathcal{U}^j. \label{eq:fe:upvi}
\end{equation}
The reason we can solve over $\mathcal{U}^j$ is that, unlike during the downward phase, there is no forthcoming coarser correction which could violate admissibility.

These ideas come together in Algorithm \ref{alg:nmcd}, an NMCD V-cycle.  This procedure acts in-place on the final argument $w^J$, but it leaves the other vector inputs $\ell^J,\gamma^J$ unchanged.

\begin{pseudofloat}[H]
\begin{pseudo} \label{ps:nmcd-vcycle}
\pr{nmcd-vcycle}(\ell^J,\gamma^J;w^J)\text{:} \\+
    $\chi^J = \gamma^J - w^J$ \\
    $g^J = w^J$ \\
    for $j=J$ downto $j=1$ \\+
      $\chi^{j-1} = \maxR \chi^j$ \\
      $\phi^j = \chi^j - P\chi^{j-1}$ \\
      $y^j = 0$ \\
      $\text{\pr{smooth}}^{\text{\id{down}}}(\ell^j,\phi^j,g^j;y^j)$  \ct{smoothing of \eqref{eq:fe:downvi} in $\mathcal{D}^j$}\\
      $g^{j-1} = \iR(g^j + y^j)$ \\
      $\ell^{j-1} = f^{j-1}(g^{j-1}) + R \left(\ell^j - f^j(g^j+y^j)\right)$ \\-
    $y^0 = 0$ \\
    $\text{\pr{solve}}(\ell^0,\chi^0,g^0;y^0)$ \ct{solving of \eqref{eq:fe:downvi} in $\mathcal{D}^0=\mathcal{U}^0$} \\
    $z^0 = y^0$ \\
    for $j=1$ to $j=J$ \\+
      $z^j = P z^{j-1} + y^{j}$ \\
      $\text{\pr{smooth}}^{\text{\id{up}}}(\ell^j,\chi^j,g^j;z^j)$  \ct{smoothing of \eqref{eq:fe:upvi} in $\mathcal{U}^j$} \\-
    $w^J \gets w^J+z^J$
\end{pseudo}
\caption{Nonlinear multilevel constraint decomposition V-cycle for the finest-level VI problem \eqref{eq:fe:vi}, as an in-place method on $w^J$.  $f^j$ denotes a discretization of $f$ in problem \eqref{eq:vi}.}
\label{alg:nmcd}
\end{pseudofloat}

During one application of \pr{nmcd-vcycle}, storage must be allocated for $\chi^j$, $\phi^j$, $g^j$, $y^j$, and $\ell^j$ on each level.  However, line 3 is essentially notational in that the finest-level iterate $g^J$ is the same as $w^J$; no copy is needed.  Similarly, in algorithm lines 13 and 15 no allocation is required because the $y^j$ space can be reused for $z^j$ as the coarser corrections are prolonged and accumulated.  Let $m^j$ be the number of interior nodes in the $j$th-level mesh.  The total storage of \pr{nmcd-vcycle}, excluding any in the smoothers/solver, but including for the original obstacle $\gamma^J$, is about $8 m^J$ real numbers\footnote{That is: $8\approx 5(4/3) + 1$.} under the standard estimate that a 2D, uniform-mesh V-cycle uses $4/3$ the memory of the finest level alone \cite[Section 2.4]{Trottenbergetal2001}.  For comparison, a single-level method needs at least $3 m^J$ storage for the vectors $\ell^J,\gamma^J,w^J$.

\pr{nmcd-vcycle} assumes certain semantics for the smoothers and the coarsest-level solver; the latter might also be a smoother algorithm.  They do in-place modifications of their final arguments, and they do not modify their first three arguments.  The smoothers are in-exact solvers of VIs \eqref{eq:fe:downvi} and \eqref{eq:fe:upvi}, respectively, and they are iterated \id{down} and \id{up} times.  Further smoother implementation details and choices are addressed in the next section.

Repeated application of \pr{nmcd-vcycle} to solve VI \eqref{eq:fe:vi} will not reduce the finest-level residual $f^J(u^J) - \ell^J$ to zero everywhere.  However, recalling that a finite-dimensional VI is equivalent to a \emph{nonlinear complementarity problem} (NCP) \cite{FacchineiPang2003}, \eqref{eq:fe:vi} corresponds to the NCP
\begin{equation}
u^J \ge \gamma^J, \qquad f^J(u^J) \ge \ell^J, \qquad (u^J - \gamma^J)\left(f^J(u^J) - \ell^J\right) = 0.  \label{eq:fe:ncp}
\end{equation}
Thus, at convergence we only expect $f^J(u^J) = \ell^J$ at those locations where the constraint is inactive, namely where $u^J > \gamma^J$.  Therefore, recalling that the hat function at node $x_q^J$ is denoted $\psi_q^J$, for $w^J \in \mathcal{K}^J$ we define the following vector in $\RR^{m^J}$ as the finest-level \emph{nodal residual}:
\begin{equation}
\hat r(w^J)_q = \begin{cases} \ip{f^J(w^J)-\ell^J}{\psi_q^J}, & w^J(x_q^J) > \gamma^J(x_q^J), \\
                                  \min\{\ip{f^J(w^J)-\ell^J}{\psi_q^J},0\}, & w^J(x_q^J) = \gamma^J(x_q^J). \end{cases} \label{eq:cpresidual}
\end{equation}
(Formally, $\hat r:\mathcal{V}^J \to \RR^{m^J}$ is a nonlinear map.)  Observe that $\hat r(w^J)$ is nonzero if there are nodes where the NCP \eqref{eq:fe:ncp} is violated, either because $f^J(w^J)-\ell^J$ remains nonzero on the inactive set or because the same quantity is negative on the active set, and that nodal values of $f^J(w^J)-\ell^J$ are in evaluated in a weak sense, by application to a hat function.

A solver for \eqref{eq:fe:vi} using repeated applications of \pr{nmcd-vcycle} would start with an initial iterate $w^{J,0}$, do V-cycles, and stop once the finest-level nodal residual $\hat r(w^J)$ was sufficiently small.  For example, using tolerances \id{atol}$>0$ and \id{rtol}$>0$ the stopping criterion might be
\begin{equation}
\|\hat r(w^J)\| < \id{atol} \qquad \text{or} \qquad \frac{\|\hat r(w^J)\|}{\|\hat r(w^{J,0})\|} < \id{rtol},
\end{equation}
where $\|\cdot\|$ denotes a norm on $\RR^{m^J}$.  A step tolerance could also be used, measuring the difference between successive iterates.

FIXME in Appendix we show how NMCD reduces to FAS if the obstacle is removed and to MCD if $f$ is assumed to be linear



\section{Smoother implementations} \label{sec:smoothers}

FIXME

Recall that a $j$th-level hat function based at node $x_p^j$ is denoted $\psi_p^j$, for an iterate $g^j$ and an admissible correction $y^j \in \mathcal{D}^j$ we define the \emph{downward nodal residual} as
\begin{equation}
(\hat r_{\text{down}}(y^j))_p = \begin{cases} \ip{f^j(g^j+y^j)-\ell^j}{\psi_p^j}, & y^j(x_p^j) > \phi^j(x_p^j), \\
                                  \min\{\ip{f^j(g^j+y^j)-\ell^j}{\psi_p^j},0\}, & y^j(x_p^j) = \phi^j(x_p^j). \end{cases} \label{eq:dncpresidual}
\end{equation}
The \emph{upward nodal residual} $\hat r_{\text{up}}(z^j)$ for $z^j \in \mathcal{U}^j$ is similarly defined using $\chi^j$.

FIXME stopping criteria for smoother


\section{Results} \label{sec:results}

FIXME show results for 2D Firedrake implementation of NMCD for $p$-Laplacian obstacle problem (Example \ref{ex:plaplacian}), advection-diffusion obstacle problem (Example \ref{ex:advectiondiffusion}), and porous-media obstacle problem (Example \ref{ex:porous})

FIXME observe that up-smoothing is more efficient

% A BRIDGE TOO FAR:  \section{Results for a nonlocal variational inequality}


\bibliography{mcd2}
\bibliographystyle{siam}


\appendix
\section{Reduction of NMCD to unconstrained FAS}

\pr{nmcd-vcycle}, Algorithm \ref{alg:nmcd} in the main text, generalizes two multilevel algorithms, first the FAS multigrid V-cycle for nonlinear PDEs, which itself generalizes the standard linear multigrid V-cycle, and second the MCD V-cycle for linear variational inequalities.  This Appendix helps readers understand \pr{nmcd-vcycle} by identifying the assumptions which reduces it to these existing methods.

To derive the FAS multigrid V-cycle we need to remove the inequality constraints from \pr{nmcd-vcycle}.  To do this, suppose that the finest-level obstacle $\gamma^J$ is very negative, with magnitude greater than $u^J$ or any prospective iterate $w^J$.  Problem \eqref{eq:fe:vi} is then unconstrained, so by replacing ``$v-u^J$'' with an arbitrary $v \in \mathcal{V}^J$ we have the finest-level, weak form problem
\begin{equation}
\ip{f^J(u^J)}{v} = \ip{\ell^J}{v} \qquad \text{for all } v\in \mathcal{V}^J \label{eq:app:fas:pde}
\end{equation}
corresponding to the strong-form PDE $f(u)=\ell$.  Now, \pr{nmcd-vcycle} would generate downward defect constraints by monotone restriction.  However, all that is required for the downward system is that $\chi^{j-1} \ge \chi^j$, so that $\phi^j = \chi^j - P \chi^{j-1} \le 0$.  In the unconstrained case we can assume, more simply, that the fine level defect constraint is a negative constant $\chi^J = -JZ$ for $Z>0$ such that $Z \gg \|u^J\|_\infty$.  Defining $\chi^j = - jZ$, and calculating $\phi^j = \chi^j - P \chi^{j-1} = - Z$, we see that both the downward VI \eqref{eq:fe:downvi} and the upward VI \eqref{eq:fe:upvi} can also be made unconstrained.  It then follows that lines 2, 5, and 6, can be removed from Algorithm \ref{alg:nmcd}, and further that the constraint arguments can be removed from the smoothers and the coarse-level solver.  Because the downward corrections $y^j$ are no longer separately constrained, we may simply refer to the corrected solution $w^j=g^j+y^j$.  The smoothers can be regarded as in-place modifications of $w^j$.  Also recognizing that the initial iterate for the coarse-level smoother is $w^{j-1}=\iR w^j$, the definition of the coarsened source functional can be rewritten
\begin{equation}
\ell^{j-1} = f^{j-1}\left(w^{j-1}\right) + R\left(\ell^j-f^j(w^j)\right). \label{eq:app:fas:levelsource}
\end{equation}
However, now we must be careful that going upward only the correction gets prolonged;compare Remark 5.3.9 in \cite{Trottenbergetal2001}.  We note that the coarse correction produces an updated $w^{j-1}$ (previously denoted $g^{j-1}+y^{j-1}$).  If $w^j$ denotes the result of down-smoothing, and if $z^j = y^j + P z^{j-1} = y^j + y^{j-1} + \dots + y^0$ is the sum of all prolonged corrections, we want to set $w^j = g^j + z^j = (g^j+y^j) + P(y^{j-1} + \dots + y^0)$ as the iterate going into the up-smoother.  For this we write $w^j \gets w^j + P(w^{j-1} - \iR w^j)$.  With these modifications we get pseudocode in Algorithm \ref{alg:fas}.

\begin{pseudofloat}[H]
\begin{pseudo} \label{ps:fas-vcycle}
\pr{fas-vcycle}(\ell^J; w^J)\text{:} \\+
    for $j=J$ downto $j=1$ \\+
      $\text{\pr{smooth}}^{\text{\id{down}}}(\ell^j;w^j)$ \\
      $w^{j-1} = \iR w^j$ \\
      $\ell^{j-1} = f^{j-1}(w^{j-1}) + R \left(\ell^j - f^j(w^j)\right)$ \\-
    $\text{\pr{solve}}(\ell^0;w^0)$ \\
    for $j=1$ to $j=J$ \\+
      $w^j \gets w^j + P (w^{j-1} - \iR w^j)$ \\
      $\text{\pr{smooth}}^{\text{\id{up}}}(\ell^j;w^j)$ \\-
\end{pseudo}
\caption{An FAS V-cycle results from removing the inequality constraints from \pr{nmcd-vcycle}, and also grouping $w^j=g^j+y^j$ (downward) and $w^j=g^j+z^j$ (upward).}
\label{alg:fas}
\end{pseudofloat}

a negative constant, $\gamma^J=-JZ$, where $Z>0$ is larger than the magnitude of $u^J$, or of any prospective iterate $w^J$, say $Z > \|w^J\|_\infty$.

FIXME get Algorithm 14 \cite{Bruneetal2015}


\section{Reduction of NMCD for linear operators}

FIXME by assuming $f$ is linear, reduce to Algorithm 4.7 in \cite{GraeserKornhuber2009} but with $\text{V}(\nu_1,\nu_2)$ cycles

\end{document}

