\documentclass[letterpaper,final,12pt,reqno]{amsart}

\usepackage[total={6.3in,9.2in},top=1.1in,left=1.1in]{geometry}

\usepackage{times,bm,bbm,empheq,fancyvrb,graphicx,amsthm,amssymb}
\usepackage[dvipsnames]{xcolor}
\usepackage{longtable}
\usepackage{booktabs}

\usepackage{tikz}
\usetikzlibrary{decorations.pathreplacing}

\usepackage[kw]{pseudo}
\pseudoset{left-margin=15mm,topsep=5mm,idfont=\texttt}

\usepackage{float}

% hyperref should be the last package we load
\usepackage[pdftex,
colorlinks=true,
plainpages=false, % only if colorlinks=true
linkcolor=blue,   % ...
citecolor=Red,    % ...
urlcolor=black    % ...
]{hyperref}

\renewcommand{\baselinestretch}{1.05}

\allowdisplaybreaks[1]  % allow display breaks in align environments, if they avoid major underfulls

\newtheoremstyle{cstyle}% name
  {5pt}% space above
  {5pt}% space below
  {\itshape}% body font
  {}% indent amount
  {\itshape}% theorem head font
  {.}% punctuation after theorem head
  {.5em}% space after theorem head
  {\thmname{#1}\thmnumber{ #2}\thmnote{ (#3)}}% theorem head spec
\theoremstyle{cstyle}

\newtheorem{theorem}{Theorem}
\newtheorem{lemma}[theorem]{Lemma}
\newtheorem{assumptions}[theorem]{Assumptions}

\newtheoremstyle{cstyle*}% name
  {5pt}% space above
  {5pt}% space below
  {\itshape}% body font
  {}% indent amount
  {\itshape}% theorem head font
  {.}% punctuation after theorem head
  {.5em}% space after theorem head
  {\thmname{#1}}% theorem head spec
\theoremstyle{cstyle*}
\newtheorem{assumptions*}{Assumptions}

\newtheoremstyle{dstyle}% name
  {5pt}% space above
  {5pt}% space below
  {}%{\itshape}% body font
  {}% indent amount
  {\itshape}% theorem head font
  {.}% punctuation after theorem head
  {.5em}% space after theorem head
  {\thmname{#1}\thmnumber{ #2}\thmnote{ (#3)}}% theorem head spec
\theoremstyle{dstyle}

\newtheorem{definition}[theorem]{Definition}
\newtheorem{example}[theorem]{Example}

% numbering
\numberwithin{equation}{section}
\numberwithin{figure}{section}
\numberwithin{table}{section}
\numberwithin{theorem}{section}

\newcommand{\eps}{\epsilon}
\newcommand{\RR}{\mathbb{R}}

\newcommand{\grad}{\nabla}
\newcommand{\Div}{\nabla\cdot}
\newcommand{\trace}{\operatorname{tr}}

\newcommand{\hbn}{\hat{\mathbf{n}}}

\newcommand{\bb}{\mathbf{b}}
\newcommand{\be}{\mathbf{e}}
\newcommand{\bbf}{\mathbf{f}}
\newcommand{\bg}{\mathbf{g}}
\newcommand{\bn}{\mathbf{n}}
\newcommand{\br}{\mathbf{r}}
\newcommand{\bu}{\mathbf{u}}
\newcommand{\bv}{\mathbf{v}}
\newcommand{\bw}{\mathbf{w}}
\newcommand{\bx}{\mathbf{x}}
\newcommand{\bF}{\mathbf{F}}
\newcommand{\bV}{\mathbf{V}}
\newcommand{\bX}{\mathbf{X}}
\newcommand{\bxi}{\bm{\xi}}
\newcommand{\bzero}{\bm{0}}

\newcommand{\cK}{\mathcal{K}}
\newcommand{\cV}{\mathcal{V}}

\newcommand{\rhoi}{\rho_{\text{i}}}

\newcommand{\ip}[2]{\left<#1,#2\right>}

\newcommand{\mR}{R^{\bm{\oplus}}}
\newcommand{\iR}{R^{\bullet}}

\newcommand{\nn}{{\text{n}}}
\newcommand{\pp}{{\text{p}}}
\newcommand{\qq}{{\text{q}}}
\newcommand{\rr}{{\text{r}}}

\newcommand{\supp}{\operatorname{supp}}
\newcommand{\Span}{\operatorname{span}}


\begin{document}
\appendix

\begin{quote}
\emph{The Lemmas in this former appendix are correct, but I do not actually see how to prove the claimed Theorem.}
\end{quote}

\section{Porous-medium problems are never monotone}

The property of $p$-coercivity would seem to be a mild generalization of uniform ellipticity, if we include its quasi-linear meaning \cite[Section 8.3]{Evans2010}, and weak-form operators with degenerate quasi-linear type (e.g.~$p$-Laplacian) or certain lower-order terms (e.g.~advection-diffusion) can be monotone and coercive.  However, even the weaker property of monotonicity is apparently ``fragile'', as explained next, while ellipticity is more robust.  We attempt to show that no nontrivial porous-medium type functional is monotone, nor $p$-coercive.  We have not found any such explanation in the literature, but the observation of non-monotonicity in \cite{RocknerWang2008} is related.

\begin{lemma}  Suppose $a,b$ are real numbers.  Then $(ax-by)\cdot(x-y) \ge 0$ for all $x,y\in\RR^d$ if and only if $a=b\ge 0$.
\end{lemma}

\begin{proof}
If $a=b\ge 0$ then $(ax-by)\cdot(x-y) = a\|x-y\|^2 \ge 0$.

Conversely, if $a< 0$ then choose $y=0$ and $x\ne 0$ to get $(ax-by)\cdot(x-y) = a\|x\|^2<0$, a negative value from the form.  A similar strategy gives a negative value if $b<0$.  So suppose $b > 0$, $a\ge 0$, and $a\ne b$.  Consider $y=mx$ for $m$ real and define the convex quadratic scalar function $f(m) = (a-bm)(1-m)$ so that $(ax-by)\cdot(x-y) = f(m) \|x\|^2$.  We minimize $f$, giving $f(m^*) = -b(a-b)^2/4 < 0$ for $m^*=(a+b)/(2b)$.  Thus there exists $x,y$ so that $(ax-by)\cdot(x-y)<0$.  A similar argument applies if $b=0$ and $a>0$.
\end{proof}

A version of the same idea applies if $a,b$ are replaced by matrices.  Consider the expression $(Ax - By)\cdot (x-y)$ where $A,B \in \RR^{d\times d}$ are symmetric matrices and $x,y\in \RR^d$.  This expression is the quadratic form for a certain symmetric block matrix.  Regarding $x,y$ as column vectors:
\begin{equation}
(Ax - By)\cdot (x-y) = \begin{bmatrix} x^\top & y^\top \end{bmatrix} M_{A,B} \begin{bmatrix} x \\ y \end{bmatrix} \label{eq:blockmatrix}
\end{equation}
where
	$$M_{A,B} = \begin{bmatrix} A & -\frac{1}{2}(A+B) \\
                                -\frac{1}{2}(A+B) & B \end{bmatrix}$$
is a symmetric $2d \times 2d$ matrix.  Clearly \eqref{eq:blockmatrix} is only bounded below if $M_{A,B}$ is nonnegative definite.  Stated as the following Lemma, the important point here is that exact equality $A=B$ is required for nonnegativity of the expression; knowing $A$ is close to $B$ is not helpful.

\begin{lemma}  Suppose $A,B \in \RR^{d\times d}$ are symmetric matrices which commute.  Then $(Ax - By)\cdot (x-y)\ge 0$ for all $x,y\in \RR^d$ if and only if $A=B$ and $A$ is nonnegative definite. \end{lemma}

\newcommand{\sbvec}[2]{\left[\begin{smallmatrix} #1 \\ #2 \end{smallmatrix}\right]}
\newcommand{\sbmat}[4]{\left[\begin{matrix} #1 & #2 \\ #3 & #4 \end{matrix}\right]}

\begin{proof}
If $A=B$ is nonnegative definite then $(Ax - By)\cdot (x-y) = z^\top A z\ge 0$ where $z=x-y$.

Conversely, since $A$ is symmetric there exists an orthogonal basis $\{x_i\}$ of eigenvectors, with $Ax_i = \lambda_i x$.  By commutativity these are also eigenvectors of $B$, with $Bx_i = \mu_i x_i$.  Note $\lambda_i,\mu_i$ are real.

Let $\big\{\sbvec{x_1}{x_1}$, $\sbvec{x_1}{-x_1}$, $\dots$, $\sbvec{x_d}{x_d}$, $\sbvec{x_d}{-x_d}\big\}$ be an ordered basis of $\RR^{2d}$.  In this basis $M_{A,B}$ is block diagonal with symmetric $2\times 2$ diagonal blocks
   $$\sbmat{0}{\frac{1}{2}(\lambda_i-\mu_i)}{\frac{1}{2}(\lambda_i-\mu_i)}{\lambda_i+\mu_i}.$$
Thus the eigenvalues of $M_{A,B}$ come in pairs $\frac{1}{2}(\lambda_i + \mu_i) \pm \sqrt{\frac{1}{2}(\lambda_i^2 + \mu_i^2)}$ for $i=1,\dots,d$.  By the strict concavity of the square root, $\sqrt{\frac{1}{2}(\lambda_i^2 + \mu_i^2)} > \frac{1}{2}(|\lambda_i| + |\mu_i|)$ if $\lambda_i\ne \mu_i$.  Thus if any eigenvalues of $A$ differ from those of $B$, i.e.~$\lambda_i\ne \mu_i$ for any $i$, then there exists a negative eigenvalue of $M_{A,B}$.
\end{proof}

It is not clear if the hypotheses of symmetry and commutativity can be removed.

I DO NOT KNOW HOW TO PROVE THE FOLLOWING CLAIM:

\begin{theorem}
Suppose $\phi:[0,\infty) \to (0,\infty)$ is continuous.  For $\Omega \subset \RR^d$ open and nonempty, $\cV = H_0^1(\Omega)$, and $g\in \cV'$, consider the following functional on $\cK = \{v\ge 0\} \subset \cV$:
\begin{equation}
\ip{f(u)}{v} = \int_\Omega \phi(u) \grad u\cdot \grad v\,dx - \ip{g}{v}.  \label{eq:porousagain}
\end{equation}
If $\phi$ is not constant then $f$ is not monotone.
\end{theorem}

\begin{proof}
Suppose $\phi(\alpha)\ne \phi(\beta)$ for $\alpha > 0$ and $\beta \ge 0$.  (That is, suppose $\phi$ is not constant.)  For $x\in\Omega$, construct $u\in \cK$ smooth so that $u(x)=\alpha$ and $\xi = \grad u(x) \ne 0$.  Using the first Lemma, choose $\eta \in \RR^d$ so that $(\phi(\alpha) \xi - \phi(\beta) \eta) \cdot (\xi - \eta) < 0$.  FIXME NOT SURE IF THIS IS GOING TO WORK
\end{proof}

The main idea is that even the slightest variation in the coefficient $\phi(u)$ violates monotonicity.  A strong and tight ellipticity hypothesis, e.g.~the existence of $c_0>0$ and $\eps>0$ so that $c_0 \le \phi(u) \le (1+\eps)c_0$, will not imply monotonicity.  Contrast the nontrivial coefficient dependence on $\grad u$ in the $p$-coercive $p$-Laplacian example.

The above Theorem, if true, shows that no nontrivial functional \eqref{eq:porousagain} is monotone, nor is it ever $p$-coercive for any $p>1$.

\bibliography{../mcd2}
\bibliographystyle{siam}

\end{document}

