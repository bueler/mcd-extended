\documentclass[svgnames,
               hyperref={colorlinks,citecolor=DeepPink4,linkcolor=FireBrick,urlcolor=Maroon},
               usepdftitle=false]  % see \hypersetup{} below
               {beamer}

\mode<presentation>{
  \usetheme{Madrid}
  %\usecolortheme{seagull}
  \usecolortheme{seagull}
  \setbeamercovered{transparent}
  \setbeamerfont{frametitle}{size=\large}
}

\setbeamercolor*{block title}{bg=red!10}
\setbeamercolor*{block body}{bg=red!5}

%\usepackage[svgnames]{xcolor}
\usepackage{hyperref}
\hypersetup{
    pdftitle = {Toward nonlinear multigrid for nonlinear and nonlocal variational inequalities},
    pdfauthor = {Ed Bueler},
    pdfsubject = {},
    pdfkeywords = {}
}

\usepackage[english]{babel}
\usepackage[latin1]{inputenc}
\usepackage{times}
\usepackage[T1]{fontenc}
\usepackage{empheq,bm,xspace,fancyvrb,soul}
\usepackage{tikz}
\usetikzlibrary{shapes,arrows.meta,decorations.markings,decorations.pathreplacing,fadings,positioning}
\usepackage[kw]{pseudo}
\pseudoset{left-margin=15mm,topsep=5mm,label=,idfont=\texttt,st-left=,st-right=}

\makeatletter
%\newcommand\notsotiny{\@setfontsize\notsotiny\@vipt\@viipt}
\newcommand\notsotiny{\@setfontsize\notsotiny\@viipt\@viiipt}
\makeatother

\newcommand{\eps}{\epsilon}
\newcommand{\RR}{\mathbb{R}}

\newcommand{\grad}{\nabla}
\newcommand{\Div}{\nabla\cdot}
\newcommand{\trace}{\operatorname{tr}}

\newcommand{\hbn}{\hat{\mathbf{n}}}

\newcommand{\bb}{\mathbf{b}}
\newcommand{\be}{\mathbf{e}}
\newcommand{\bbf}{\mathbf{f}}
\newcommand{\bg}{\mathbf{g}}
\newcommand{\bn}{\mathbf{n}}
\newcommand{\bq}{\mathbf{q}}
\newcommand{\br}{\mathbf{r}}
\newcommand{\bu}{\mathbf{u}}
\newcommand{\bv}{\mathbf{v}}
\newcommand{\bw}{\mathbf{w}}
\newcommand{\bx}{\mathbf{x}}

\newcommand{\bF}{\mathbf{F}}
\newcommand{\bQ}{\mathbf{Q}}
\newcommand{\bU}{\mathbf{U}}
\newcommand{\bV}{\mathbf{V}}
\newcommand{\bX}{\mathbf{X}}

\newcommand{\btau}{\bm{\tau}}
\newcommand{\bxi}{\bm{\xi}}

\newcommand{\bzero}{\bm{0}}

\newcommand{\rhoi}{\rho_{\text{i}}}

\newcommand{\ip}[2]{\left<#1,#2\right>}

\newcommand{\nn}{{\text{n}}}
\newcommand{\pp}{{\text{p}}}
\newcommand{\qq}{{\text{q}}}
\newcommand{\rr}{{\text{r}}}

\newcommand{\bus}{\bu|_s}
\newcommand{\oo}[1]{\displaystyle O\left(#1\right)}
\newcommand{\sold}{s_{\text{o}}}

\newcommand{\maxR}{R^{\bm{\oplus}}}
\newcommand{\minR}{R^{\bm{\ominus}}}
\newcommand{\iR}{R^{\bullet}}


\title[Multigrid for nonlinear VI]{Toward nonlinear multigrid \\ for nonlinear and non-local variational inequalities}

%\subtitle{\emph{x}}

\author{Ed Bueler}

\institute[UAF]{University of Alaska Fairbanks}

\date[]{February 2023}

%\titlegraphic{\begin{picture}(0,0)
%    \put(0,180){\makebox(0,0)[rt]{\includegraphics[width=4cm]{figs/software.png}}}
%  \end{picture}
%}

\titlegraphic{\hfill \includegraphics[width=0.15\textwidth]{images/uafbw.png}}

%% to start section counter at 0 see
%% https://tex.stackexchange.com/questions/170222/change-the-numbering-in-beamers-table-of-content


\begin{document}
\beamertemplatenavigationsymbolsempty

%\begin{frame}
%  \maketitle
%\end{frame}

{
  %\usebackgroundtemplate{\includegraphics[width=\paperwidth]{images/gray-british-clark2022.png}}
  \begin{frame}
    \titlepage
  \end{frame}
}

\begin{frame}{Outline}
  \tableofcontents[hideallsubsections]
\end{frame}


\section{variational inequalities (VIs)?}

\begin{frame}{example: a classical obstacle problem}

\includegraphics[width=0.55\textwidth]{images/obstacle65.pdf} \qquad \includegraphics[width=0.35\textwidth]{images/obstacle-sets.png}

\bigskip
\only<1>{
\begin{itemize}
\item \emph{problem.} on a domain $\Omega \subset \RR^2$, find the displacement $u(x)$ of a membrane, with fixed value $u = g$ on $\partial \Omega$, above an \emph{obstacle} $\psi(x)$, which minimizes the elastic energy
    $$J(v) = \int_\Omega \frac{1}{2} |\grad v|^2 - f\, v$$
\item shown above: \quad $\Omega=(-2,2)^2$, $\psi(x)$ a hemisphere, $f(x)=0$
\end{itemize}

\phantom{x}
}
\only<2>{
\begin{itemize}
\item i.e.~constrained optimization over an \emph{admissible set}
	$$\mathcal{K} = \left\{v \in H^1(\Omega) \,:\, v\big|_{\partial \Omega} = g \text{ and } v \ge \psi\right\}$$
\item $J'(u)$ points directly into $\mathcal{K}$, the {\color{FireBrick} \emph{variational inequality} (VI)}:
    $$\ip{J'(u)}{v-u} = \int_\Omega \grad u\cdot \grad (v-u) - f (v-u) \ge 0 \quad \text{for all } v \in \mathcal{K}$$
\end{itemize}
}
\only<3>{
\begin{itemize}
\item the solution defines \emph{active} $A_u = \{u = \psi\}$ and \emph{inactive} $R_u = \{u> \psi\}$ subsets of $\Omega$, and a \emph{free boundary} $\Gamma_u=\partial R_u \cap \Omega$
\item naive strong form would pose the problem in terms of its solution:
\begin{align*}
-\grad^2 u &= f \quad \text{ on $R_u$} \\
u &= \psi \quad \text{ on $A_u$}
\end{align*}

\phantom{x}

\phantom{x}
\end{itemize}
}
\only<4>{
\begin{itemize}
\item the \emph{complementarity problem} (CP) is meaningful as a strong form:
\begin{align*}
u - \psi &\ge 0 \\
-\grad^2 u - f &\ge 0 \\
(u - \psi)(-\grad^2 u - f) &= 0
\end{align*}
   \begin{itemize}
   \item[$\circ$] for optimization problems: \quad CP $=$ KKT conditions
   \end{itemize}

\phantom{x}

\phantom{x}
\end{itemize}
}
\end{frame}


\begin{frame}{general variational inequalities}

\begin{itemize}
\item let $\mathcal{K}$ be a closed and convex subset of a Banach space $\mathcal{V}$
\item suppose $F:\mathcal{K} \to \mathcal{V}'$ is a continuous, generally nonlinear operator
    \begin{itemize}
    \item[$\circ$] $F$ may be defined only on $\mathcal{K}$
    \item[$\circ$] $F$ may \emph{not}\, be the derivative of an objective function $J$
    \end{itemize}
\item the general problem {\color{FireBrick} VI($F$,$\mathcal{K}$)} is
	$${\color{FireBrick} \ip{F(u)}{v-u} \ge 0 \quad \text{ for all } v \in \mathcal{K}}$$
\item when $\mathcal{K}$ is nontrivial the problem {\color{FireBrick} VI($F$,$\mathcal{K}$)} is nonlinear \emph{even when $F$ is a linear operator}
\end{itemize}
\end{frame}


\begin{frame}{VI $=$ ``constrained equation''}

\begin{center}
\begin{tabular}{l|l}
\begin{minipage}[t][16mm][t]{0.4\textwidth}
unconstrained optimization:
$$\min_{u\in\mathcal{V}} J(u)$$
\end{minipage}
&
\begin{minipage}[t][16mm][t]{0.4\textwidth}
constrained optimization:
$$\min_{u\in\mathcal{K}} J(u)$$
\end{minipage}
\\ \hline
\begin{minipage}[t][16mm][t]{0.4\textwidth}
equation for $u \in \mathcal{V}$: {\LARGE \strut}

$$F(u)=0$$
\end{minipage}
&
\begin{minipage}[t][16mm][t]{0.4\textwidth}

\vspace{-2mm}
{\color{FireBrick} VI} for $u \in \mathcal{K}$:
$${\color{FireBrick} \ip{F(u)}{v-u} \ge 0 \quad \forall v \in \mathcal{K}}$$
\end{minipage}
\end{tabular}
\end{center}
\end{frame}


\begin{frame}{applications of VIs}

\begin{itemize}
\item elastic contact, Signorini problems (e.g.~Kikuchi \& Oden 1988)
\item viscous contact problems (de Diego et al.~2022)
\item pricing of American options in the Black-Scholes model
\item the geometry of glaciers \hfill $\longleftarrow$ \emph{more soon}

\bigskip
\item first-semester calculus
\end{itemize}

\begin{center}
\includegraphics[width=0.9\textwidth]{images/calcone.png}
\end{center}
\end{frame}


\AtBeginSection[]
{
  \begin{frame}{Outline}
    \tableofcontents[currentsection]
  \end{frame}
}

\section{full approximation scheme (FAS) multigrid for PDEs?}

\begin{frame}{nonlinear 2-grid scheme}

\begin{center}
$\Omega^h$\, \includegraphics[height=0.16\textheight]{images/fine-grid.png} \hspace{25mm} \includegraphics[height=0.16\textheight]{images/coarse-grid.png} \,$\Omega^H$
\end{center}

\only<1>{
\begin{itemize}
\item consider a nonlinear elliptic PDE problem:
	$$F(u) = \ell$$

	\begin{itemize}
	\item for example, $F : \mathcal{V} \to \mathcal{V}'$ for $\mathcal{V}=H^1(\Omega)$, with $\ell\in \mathcal{V}'$
	\end{itemize}
\item discretization gives algebraic system on fine grid $\Omega^h$:
    $$F^h(u^h) = \ell^h$$
\item suppose $w^h$ yields residual norm $\|\ell^h - F^h(w^h)\| > \text{TOL}$
\end{itemize}

\phantom{x}
}
\only<2-3>{
\begin{itemize}
\item how can we improve $w^h$ \emph{without} globally linearizing $F^h$?

(are there alternatives to Newton's method?)
\item note the \emph{residual} $r^h(w^h) = \ell^h - F^h(w^h)$ is computable,

while the \emph{error} $e^h = w^h-u^h$ is unknown
\item the residual definition can be rewritten
    $$F^h(u^h) - F^h(w^h) = r^h(w^h)$$
\only<2>{\item for $F^h$ linear, try to solve this \emph{error equation} $F^h(e^h) = -r^h(w^h)$ for $\tilde e^h$, and correct $w^h \leftarrow w^h-\tilde e^h$ to improve $w^h$?}
\only<3>{\item \st{for $F^h$ linear, try to solve this \emph{error equation} $F^h(e^h) = -r^h(w^h)$ for $\tilde e^h$, and correct $w^h \leftarrow w^h-\tilde e^h$ to improve $w^h$?}}
\end{itemize}
}
\only<4>{
\begin{itemize}
\item goal: use a coarser mesh to estimate the error
\item \emph{nodewise problem}: for $\psi_i^h$ a hat function or dof, solve for $c\in\RR$:
	$${\color{FireBrick} \phi_i(c) = r^h(w^h + c \psi_i^h)[\psi_i^h] = 0}$$
\item sweeping through and solving nodewise problems is a \emph{smoother}
    \begin{itemize}
    \item[$\circ$] Fourier analysis on linear PDEs shows smoothing property
    \item[$\circ$] post-smoothing, $e^h$ and $r^h(w^h)$ have smaller high-frequencies
    \end{itemize}
\item Brandt (1977): post-smoothing, $F^h(u^h) - F^h(w^h) = r^h(w^h)$ should be accurately approximate-able on a coarser grid
\end{itemize}
}
\only<5>{
\begin{itemize}
\item \emph{full approximation storage} (FAS) equation:
	$${\color{FireBrick} F^H(w^H) - F^H(\iR w^h) = R \, r^h(w^h)}$$

    \begin{itemize}
    \item[$\circ$] $\iR:\mathcal{V}^h \to \mathcal{V}^H$ is \emph{injection}
    \item[$\circ$] $R:(\mathcal{V}^h)' \to (\mathcal{V}^H)'$ is \emph{canonical restriction}
    \item[$\circ$] if $w^h=u^h$ exactly then $w^H = \iR w^h$ by well-posedness
    \end{itemize}

\item rewritten: \quad ${\color{FireBrick} F^H(w^H) = \ell^H}$ where ${\color{FireBrick} \ell^H = F^H(\iR w^h) + R\, r^h(w^h)}$
\end{itemize}
}
\end{frame}


\begin{frame}{FAS 2-grid solver}

\begin{align*}
&\text{smooth by sweeps over grid:} & &w^h \leftarrow \left[\phi_i(c) = 0 \,\forall i\right] \\
&\text{restrict:}                   & &\ell^H = F^H(\iR w^h) + R\, r^h(w^h) \\
&\text{solve coarse:}                      & &F^H(w^H) = \ell^H \\
&\text{correct}:                    & &w^h \leftarrow w^h + P(w^H - \iR w^h) \\
&\text{smooth by sweeps over grid:} & &w^h \leftarrow \left[\phi_i(c) = 0 \,\forall i\right]
\end{align*}

\bigskip
{\small
\begin{itemize}
\item $P: \mathcal{V}^H \to \mathcal{V}^h$ is \emph{prolongation}
\item recall: \quad $\phi_i(c) = r^h(w^h + c \psi_i^h)[\psi_i^h]$
\item restrict$+$(solve coarse)$+$correct \, $=$ \, \emph{coarse grid correction}
\end{itemize}
}
\end{frame}


\begin{frame}{nonlinear multigrid by FAS V-cycle or F-cycle}

\bigskip

\includegraphics[height=0.15\textheight]{images/mg-grids.png}

\bigskip

\hfill \mbox{
\includegraphics[height=0.15\textheight]{images/mg-vcycle.png} \quad
\includegraphics[height=0.15\textheight]{images/mg-fcycle.png}
}

\vspace{-3mm}

{\small
\begin{pseudo}
\pr{fas-vcycle}$(\ell^J;w^J)$: \\+
    for $j=J$ downto $j=1$ \\+
      $\text{\pr{smooth}}^{\text{\id{down}}}(\ell^j; w^j)$ \\
      $w^{j-1} \gets \iR w^j$ \\
      $\ell^{j-1} = F^{j-1}(w^{j-1}) + R \left(\ell^j - F^j(w^j)\right)$ \\-
    $\text{\pr{solve}}(\ell^0;w^0)$ \\
    for $j=1$ to $j=J$ \\+
      $w^j \gets w^j + P (w^{j-1} - \iR w^j)$ \\
      $\text{\pr{smooth}}^{\text{\id{up}}}(\ell^j;w^j)$ \\-
\end{pseudo}
}
\end{frame}


\begin{frame}{does it work?}

\begin{itemize}
\item FAS multigrid works well on the right nonlinear PDE problem!
\item example: Liouville-Bratu equation\footnote{exact solution by Liouville (1853) makes a nice test case}
    $$-\nabla^2 u - e^u = 0$$
with Dirichlet boundary conditions on $\Omega=(0,1)^2$
\item minimal problem-specific code:
    \begin{enumerate}
    \item[1.] residual evaluation on grid level: $F^j(\cdot)$
    \item[2.] pointwise smoother: $\phi_i(c) = 0 \,\forall i$
        \begin{itemize}
        \item[$\circ$] e.g.~nonlinear Jacobi or Gauss-Seidel iteration
        \end{itemize}
    \item[3.] coarse solve can be same as smoother, or use Newton etc.
    \end{enumerate}
\end{itemize}
\end{frame}


\begin{frame}[fragile]
\frametitle{multigrid solver composition in PETSc}

\begin{itemize}
\item implemented using an FD discretization and PETSc\footnote{Portable Extensible Toolkit for Scientific computing \quad \href{https://petsc.org/release/}{\texttt{petsc.org}} \quad \includegraphics[height=3mm]{images/petsc.png}}
    \begin{itemize}
    \item[$\circ$] FAS multigrid is a nonlinear solver (SNES) type
    \item[$\circ$] implement in C/Fortran/python/Julia(?)
    \end{itemize}
\item multigrid solvers in PETSc are \emph{composed} from smoothers on each level, and a solver on the coarse mesh
\item FAS F-cycle run:
\end{itemize}
\begin{Verbatim}[xleftmargin=15mm,fontsize=\scriptsize]
./bratu -da_grid_x 5 -da_grid_y 5 -da_refine J \
    -snes_rtol 1.0e-12 \
    -snes_type fas \
    -snes_fas_type full \
    -fas_levels_snes_type ngs \
    -fas_levels_snes_ngs_sweeps 2 \
    -fas_levels_snes_ngs_max_it 1 \
    -fas_levels_snes_norm_schedule none \
    -fas_coarse_snes_type ngs \
    -fas_coarse_snes_max_it 1 \
    -fas_coarse_snes_ngs_sweeps 4 \
    -fas_coarse_snes_ngs_max_it 1 
\end{Verbatim}

%NOT IN MY BOOK \hfill \includegraphics[width=8mm]{images/frontcover.jpg}
\end{frame}


\begin{frame}{Bratu model problem: optimality}

\begin{columns}
\begin{column}{0.45\textwidth}
\begin{itemize}
\item observed optimality:
\begin{align*}
\text{flops} &= O(N^1) \\
\text{exp evaluations} &= O(N^1) \\
\text{processor time} &= O(N^1)
\end{align*}
\item<1-> up to almost $N=10^8$ dofs ($J=11$)
\item<2> compare $\approx 20\,\mu\,\text{s}/\text{N}$ for Poisson equation using Firedrake $P_1$ elements and geometric multigrid
    \begin{itemize}
    \item<2>[$\circ$] on my 2022 laptop
    \end{itemize}
\end{itemize}
\end{column}
\begin{column}{0.55\textwidth}
\includegraphics<1>[width=\textwidth]{images/bratu-exps.png}

\includegraphics<2>[width=\textwidth]{images/bratu-time.png}
\end{column}
\end{columns}
\end{frame}


\begin{frame}{benefits of FAS multigrid for nonlinear PDEs?}

\begin{itemize}
\item \alert{benefits} of FAS multigrid?
    \begin{enumerate}
    \item[1.] minimal code, esp.~in from-scratch implementations
        \begin{itemize}
        \item[$\circ$] just write residual plus pointwise smoother!
        \end{itemize}
    \item[2.] composition with nonlinear preconditioners (Brune et al.~2015)
    \end{enumerate}

\bigskip
\item \alert{disadvantages?}
    \begin{enumerate}
    \item[1.] Firedrake/FENiCs \emph{do} automatically provide linearizations from UFL statements of weak forms
    \item[2.] small literature of convergence or descriptive performance for FAS (Trottenberg et al.~(2001), Reusken (1987))
    \item[3.] not enough tutorial literature?
    \end{enumerate}
\end{itemize}
\end{frame}


\section{the nonlocal VI for a fluid layer in a climate}

\begin{frame}{problem: fluid layer in a climate}

\begin{itemize}
\item let's not get stuck on textbook example problems!
\item multigrid for a real-world VI problem?
\item consider the steady state of a viscous layer of surface elevation $s(x,y)$ flowing with velocity $\bu(x,y,z)$ under its own weight, over fixed bed topography with elevation $b(x,y)$, in a \emph{climate} which adds or removes fluid at a signed rate $a(x,y)$ [$\text{m}\,\text{s}^{-1}$]
    \begin{itemize}
    \item[$\circ$] data $a,b$ defined on domain $\Omega \subset \RR^2$ (left figure)
    \end{itemize}
\item geophysical examples: \alert{glaciers and ice sheets}, sea ice, oceans and lakes
\end{itemize}

\bigskip
\hfill \mbox{\includegraphics[height=0.25\textheight]{images/domain-data.png} \hspace{7mm} \includegraphics[height=0.25\textheight]{images/domain-velocity.png}}
\end{frame}


\begin{frame}{example: glacier ice coverage of the Alps in prior climates}

\includegraphics[width=1.02\textwidth]{images/alps-seguinot2018.png}

\vspace{-2mm}
\hfill {\tiny Sequinot et al.~(2018)}

%\vspace{-5mm}
{\footnotesize
\begin{itemize}
\item<2> more ice sheet modeling at my Math.~Geoscience Seminar tomorrow 2pm L5
\end{itemize}
}
\end{frame}


\begin{frame}{strong form}

\begin{itemize}
\item inequality constraint generates free boundary: $s \ge b$
    \begin{itemize}
    \item[$\circ$] ablative climate $a(x) < 0$ forces surface down to bed
    \item[$\circ$] $s\ge b$ $\iff$ thickness $s-b$ must be nonnegative
    \end{itemize}
\item naive strong form of the steady model: % ($\bn_s = \left<-s_x,-s_y,1\right>$ is surface normal):
\begin{align*}
s &\ge b                    & &\text{everywhere in } \Omega \\
-\bu|_s \cdot \bn_s &= a    & &\text{where } s(x) > b(x)
\end{align*}

    \begin{itemize}
    \item[$\circ$] surface velocity $\bu|_s$ is determined by fluid domain geometry
    \item[$\circ$] true $\Phi(s) = -\bu|_s \cdot \bn_s$ is a \emph{non-local} function of $s$
    \end{itemize}
\end{itemize}

\bigskip
\hfill \mbox{\includegraphics[height=0.2\textheight]{images/domain-data.png} \hspace{7mm} \includegraphics[height=0.2\textheight]{images/domain-velocity.png}}
\end{frame}


\begin{frame}{$\Phi(s) = - \bu|_s \cdot \bn_s$ for glacier ice?}

\begin{itemize}
\only<1>{
\item \alert{Stokes model}

solve the Stokes problem for $\bu,p$, then evaluate at surface:
    $$\int_{\Lambda = \{b < z <s\}} 2 \nu(D\bu) D\bu : D\bv - p \Div\bv - (\Div\bu) q - \rhoi \bg \cdot \bv\,d\bx = 0 \quad \forall v,q$$
    $$\Phi(s) = - \bu|_s \cdot \bn_s$$

    \begin{itemize}
    \item[$\circ$] also extend $\Phi(s)$ by 0 to all of $\Omega$
    \item[$\circ$] non-Newtonian viscosity: $\nu(D\bu) = \frac{1}{2} \Gamma |D\bu|^{\pp-2}$ with e.g.~$\pp = \frac{4}{3}$
    \item[$\circ$] well-posed problem for $\bu,p$
    \item[$\circ$] near-optimal solvers available (Isaac et al 2015)
    \end{itemize}
}

\only<2>{
\item \alert{lubrication approximation model}

apply a nonlinear differential operator to $s$:
    $$\Phi(s) = - \frac{\gamma}{\qq} (s-b)^{\qq} |\grad_{\bx} s|^{\qq} - \grad_{\bx} \cdot\left(\frac{\gamma}{\qq+1} (s-b)^{\qq+1} |\grad_{\bx} s|^{\qq-2} \grad_{\bx} s\right)$$
where $\qq = \nn+1$ and $\gamma>0$ is related to ice softness and density
    \begin{itemize}
    \item e.g.~$\qq = 4$
    \item $\Phi(s)$ is a nonlinear differential operator only because membrane (longitudinal) stresses are not balanced in lubrication approximation
    \end{itemize}
}
\end{itemize}
\end{frame}

\begin{frame}{VI for fluid layer in a climate}

\begin{itemize}
\item admissible surface elevations: \quad $\mathcal{K} = \left\{r \in \mathcal{V} \,:\, r \ge b\right\}$

    \begin{itemize}
    \item[$\circ$] $\mathcal{V}$ to be determined by viscosity model\footnote{in lubrication approximation for glaciers, $u^{8/3} \in W^{1,4}(\Omega)$ (Jouvet \& Bueler, 2012)}
    \end{itemize}
\item VI problem:

\end{itemize}
\end{frame}


\begin{frame}{the Stokes problem for ice}

\begin{itemize}
\item non-shallow model solves a Stokes problem at each step:
\begin{align*}
- \nabla \cdot \left(2 \nu_\eps(D\bu)\, D\bu\right) + \nabla p - \rhoi \mathbf{g} &= \bzero && \text{in $\Lambda$} \\
\nabla \cdot \bu &= 0 && \text{''} \\
\btau_b - \bbf(\bu|_b) &= \bzero && \text{on $\Gamma_b$} \\
\bu|_b \cdot \bn_b &= 0 && \text{''} \\
\left(2 \nu_\eps(D\bu) D\bu - pI\right) \bn &= \bzero && \text{on $\Gamma_s$}
\end{align*}
\item this is the \alert{stress balance} (conservation of momentum) problem which determines velocity $\bu$ and pressure $p$
\end{itemize}
\end{frame}


\section{scalable approaches for VIs}

\begin{frame}[fragile]
\frametitle{Newton-multigrid for VIs}

\begin{itemize}
\item multigrid for the classical obstacle problem?
    \begin{itemize}
    \item[$\circ$] a nonlinear problem, though $-\nabla^2$ is linear
    \end{itemize}
\item Newton-multigrid straightforward in PETSc:
\begin{Verbatim}[xleftmargin=13mm,fontsize=\scriptsize]
./obstacle -da_grid_x 3 -da_grid_y 3 \
    -snes_type vinewtonrsls -ksp_type cg \
    -pc_type mg -da_refine J
\end{Verbatim}
    \begin{itemize}
    \item[$\circ$] linear solver applies to inactive variables
    \end{itemize}
\item the outer Newton iteration must determine the active set \alert{before} V-cycle multigrid can provide effective preconditioning
    \begin{itemize}
    \item[$\circ$] growing Newton (SNES) iterations
    \end{itemize}
\end{itemize}


\hfill \mbox{\includegraphics[height=0.2\textheight]{images/obstacle65.pdf} \hspace{15mm}
\includegraphics[height=0.25\textheight]{images/vi-newton-gmg-bad.png}}
\end{frame}


\begin{frame}[fragile]
\frametitle{Newton-multigrid for VIs: nested iteration}

\begin{itemize}
\item applying nested iteration (nonlinear F-cycle) resolves this:

\vspace{2mm}
\begin{Verbatim}[xleftmargin=13mm,fontsize=\scriptsize]
./obstacle -da_grid_x 3 -da_grid_y 3 \
    -snes_type vinewtonrsls -ksp_type cg \
    -pc_type mg
\end{Verbatim}

\vspace{-4.9mm}
\hspace{39mm} {\scriptsize \color{FireBrick} \texttt{-snes\_grid\_sequence J}}

\vspace{2mm}
    \begin{itemize}
    \item[$\circ$] constant Newton (SNES) iterations
    \item[$\circ$] grid-independent flops/time restored
    \item[$\circ$] optimal $O(N^1)$ flops
    \item[$\circ$] FIXME example in my book
    \end{itemize}

\vspace{-12mm}
\hfill \includegraphics[width=0.25\textwidth]{images/mg-fcycle.png}
\end{itemize}

\bigskip
\hspace{10mm} \includegraphics[height=0.25\textheight]{images/vi-newton-gmg-good.png} 

\vspace{-22mm}
\hfill \includegraphics[height=0.45\textheight]{images/obstacle-flops-per-n.png}
\end{frame}



\begin{frame}{Newton-multigrid for VIs: semi-smooth}

xxx

semi-smooth Newton with penalty

scaling also restores optimal

performance (Farrell et al.~2020)

FIXME Ulrich (20xx)

\end{frame}


\section{scalable approaches for \emph{nonlocal} VIs}


\begin{frame}{\alert{summary}}

\begin{itemize}
\item x
\end{itemize}
\end{frame}


\begin{frame}{references}

{\scriptsize
%{\notsotiny
% inputed at end of slides.tex

\newcommand{\shref}[2]{\,{\tiny \href{#1}{#2}}}
\begin{itemize}
\item[] A.~Brandt (1977). \emph{Multi-level adaptive solutions to boundary-value problems}, Mathematics of Computation 31 (138), 333--390
\item[] A.~Brandt \& C.~Cryer (1983). \emph{Multigrid algorithms for the solution of linear complementarity problems \dots}, SIAM J.~Sci.~Stat.~Comput.~4 (4), 655--684 \shref{https://doi.org/10.1137/0904046}{doi:10.1137/0904046}
%FTITLE Multigrid algorithms for the solution of linear complementarity problems arising from free boundary problems
\item[] E.~Bueler (2021). \emph{Conservation laws for free-boundary fluid layers}, SIAM J.~Appl.~Math.~81 (5), 2007--2032 \shref{https://doi.org/10.1137/20M135217X}{doi:10.1137/20M135217X}
\item[] E.~Bueler (2021). \emph{PETSc for Partial Differential Equations}, SIAM Press, Philadelphia
%\item[] E.~Bueler (2022). \emph{Performance analysis of high-resolution ice-sheet simulations}, J.~Glaciol., \href{https://doi.org/10.1017/jog.2022.113}{doi:10.1017/jog.2022.113}
\item[] P.~Farrell, M.~Croci, \& T.~Surowiec (2020). \emph{Deflation for semismooth equations}, Optimization Methods \& Software 35 (6), 1248--1271 \shref{https://doi.org/10.1080/10556788.2019.1613655}{doi:10.1080/10556788.2019.1613655}
\item[] C.~Gr{\"a}ser \& R.~Kornhuber (2009). \emph{Multigrid methods for obstacle problems}, J.~Comput.~Math., 1--44
\item[] G. Jouvet \& E. Bueler (2012). \emph{Steady, shallow ice sheets as obstacle problems \dots}, SIAM J.~Appl.~Math.~72 (4), 1292--1314 \shref{https://doi.org/10.1137/110856654}{doi:10.1137/110856654}
%FTITLE Steady, shallow ice sheets as obstacle problems: well-posedness and finite element approximation
\item[] D. Kinderlehrer \& G. Stampacchia (1980). \emph{An Introduction to Variational Inequalities and their Applications}, Academic Press, New York
%\item[] T.~Isaac, G.~Stadler, \& O.~Ghattas (2015). \emph{Solution of nonlinear Stokes equations \dots, with application to ice sheet dynamics}, SIAM J.~Sci.~Comput.~37 (6), B804--B833 \shref{https://doi.org/10.1137/140974407}{doi:10.1137/140974407}
%FTITLE Solution of nonlinear Stokes equations discretized by high-order finite elements on nonconforming and anisotropic meshes, with application to ice sheet dynamics
\item[] U.~Trottenberg, C.~Oosterlee, \& A. Schuller (2001).  \emph{Multigrid}, Elsevier, Oxford
\end{itemize}


}
\end{frame}

\begin{frame}{\emph{background} references}

{\scriptsize
%{\notsotiny
% inputed at end of slides.tex

\newcommand{\sdoi}[1]{\,{\tiny \href{https://doi.org/#1}{doi:#1}}}
\begin{itemize}
\item[] A.~Brandt (1977). \emph{Multi-level adaptive solutions to boundary-value problems}, Mathematics of Computation 31 (138), 333--390
\item[] E.~Bueler (2021). \emph{PETSc for Partial Differential Equations}, SIAM Press, Philadelphia
\item[] N.~Kikuchi \& J.~Oden (1988).  \emph{Contact Problems in Elasticity: A Study of Variational Inequalities and Finite Element Methods}, SIAM Press, Philadelphia
\item[] D.~Kinderlehrer \& G.~Stampacchia (1980). \emph{An Introduction to Variational Inequalities and their Applications}, Academic Press, New York
\item[] U.~Trottenberg, C.~Oosterlee, \& A. Schuller (2001).  \emph{Multigrid}, Elsevier, Oxford
\end{itemize}


}
\end{frame}

\end{document}
